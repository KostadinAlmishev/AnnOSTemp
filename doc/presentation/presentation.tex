\documentclass[12pt]{article}

\usepackage[utf8]{inputenc}
\usepackage[english, russian]{babel}

\usepackage{slides}

% ШРИФТЫ
% Нужны рубленные шрифты -- раскомментируйте стоку ниже.
% Нужны шрифты с засечками --- закомментируйте эту строку.
% \renewcommand{\familydefault}{\sfdefault} % Переключает на рубленный шрифт.
% Шрифты Times и Arial, если стоит пакет cyrtimes.
\IfFileExists{cyrtimes.sty}
    {
        \usepackage{cyrtimespatched}
    }
    {
        % А если Times нету, то будет CM...
    }

% \renewcommand{\seriesdefault}{b} % для шрифта с засечками, это предпочтительно
% \renewcommand{\seriesdefault}{sbc} % для рубленного шрифта

\usepackage{graphicx}

\def\Student{Белоус Вячеслав Сергеевич}
\def\Advisor{Горин Сергей Викторович}

% Титул
\def\Title{Выпускная квалификационная работа \\ специалиста на тему:}
\def\FooterTitle{}
\def\SubTitle{Разработка учебной операционной системы}
\newcommand{\TitleSlide}{
    \addcontentsline{toc}{section}{\Title}%
    ~\vspace{1cm}

    \begin{center}
    {\huge \begin{spacing}{1}\Title\end{spacing}}

    {\SubTitle}
    \end{center}
    \vspace{2cm}

    \begin{flushright}
    \ifthenelse{\isundefined{\Student}}{}
        {\small Студент: \Student\\}
    \ifthenelse{\isundefined{\Advisor}}{}
        {\small Руководитель: \Advisor\\}
    \ifthenelse{\isundefined{\Person}}{}
        {\Person\\}
    \ifthenelse{\isundefined{\Affilation}}{}
        {\Affilation\\}
    \end{flushright}

    \vspace{2cm}

    \begin{center}
        {\tiny Москва, 2016}
    \end{center}

    \thispagestyle{empty}
}


% Верхний заголовок: пустой
% Нижний заголовок по-умолчанию:
\lfoot{} % слева
\cfoot{} % цент пуст
\rfoot{\thepage} % справа

% \renewcommand{\baselinestretch}{1.5}
% \linespread{1.6}

%% Переносы в презентации смотрятся не очень.
\hyphenpenalty 10000
\sloppy

\begin{document}

\TitleSlide

\section{\textbf{Цель и задачи работы}}
\emph{Целью работы} является создание учебной операционной системы под архитектуру AMD64.

\subsubsection{Задачи:}
\begin{enumerate}
\item Анализ архитектуры процессора AMD64 и существующих учебных ОС;
\item Разработка структуры ОС;
\item Разработка загрузчика ОС;
\item Разработка модулей ядра;
\item Определение последовательности проведения лабораторных работ.
\end{enumerate}


\section{\textbf{Основные допущения и ограничения}}
\begin{itemize}
\item Наличие не менее 32 мегабайт доступной физической памяти;
\item Наличие не менее 40 мегабайт доступного дискового пространства;
\item Запуск в эмуляторе QEMU;
\item Отсутствие многопроцессорной обработки;
\item Отсутствие поддержки файловой системы;
\item Запрет прерываний в ядре.
\end{itemize}

\section{\textbf{Структура ОС}}
\begin{center}
\includegraphics[width=1.0\textwidth]{dia/os-structure}
\end{center}


\section{\textbf{Загрузчик ОС}}
\subsubsection{Действия первого загрузчика:}
\begin{enumerate}
\item Открытие линии А20;
\item Определение доступных областей физической памяти;
\item Перевод процессора в защищенный режим;
\item Загрузка и передача управления второму загрузчику.
\end{enumerate}

\subsubsection{Действия второго загрузчика:}
\begin{enumerate}
\item Определение доступного объема физической памяти;
\item Подготовка требуемых отображений и структур данных;
\item Перевод процессора в длинный режим;
\item Загрузка и передача управления ядру.
\end{enumerate}


\section{\textbf{Действия для перехода в длинный режим}}
\begin{enumerate}
\item Активация PAE (CR4.PAE=1);
\item Создание 64-битных дескрипторов кода;
\item Активация длинного режима (EFER.LME=1);
\item Загрузка адреса PML4 в регистр CR3.
\item Активация страничного преобразования (CR0.PG=1).
\end{enumerate}


\section{\textbf{Управление памятью}}
\includegraphics[width=.9\textwidth]{pdf/long-mode-4kb-page-translation}


\section{\textbf{Схема алгоритма создания отображения}}
\begin{center}
\includegraphics[scale=.35]{pdf/page-translation-schema}
\end{center}


\section{\textbf{Схема алгоритма поиска элемента в таблице страниц}}
\begin{center}
\includegraphics[scale=.3]{pdf/page-table-entry-search}
\end{center}


\section{\textbf{Управление процессами}}
\begin{center}
\includegraphics[scale=.7]{pdf/process-context}
\end{center}


\section{\textbf{Дескриптор процесса}}
\begin{center}
\includegraphics[width=.5\textwidth]{pdf/process-descriptor}
\end{center}


\section{\textbf{Состояния процессов}}
\begin{center}
\includegraphics[width=.5\textwidth]{pdf/process-state-diagramm}
\end{center}


\section{\textbf{Системные вызовы}}
\begin{itemize}
\item PUTS -- вывод строки на экран;
\item YIELD -- передача управления следующему процессу;
\item EXIT -- уничтожение процесса;
\item FORK -- создание копии процесса.
\end{itemize}


\section{\textbf{Копирование при записи}}
\includegraphics[width=\textwidth]{pdf/fork-copy-on-write}


\section{\textbf{Тестирование и отладка}}
\begin{itemize}
\item Вывод на экран;
\item Чтение и запись данных в различные области памяти;
\item Создание и уничтожение процессов;
\item Добровольное освобождение процессора;
\item Вытесняющая многозадачность.
\end{itemize}


\section{\textbf{Последовательность проведения лабораторных работ}}
\begin{center}
\includegraphics[width=0.9\textwidth]{dia/lab1}
\end{center}

Лабораторная~№1: начальная загрузка компьютера, устройство GDT, переход в защищенный режим.


\section{\textbf{Последовательность проведения лабораторных работ}}
\begin{center}
\includegraphics[width=0.9\textwidth]{dia/lab2}
\end{center}

Лабораторная~№2: механизм страничного преобразования и переход в длинный режим.

\section{\textbf{Последовательность проведения лабораторных работ}}
\begin{center}
\includegraphics[width=0.9\textwidth]{dia/lab3}
\end{center}

Лабораторная~№3: обработка прерываний, таблица IDT, IST, настройка APIC и IOAPIC.

\section{\textbf{Последовательность проведения лабораторных работ}}
\begin{center}
\includegraphics[width=0.9\textwidth]{dia/lab4}
\end{center}

Лабораторная~№4: Контекст и дескриптор процесса, создание и уничтожение процессов.

\section{\textbf{Последовательность проведения лабораторных работ}}
\begin{center}
\includegraphics[width=0.9\textwidth]{dia/lab5}
\end{center}

Лабораторная~№5: Реализация системных вызовов и механизма копирования при записи.

\section{\textbf{Выводы}}
В результате выполнения дипломной работы была разработана ОС,
имеющая следующие особенности:

\begin{enumerate}
\item Поддержка современной архитектуры AMD64;
\item Двухэтапный загрузчик;
\item Страничное управление памятью;
\item Поддержка прикладных процессов и потоков ядра;
\item Вытесняющая многозадачность;
\item Поддержка механизма копирования при записи.
\end{enumerate}

\end{document}

