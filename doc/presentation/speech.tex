\documentclass[12pt]{article}

\usepackage[utf8]{inputenc}
\usepackage[english, russian]{babel}

\RequirePackage[left=30mm,right=20mm,top=20mm,bottom=20mm,headsep=0pt]{geometry}

\begin{document}

\paragraph{Приветствие (слайд 1).}
Моя работа направлена на разработку учебной операционной системы, которая может
быть использована в курсе <<Проектирование операционных систем>>.

\paragraph{Цели и задачи работы (слайд 2).}
Для выполнения работы необходимо было выполнить следующие задачи: провести
анализ существующих учебных ОС. Разработать структуру ОС и саму ОС. А также определить
последовательность выполнения лабораторных работ.

Следует отметить, что поскольку ОС является учебной, дополнительным требованием
является обеспечение возможности работать с постепенно усложняющимся кодом ОС.


\paragraph{Существующие аналоги (слайд 3).}
На данный момент существует несколько учебных ОС: JOS, xv6 и PhantomEx. На слайде 3
представлены их основные достоинства и недостатки. Как видно из таблицы ни одна из
ОС не поддерживает длинный режим, что делает разработку новой учебной ОС под современную
архитектуру актуальной задачей.

Так как сейчас в курсе ПОС используется JOS, необходимо было сделать чтобы новая ОС
по крайней мере не уступала ей в функционале, т.е. в ней должна быть поддержка
механизма страничного преобразования, поддержка многозадачности и обработка системных
вызовов.


\paragraph{Структура ОС (слайд 4).}
В ходе разработки была получена структура ОС, представленная на слайде 4.
Структура была выбрана таким образом для обеспечения возможности работы с постепенно
усложняющимися версиями ядра. При этом минимально необходимый набор модулей состоит из
1го загрузчика и модуля работы с диском.


\paragraph{Допущения и ограничения (слайд 5).}
На слайде 5 представлены ограничения и допущения, сделанные в работе. Опытным путем было установлено,
что для работы ОС необходимо не менее 32 мегабайт доступной физической памяти и не менее 40 мегабайт
дискового пространства.

В разработанной ОС отсутствует поддержка файловой системы и многопроцессорной обработки.
Для упрощения реализации, код ядра выполняется с отключенными прерываниями.

Для запуска ОС необходимо использовать эмулятор QEMU.


\paragraph{Загрузчик ОС (слайд 6).}
Итак, перейдем к тому, как это все было реализовано. Поскольку размер первого загрузчика
ограничен 510 байтами, было решено для загрузки ОС использовать 2 загрузчика.

Первый загрузчик открывает линию А20, определяет доступные физические области, используя
прерывания BIOS, переходит в защищенный режим, загружает и передает управление второму
загрузчику.

Второй загрузчик определяет доступный объем физической памяти, используя информацию, полученную
первым загрузчиком, подготавливает требуемые для работы ядра отображения и структуры данных:
GDT, PML4, массив дескрипторов страниц. Переходит в длинный режим, загружает и передает
управление ядру.


\paragraph{Управление памятью (слайд 7).}
Одной из задач ОС является выделение процессам отдельных адресных пространств, защищенных
друг от друга. В разработанной ОС для этого используется страничное преобразование,
поскольку сегментное преобразование в длинном режиме отключено.

В длинном режиме механизм страничного преобразования использует 4-уровневую иерархию
таблиц страниц. Для работы прикладных процессов эту иерархию необходимо заполнить,
отобразив виртуальные адреса процессов на физические, в которых находятся данные этих
процессов. Как происходит отображение? Виртуальный адрес делится на несколько частей,
4 из которых используются как индексы в таблицах страниц.... (тут показываю как это происходит).

\paragraph{Схема алгоритма поиска элемента в таблице страниц (слайд 8).}
Схема алгоритма поиска очередного элемента в иерархии приведена на слайде 8.

\paragraph{Схема алгоритма создания отображения (слайд 9).}
Схема алгоритма создания отображения виртуального адреса в физический приведена
на слайде 9.


\paragraph{Контекст процесса (слайд 10).}
Еще одной задачей ОС является управление процессами. Для реализации многозадачности с
использованием одного ядра ЦП необходимо иметь возможность приостанавливать и продолжать процессы.

Для этого в ОС добавлено понятие контекста процесса. Формат которого показан на слайде 10.
Он выбран таким образом, чтобы часть структуры контекста заполнялась естественным образом при
возникновении прерываний, поскольку в этом случае процессор кладет на стек вот эти данные (показать).

\paragraph{Дескриптор процесса (слайд 11).}
Однако, одного контекста процесса недостаточно, чтобы ОС могла определить что сейчас происходит
с процессом, было введено понятие дескриптора процесса, структура которого показана на слайде 11.
При управлении процессами ОС оперирует именно дескрипторами процессов.

\paragraph{Состояние процессов (слайд 12).}
В разработанной ОС дескриптор процесса может находится в одном из 4х состояний.
На слайде 12 показана диаграмма переходов состояний процессов.


\paragraph{Системные вызовы (слайд 13).}
В разработанной ОС реализованы 4 системных вызова: PUTS - выводит на экран строку.
YIELD - вызывает планировщик, для передачи управления следующему процессу в очереди.
EXIT - уничтожает процесс, FORK - создает копию процесса.


\paragraph{Копирование при записи (слайд 14).}
Наиболее интересным является системный вызов FORK, при выполнении которого создается новый процесс (процесс-потомок).

Процесс потомок полностью идентичен процессу-родителю, за исключением одного момента: системный вызов
возвращает 0 в процессе-потомке и идентификатор процесса, т.е. отличное от 0 значение, в процессе-родителе.
При этом все доступные для записи страницы помечаются как доступные только для чтения.

При попытке записи одним из процессов в страницу доступную только для чтения произойдет страничное исключение.
Обработчик страничного исключения, выделит новую страницу, скопирует в нее содержимое оригинальной страницы
и отобразит ее вместо старой.

Данный процесс схематично показан на слайде 14.


\paragraph{Тестирование и отладка (слайд 15).}
Для тестирования ядра ОС написаны прикладные программы, которые позволяют проверить функции вывода данных на экран;
ограничения чтения и записи данных в область ядра и неотображенные области; функции создания и уничтожения процессов;
и работу вытесняющей многозадачности.


\paragraph{Последовательность проведения лабораторных работ.}
А теперь я расскажу о том, как можно использовать данную ОС при проведении лабораторных работ.
Предполагается что ОС будет разделена на несколько модулей и в каждой лабораторной работе необходимо
будет изучить один или несколько модулей.

Разделение можно было произвести 2мя способами: используя ветки в системе контроля версии и используя
директивы условной компиляции. Второй способ позволяет упростить внесение исправлений в ядро, но
ухудшает читаемость кода, поэтому был выбран первый способ, с использованием веток, поскольку в этом
случае можно избавиться от лишнего кода.

Итак, лабораторная №1. Изучается устройство GDT и процесс перехода в защищенный режим.

Далее, лабораторная №2. Здесь студенты знакомятся с механизмом работы страничного преобразования для загрузки
ядра в оперативную память и процессом перехода в длинный режим.

В лабораторной №3 студенты познакомятся с таблицей IDT, новым механизмом переключения стека -- IST
и узнают как настраивать APIC и IOAPIC.

В лабораторной №4 будет введено понятие контекста и дескриптора процесса. Изучены функции создания,
уничтожения и переключения процессов.

Лабораторная №5. В заключении предлагается реализовать несколько обработчиков системных вызовов,
используя механизм обработки прерываний. И реализовать механизм копирования при записи.


\paragraph{Выводы (слайд 21).}
Итак, в результате выполнения работы был проведен анализ существующих ОС, разработана структура ОС,
определена последовательность проведения лабораторных работ и разработана учебная ОС, имеющая следующие свойства:
поддержка современной архитектуры процессора; двухэтапная загрузка; страничное управление памятью; поддержка
прикладных процессов и потоков ядра; вытесняющая многозадачность; поддержка механизма копирования при записи.

Дальнейшим развитием системы может быть поддержка многопроцессорной обработки, средств синхронизации процессов,
реализация файловой системы и поддержка дополнительного оборудования, например, сетевой карты.

Доклад окончен. Готов ответить на ваши вопросы.

\end{document}
