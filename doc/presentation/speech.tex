\documentclass[12pt]{article}

\usepackage[utf8]{inputenc}
\usepackage[english, russian]{babel}

\RequirePackage[left=30mm,right=20mm,top=20mm,bottom=20mm,headsep=0pt]{geometry}

\begin{document}

\paragraph{Приветствие (слайд 1).}
Здравствуйте, уважаемые члены комиссии.


\paragraph{Цели и задачи работы (слайд 2).}
Моя работа направлена на разработку учебной операционной системы, которая может
быть использована в курсе <<Проектирование операционных систем>>.

Для выполнения работы необходимо было выполнить следующие задачи: провести
анализ существующих учебных ОС, на основе которого разработать
структуру ОС, после чего разработать саму ОС. А также определить
последовательность выполнения лабораторных работ.

Следует отметить, что поскольку ОС является учебной, дополнительным требованием
является возможность работать с постепенно усложняющимся кодом ОС.


\paragraph{Существующие аналоги (слайд 3).}
На данный момент существует несколько учебных ОС: JOS, xv6 и PhantomEx. Первые две
используются в курсе MIT, а последняя упоминается на некоторых ресурсах, посвященных
разработке ОС. Они имеют различные достоинства и недостатки, которые перечислены в РПЗ,
однако, все имеют один существенный недостаток: они разработаны под устаревшую архитектуру x86.
Что делает разработку учебной ОС под современную архитектуру актуальной задачей.


\paragraph{Допущения и ограничения (слайд 4).}
На слайде 4 представлены ограничения и допущения, сделанные в работе: для ОС необходимо
не менее 32 мегабайт доступной физической памяти и не менее 40 мегабайт дискового пространства.

В разработанной ОС отсутствует поддержка файловой системы и многопроцессорной обработки.
Для упрощения реализации, код ядра выполняется с отключенными прерываниями.

Для запуска ОС необходимо использовать эмулятор QEMU.


\paragraph{Структура ОС (слайд 5).}
На слайде 4 представлена структура ОС. Основными компонентами являются загрузчик и ядро.
Рассмотрим их более подробно.


\paragraph{Загрузчик ОС (слайд 6).}
Для загрузки ОС используется два загрузчика.

Первый загрузчик открывает линию А20, определяет доступные физические области, используя
прерывания BIOS, переходит в защищенный режим, загружает и передает управление второму
загрузчику.

Второй загрузчик определяет доступный объем физической памяти, используя информацию, полученную
первым загрузчиком, подготавливает требуемые для работы ядра отображения и структуры данных:
GDT, PML4, массив дескрипторов страниц. Переходит в длинный режим, загружает и передает
управление ядру.


\paragraph{Управление памятью (слайд 7).}
Одной из задач ОС является управление памятью. В частности, выделение процессам отдельных
адресных пространств, защищенных от других процессов. Для этого используется 4-уровневая
иерархия таблиц страниц, при этом виртуальный адрес делится на 6 частей, 4 из которых
используются в качестве индексов в таблицах страниц.

\paragraph{Схема алгоритма создания отображения (слайд 8).}
Для создания отображения виртуальных адресов в определенные физические необходимо заполнить
таблицы страниц. На слайде 8 показана схема алгоритма создания отображения.

\paragraph{Схема алгоритма поиска элемента в таблице страниц (слайд 9).}
При создании отображения необходимо выполнять поиск элемента, указывающего на следующий элемент
в иерархии таблиц страниц. На слайде 9 показана схема алгоритма поиска элемента в таблице страниц.


\paragraph{Контекст процесса (слайд 10).}
Еще одной задачей ОС является управление процессами. Для реализации многозадачности с
использованием одного ядра ЦП необходимо иметь возможность приостанавливать и продолжать процессы.

Для этого необходимо хранить контекст процесса (значения регистров). На слайде 10 показан формат
дескриптора процесса. Формат выбран таким способом, чтобы он заполнялся естественным образом при
возникновении прерываний.

\paragraph{Дескриптор процесса (слайд 11).}
Однако, для управления процессами одного контекста процесса недостаточно, поэтому каждый
процесс имеет дескриптор процесса, который включает: контекст процесса, имя и идентификатор процесса,
текущее состояние, и виртуальный адрес PML4.

\paragraph{Состояние процессов (слайд 12).}
На слайде 12 показана диаграмма переходов состояний процессов.

\paragraph{Системные вызовы (слайд 13).}
Для работы прикладных программ, как правило, необходим доступ к различным сервисам ядра: вывод
на экран, выделение памяти, создание процессов. Однако, давать прикладным процессам прямой доступ
к данным возможностям не безопасно, т.к. ошибка в прикладном процессе может привести к полной
неработоспособности всего ядра. Поэтому для доступа к сервисам ядра используются системные вызовы.

В разработанной ОС реализованы 4 системных вызова: PUTS - выводит на экран строку.
YIELD - вызывает планировщик, для передачи управления следующему процессу в очереди.
EXIT - уничтожает процесс, FORK - создает копию процесса.

\paragraph{Копирование при записи (слайд 14).}
Наиболее интересным является системный вызов FORK, при выполнении которого создается новый процесс (процесс-потомок).

Процесс потомок полностью идентичен процессу-родителю, за исключением одного момента: системный вызов
возвращает 0 в процессе-потомке и идентификатор процесса, т.е. отличное от 0 значение, в процессе-родителе.
При этом все доступные для записи страницы помечаются как доступные только для чтения.

При попытке записи одним из процессов в страницу доступную только для чтения произойдет страничное исключение.
Обработчик страничного исключения, выделит новую страницу, скопирует в нее содержимое оригинальной страницы
и отобразит ее вместо старой.

Данный процесс схематично показан на слайде 14.


\paragraph{Тестирование и отладка (слайд 15).}
Для тестирования ядра ОС написаны прикладные программы, которые позволяют проверить функции вывода данных на экран;
ограничения чтения и записи данных в область ядра и неотображенные области; функции создания и уничтожения процессов;
и работу вытесняющей многозадачности.


\paragraph{Последовательность проведения лабораторных работ.}
А теперь я расскажу о том, как можно использовать данную ОС при проведении лабораторных работ.
Предполагается что ОС будет разделена на несколько модулей и в каждой лабораторной работе необходимо
будет изучить один или несколько модулей.

Итак, лабораторная №1. Изучается устройство GDT и процесс перехода в защищенный режим.

Далее, лабораторная №2. Здесь студенты знакомятся с механизмом работы страничного преобразования для загрузки
ядра в оперативную память и процессом перехода в длинный режим.

В лабораторной №3 студенты познакомятся с таблицей IDT, новым механизмом переключения стека -- IST
и узнают как настраивать APIC и IOAPIC.

В лабораторной №4 будет введено понятие контекста и дескриптора процесса. Изучены функции создания,
уничтожения и переключения процессов.

Лабораторная №5. В заключении предлагается реализовать несколько обработчиков системных вызовов,
используя механизм обработки прерываний. И реализовать механизм копирования при записи.


\paragraph{Выводы (слайд 21).}
Итак, в результате выполнения работы был проведен анализ существующих ОС, разработана структура ОС,
определена последовательность проведения лабораторных работ и разработана учебная ОС, имеющая следующие свойства:
\begin{enumerate}
\item поддержка современной архитектуры процессора;
\item двухэтапная загрузка;
\item страничное управление памятью;
\item поддержка прикладных процессов и потоков ядра;
\item вытесняющая многозадачность;
\item поддержка механизма копирования при записи.
\end{enumerate}

Дальнейшим развитием системы может быть поддержка многопроцессорной обработки, средств синхронизации процессов,
реализация файловой системы и поддержка дополнительного оборудования, например, сетевой карты.

Доклад окончен. Готов ответить на ваши вопросы.

\end{document}
