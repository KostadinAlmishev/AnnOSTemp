\chapter{Организационно-экономический раздел}
\label{cha:econom}

\section{Организация и планирование процесса разработки}
Этап организации и планирования разработки программного продукта в рамках
использования традиционного метода планирования представляет собой разбиение
процесса на следующие работы.
\begin{enumerate}[1.]
\item формирование состава выполняемых работ и группировка их по стадиям
разработки;
\item расчет трудоемкости выполнения работ;
\item определение профессионального состава и расчет количества исполнителей;
\item определение затрат времени на выполнение каждого этапа работ;
\item построение календарного графика выполнения разработки;
\item расчет затрат на разработку программного продукта;
\item определение экономической целесообразности;
\end{enumerate}

\section{Определение этапов работ}
Планирование длительности этапов и содержания работ осуществляется в
соответствии с ЕСПД ГОСТ 19.102-77 и предполагает распределение работ по
следующим этапам:
\begin{enumerate}
\item Техническое задание
	\begin{itemize}
	\item обоснование необходимости разработки программы;
	\item научно-исследовательские работы;
	\item разработка и утверждение технического задания;
	\end{itemize}
\item Эскизный проект
	\begin{itemize}
	\item разработка эскизного проекта;
	\item утверждение эскизного проекта;
	\end{itemize}
\item Технический проект
	\begin{itemize}
	\item разработка технического проекта;
	\item утверждение технического проекта;
	\end{itemize}
\item Рабочий проект
	\begin{itemize}
	\item разработка программы;
	\item разработка программной документации;
	\item испытания программы;
	\end{itemize}
\item Внедрение
	\begin{itemize}
	\item подготовка и передача программы.
	\end{itemize}
\end{enumerate}

Анализируя требования ГОСТ, можно предложить следующее разделение работ по
этапам:
\begin{enumerate}
\item Разработка технических требований, предъявляемых к разрабатываемому ПО и
проведение исследований заданной области.
\item Разработка алгоритмов работы ПО, определение средств разработки (языки
		программирования, средства отладки).
\item Разработка программных модулей.
\item Тестирование и отладка разрабатываемого ПО.
\item Разработка документации.
\end{enumerate}

Этап внедрения отсутствует, так как разрабатываемое ПО будет внедряться силами
заказчиков данного ПО без участия разработчиков.

\section{Рассчет трудоемкости}
Определим вероятные трудозатраты на выполнение данного проекта. Определим их с
помощью экспертных оценок. Для этой цели было опрошено четыре
эксперта-разработчика, результаты их оценок приведены в
таблице~\ref{tab:expert_marks}.

\begin{table}[ht!]
  \centering
  \caption{Результаты экспертных оценок}
  \label{tab:expert_marks}
  \begin{tabular}{|c|c|}
    \hline
    Эксперт & Оценка, час \\
    \hline
    1 & 592 \\
    \hline
    2 & 724 \\
    \hline
    3 & 620 \\
    \hline
    4 & 700 \\
    \hline
  \end{tabular}
\end{table}

Общие затраты труда на разработку проекта вычисляются по формуле~\ref{equ:total_costs},
\begin{equation}
	Q_{p} = \sum_{i} t_{i}
\label{equ:total_costs}
\end{equation}

где $ t_{i} $ - затраты труда на выполнение $ i - \textup{й} $ работы.

Используя метод экспертных оценок, вычислим ожидаемую продолжительность работ
$T$ каждого этапа по формуле~\ref{equ:expected_costs},
\begin{equation}
	T = \frac{3 \cdot T_{min} + 2 \cdot T_{max}}{5}
	  = \frac{3 \cdot 592 + 2 \cdot 724}{5} = 592
\label{equ:expected_costs}
\end{equation}

где $ T_{max} $ и $ T_{min} $ -- максимальная и минимальная
продолжительность работы. Они назначаются в соответствии с экспертными
оценками, а ожидаемая продолжительность работы рассчитывается как
математическое ожидание для $\beta$-распределения.

Полный перечень работ с разделением их по этапам приведен в табилце~\ref{tab:work_list}.

\begin{center}
\begin{longtable}{|c|c|c|c|c|c|c|}
    \caption{Перечень работ}
    \label{tab:work_list}
    \\ \hline
    Этап & \thead{№ \\ работы} & \thead{Содержание \\ работы} & \thead{$T_{min}$, \\ $\textup{чел}/\textup{часы}$} &
    \thead{$T_{max}$, \\ $\textup{чел}/\textup{часы}$} & \thead{$T$, \\ $\textup{чел}/\textup{часы}$} & \thead{$T$, \\ $\textup{чел}/\textup{дни}$} \\
    \hline \endfirsthead
    \subcaption{Продолжение таблицы~\ref{tab:work_list}}
    \\ \hline \endhead
    \hline \subcaption{Продолжение на след. стр.}
    \endfoot
    \hline \endlastfoot
    \multirow{3}{*}{\makecell{Разработка\\ технических \\требований}}
	& 1 & \makecell{Получение задания, \\ анализ полученных \\ требований к \\разрабатываемому ПО} & 8 & 8 & 8 & 1 \\
	\cline{2-7}
	& 2 & \makecell{Разработка и \\ утверждение ТЗ} & 32 & 52 & 40 & 5 \\
	\cline{2-7}
	& 3 & \makecell{Анализ предметной \\ области и \\ существующих решений} & 16 & 36 & 24 & 3 \\
    \hline
    \multirow{3}{*}{\makecell{Разработка \\ алгоритмов}}
	& 4 & \makecell{Определение \\ структуры ОС} & 32 & 52 & 40 & 5 \\
	\cline{2-7}
	& 5 & \makecell{Разработка модели \\ памяти ОС} & 24 & 44 & 32 & 4 \\
    \hline
    \multirow{5}{*}{\makecell{Разработка \\ модулей}}
	& 6 & \makecell{Реализация \\ загрузчика ОС} & 16 & 36 & 24 & 3 \\
	\cline{2-7}
	& 7 & \makecell{Реализация перехода \\ в <<длинный>> режим} & 16 & 36 & 24 & 3 \\
	\cline{2-7}
	& 8 & \makecell{Настройка прерываний} & 32 & 72 & 48 & 6 \\
	\cline{2-7}
	& 9 & \makecell{Разработка драйверов} & 32 & 52 & 40 & 5 \\
	\cline{2-7}
	& 10 & \makecell{Организация \\ многозадачности} & 52 & 62 & 56 & 7 \\
    \hline
    \multirow{3}{*}{\makecell{Тестирование \\ и отладка ПО}}
	& 11 & \makecell{Тестирование ПО} & 32 & 72 & 48 & 6 \\
	\cline{2-7}
	& 12 & \makecell{Внесение \\ изменений в ПО} & 32 & 52 & 40 & 5 \\
    \hline
    \multirow{3}{*}{\makecell{Разработка \\ документации}}
	& 13 & \makecell{Разработка программной \\ документации} & 128 & 148 & 136 & 17 \\
	\cline{2-7}
	& 14 & \makecell{Разработка методических \\ указаний} & 28 & 38 & 32 & 4 \\
    \hline
    \multicolumn{5}{|c|}{Итого $Q_{p}$} & 592 & 74 \\
    \hline
\end{longtable}
\end{center}
$Q_{p} = Q_{\textup{ож}}
	= 74 \textup{чел}/\textup{дней}
	= 592 \textup{чел}/\textup{час}.$

\section{Определение численности исполнителей}
Средняя численность исполнителей определяется по формуле~\ref{equ:workers_count},
\begin{equation}
	N = \frac{Q_{p}}{F}
\label{equ:workers_count}
\end{equation}

где $ F $ - фонд рабочего времени определяется по формуле~\ref{equ:time_fond},
\begin{equation}
	F = \sum^{r}_{i = 1} F_{M_{i}} = \sum^{r}_{i = 1} (D_{\textup{О}} - D_{\textup{В}} - D_{\textup{П}} )
\label{equ:time_fond}
\end{equation}

где $ F_{M_{i}} $ - фонд времени в текущем (i-ом) месяце и вычисляется для
каждого месяца с учетом общего количества дней $D_{\textup{О}}$, выходных
$D_{\textup{В}}$ и праздничных $D_{\textup{П}}$ дней.

На реализацию проекта заказчиком отведено $r = 3$ месяца рабочего времени при
односменной работе с продолжительностью рабочего дня 8 часов. В таблице~\ref{tab:month_time_fond}
приведены сведения, необходимые для вычисления фонда времени для каждого
месяца и итоговые результаты вычиселений.

\begin{table}[ht!]
  \centering
  \caption{Месячный фонд времени}
  \label{tab:month_time_fond}
  \begin{tabular}{|c|c|c|c|c|c|}
    \hline
	Месяц & \thead{Количество \\ дней} & \thead{Количество \\ выходных \\ дней} &
		\thead{Количество \\ праздничных \\ дней} & \thead{Фонд \\ времени \\ $F_{M_{i}}$, дни} &
		\thead{Фонд \\ времени \\ $F_{M_{i}}$, часы} \\
    \hline
	Февраль & 29 & 8 & 1 & 20 & 160 \\
    \hline
	Март & 31 & 9 & 1 & 21 & 168 \\
    \hline
	Апрель & 30 & 8 & 1 & 21 & 168 \\
    \hline
	\multicolumn{4}{|c|}{Итого $F$} & 62 & 496 \\
    \hline
  \end{tabular}
\end{table}

Таким образом, фонд рабочего времени проекта составляет $F = 504$ часов.
Отсюда среднюю численность исполнителей можно вычислить по формуле~\ref{equ:work_time_fond}:
\begin{equation}
	N = \frac{Q_{p}}{F} = \frac{554}{496} = 1.12
\label{equ:work_time_fond}
\end{equation}

Таким образом, по суммарным трудозатратам для завершения проекта в заданные
сроки необходимо два исполнителя: инженер и программист. Инженер необходим для
предварительного проектирования системы и написания документации, программист
для реализации модулей и проведения тестирования.

Учитывая специфику данного проекта, данные специалисты должны обладать
знаниями в следующих областях:
\begin{itemize}
\item разработка алгоритмов и архитектуры ПО
\item проектирование операционных систем
\item системное программирование
\item архитектура ЭВМ
\item тестирование ПО
\item разработка программной документации
\end{itemize}

\section{Построение сетевого графика}
Для определения временных затрат и трудоемкости разработки ПО воспользуемся
методом сетевого планирования. Метод сетевого планирования позволяет
установить единой схемой связь между всеми работами в виде сетевого графика,
представляющего собой информационно-динамическую модель, позволяющую
определить продолжительность и трудоемкость, как отдельных этапов, так и всего
комплекса работ в целом.

Составление сетевой модели включает в себя оценку степени детализации
комплекса работ и определение логической связи между отдельными работами. С
этой целью составляется перечень всех основных работ и событий. В перечне
указываются кодовые номера событий, наименования событий в последовательности
от исходного к завершающему, кодовые номера работ, перечень всех работ, причем
подряд указываются все работы, которые начинаются после наступления данного
события.

Основные события и работы проекта представлены в таблице~\ref{tab:events_and_tasks}.

\begin{center}
\begin{longtable}{|c|c|c|c|c|c|}
    \caption{Основные работы и события проекта}
    \label{tab:events_and_tasks}
    \\ \hline
	$N_{i}$ & Наименование события & \thead{Код \\ работы} & Работа & \thead{t, \\ чел/часы} & \thead{t, \\ чел/день} \\
    \hline
    \endfirsthead
	\subcaption{Таблица~\ref{tab:events_and_tasks} -- Основные работы и события проекта (продолжение)} \\
    \hline
	$N_{i}$ & Наименование события & \thead{Код \\ работы} & Работа & \thead{t, \\ чел/часы} & \thead{t, \\ чел/день} \\
    \hline
	\endhead
    \hline
	\subcaption{Продолжение на след. стр.}
    \endfoot
    \hline \endlastfoot
	0 & \makecell{Разработка ПО \\ начата} & 0-1 & \makecell{Получение задания, \\ анализ полученных \\ требований к \\разрабатываемому ПО} & 8 & 1 \\
    \hline
	1 & \makecell{Задание получено, \\ анализ полученных \\ требований проведен} & 1-2 & \makecell{Разработка и \\ утверждение ТЗ} & 40 & 5 \\
    \hline
	2 & \makecell{ТЗ разработано \\ и утверждено} & 2-3 & \makecell{Анализ предметной \\ области и \\ существующих решений} & 24 & 3 \\
    \hline
	3 & \makecell{Анализ предметной \\ области и существующих \\ решений проведен} & 3-4 & \makecell{Разработка \\ структуры ОС} & 40 & 5 \\
    \hline
    \multirow{2}{*}{4} & \multirow{2}{*}{\makecell{Разработка структуры \\ ОС завершена}}
		  & 4-6 & \makecell{Разработка модели \\ памяти ОС} & 32 & 4 \\
		\cline{3-6}
		& & 4-5 & \makecell{Реализация \\ загрузчика ОС} & 24 & 3 \\
    \hline
	5 & \makecell{Реализация \\ загрузчика ОС \\ завершена} & 5-6 & \makecell{Фиктивная работа} & 0 & 0 \\
    \hline
	6 & \makecell{Разработка модели \\ памяти ОС \\ завершена} & 6-7 & \makecell{Реализация перехода \\ в <<длинный>> режим} & 24 & 3 \\
    \hline
	7 & \makecell{Выполнен переход \\ в <<длинный>> режим} & 7-8 & \makecell{Настройка прерываний} & 48 & 6 \\
    \hline
    \multirow{2}{*}{8} & \multirow{2}{*}{\makecell{Настройка прерываний \\ завершена}}
		  & 8-9 & \makecell{Разработка драйверов} & 40 & 5 \\
		\cline{3-6}
		& & 8-10 & \makecell{Организация \\ многозадачности} & 56 & 7 \\
    \hline
	9 & \makecell{Разработка драйверов \\ завершена} & 9-10 & \makecell{Фиктивная работа} & 0 & 0 \\
    \hline
	10 & \makecell{Реализована поддержка \\ многозадачности} & 10-11 & \makecell{Тестирование ПО} & 48 & 6 \\
    \hline
	11 & \makecell{Тестирование ПО \\ завершено} & 11-12 & \makecell{Внесение \\ изменений в ПО} & 40 & 5 \\
    \hline
	12 & \makecell{Внесение изменений в ПО \\ завершено} & 12-13 & \makecell{Разработка программной \\ документации} & 136 & 17 \\
    \hline
	13 & \makecell{Разработка программной \\ документации завершена} & 13-14 & \makecell{Разработка методических \\ указаний} & 32 & 4 \\
    \hline
	14 & \makecell{Разработка методических \\ указаний завершена} & 14-15 & \makecell{Фиктивная работа} & 0 & 0 \\
    \hline
	15 & \makecell{Разработка ПО завершена} & - & - & - & - \\
    \hline
\end{longtable}
\end{center}

Ранний срок совершения события $T_{j}^{p}$ определяет минимальное время, необходимое для выполнения всех работ,
предшествующих данному событию и равен продолжительности наибольшего из путей, ведущих от исходного события
к рассматриваемому: $T_{j}^{p} = \max{(T_{i}^{p} + t_{i-j})} $

Поздний срок завершения события $T_{j}^{\textup{п}}$ - это максимально допустимое время наступления данного события,
при котором сохраняется возможность соблюдения ранних сроков наступления последующих событий. Поздние сроки равны
разности между поздним сроком завершения события $j$ и продолжительностью работы $i-j$:
$T_{j}^{\textup{п}} = \min{( T_{i}^{\textup{п}} - t_{i-j} )}$

Критический путь -- это максимальный путь от исходного события до завершения проекта. Его определение позволяет
обратить внимание на перечень событий, савокупность которых имеет нулевой резерв времени.

Все события в сети, не принадлежащие критическому пути, имеют резерв времени $R_{i}$, показывающий, на какой предельный
срок можно задержать наступление этого события, не увеличивая сроки окончания работ: $R_{i} = T_{i}^{\textup{п}} - T_{i}^{p}$.

Полный резерв времени работы $R_{i-j}^{\textup{п}}$ и свободный резерв времени работы $R_{i-j}^{\textup{с}}$ можно
определять, используя следующие соотношения:

$R_{i-j}^{\textup{п}} = T_{j}^{\textup{п}} - T_{i}^{\textup{р}} - t_{i-j}$ и
$R_{i-j}^{\textup{с}} = T_{j}^{\textup{р}} - T_{i}^{\textup{р}} - t_{i-j}$.

Полный резерв работы показывает максимальное время, на которое можно увеличить длительность работы или
отсрочить ее начало, чтобы не нарушился срок завершения проекта в целом. Свободный резерв работы
показывает максимальное время, на которое можно увеличить продолжительность работы или отсрочить ее начало,
не меняя ранних сроков начала последующих работ. Сетевой график приведен на рисунке~\ref{fig:network-graph}.

\begin{figure}[ht!]
  \centering
  \includegraphics[width=\textwidth]{inc/dia/network-graph}
  \caption{Сетевой график}
  \label{fig:network-graph}
\end{figure}

\section{Диаграмма Гантта}
Для иллюстрации последовательности проводимых работ на календарном графике работ приведем диаграмму Гантта
данного проекта, на оси $x$ которой расположены календарные дни от начала проект, а по оси $y$ - выполняемые
этапы работ. Диаграмма Гантта приведена на рис.~\ref{fig:gantt}.
Занятость исполнителей приведена в таблице~\ref{tab:busyness}.

\begin{figure}[ht!]
  \begin{center}
  \includegraphics[width=\textwidth]{inc/gnuplot/gantt}
  \end{center}
  \caption{Диаграмма Гантта}
  \label{fig:gantt}
\end{figure}

\begin{table}[ht!]
  \caption{Занятость исполнителей}
  \label{tab:busyness}

  \centering
  \begin{tabular}{|c|c|c|c|}
    \hline
    Код работы & Дата начала & Дата окончания & Исполнитель \\
    \hline
    0-1   & 01.02.2016 & 01.02.2016 & Инженер \\
    \hline
    1-2   & 02.02.2016 & 08.02.2016 & Инженер \\
    \hline
    2-3   & 09.02.2016 & 11.02.2016 & Инженер \\
    \hline
    3-4   & 12.02.2016 & 18.02.2016 & Инженер \\
    \hline
    4-5   & 19.02.2016 & 24.02.2016 & Программист \\
    \hline
    4-6   & 19.02.2016 & 25.02.2016 & Инженер \\
    \hline
    6-7   & 26.02.2016 & 01.03.2016 & Программист \\
    \hline
    7-8   & 02.03.2016 & 10.03.2016 & Программист \\
    \hline
    8-9   & 11.03.2016 & 17.03.2016 & Программист \\
    \hline
    8-10  & 11.03.2016 & 21.03.2016 & Инженер \\
    \hline
    10-11 & 22.03.2016 & 29.03.2016 & Инженер \\
    \hline
    11-12 & 30.03.2016 & 05.04.2016 & Программист \\
    \hline
    12-13 & 06.04.2016 & 28.04.2016 & Инженер \\
    \hline
    13-14 & 29.04.2016 & 04.05.2016 & Инженер \\
    \hline
  \end{tabular}
\end{table}


\section{Анализ структуры затрат проекта}
Затраты на выполнение проекта можно вычислить по формуле~\ref{equ:cost},
\begin{equation}
	K = C_{\textup{зарп}} + C_{\textup{об}} + C_{\textup{орг}} + C_{\textup{накл}}
\label{equ:cost}
\end{equation}

где $C_{\textup{зарп}}$ - заработная плата исполнителей,
$C_{\textup{об}}$ - затраты на обеспечение необходимым оборудованием,
$C_{\textup{орг}}$ - затраты на организацию рабочих мест,
$C_{\textup{накл}}$ - накладные расходы.


\subsection{Затраты на выплату исполнителям заработной платы}
Затраты на выплату исполнителям заработной платы можно вычислить по формуле~\ref{equ:salary},
\begin{equation}
	C_{\textup{зарп}} = C_{\textup{з.осн}} + C_{\textup{з.доп}} + C_{\textup{з.отч}}
\label{equ:salary}
\end{equation}

где $C_{\textup{з.осн}}$ - основная заработная плата, $C_{\textup{з.доп}}$ - дополнительная заработная плата,
$C_{\textup{з.отч}}$ - отчисления с заработной платы.

Рассчет основной заработной платы при дневной оплате труда исполнителей проводится на основе данных
по окладам и графику занятости исполнителей и выполняется по формуле~\ref{equ:main_salary},
\begin{equation}
	C_{\textup{з.осн}} = T_{\textup{зн}} \cdot O_{\textup{дн}}
\label{equ:main_salary}
\end{equation}

где $T_{\textup{зн}}$ - число дней, отработанных исполнителем,
$O_{\textup{дн}}$ - дневной оклад исполнителя.

При 8-ми часовом рабочем дне $O_{\textup{дн}}$ вычисляется по формуле~\ref{equ:day_salary},
\begin{equation}
	O_{\textup{дн}} = \frac{O_{\textup{мес}} \cdot 8}{F_{\textup{м}}}
\label{equ:day_salary}
\end{equation}

где $O_{\textup{мес}}$ - месячный оклад, $F_{\textup{м}}$ - месячный фонд рабочего времени.

С учетом налога на доходы физических лиц размер месячного оклада вычисляется по формуле~\ref{equ:month_salary},
\begin{equation}
	O_{\textup{мес}} = O \cdot (1 + \frac{H_{\textup{дфл}}}{100})
\label{equ:month_salary}
\end{equation}

где $O$ - оклад, который позволит исполнителю заниматься проектом и который получен
из информации кадровых агенств, $H_{\textup{дфл}}$ - налог на доходы физических лиц (13\%).

Расчет заработной платы представлен в таблице~\ref{tab:salary} (размер оклада приведен с учетом налога
на доходы физических лиц). Размер заработной платы был получен на интернет-портале www.hh.ru (12.05.2016).

\begin{table}[ht!]
  \centering
  \caption{Оклад исполнителей}
  \label{tab:salary}
  \begin{tabular}{|c|c|c|c|c|}
    \hline
    Должность & \thead{<<Чистый>> \\ оклад, руб.} & \thead{Дневной \\ оклад, руб.} & \thead{Трудозатраты, \\ чел-день} & \thead{Затраты на \\ заплату, руб.} \\
    \hline
    Программист & 80 000 & 4383.03 & 21 & 92043.63 \\
    \hline
    Инженер & 100 000 & 5478.78 & 49 & 268460.22 \\
    \hline
    \multicolumn{4}{|c|}{Итого $C_{\textup{з.осн}}$} & 360503.85 \\
    \hline
  \end{tabular}
\end{table}

\texttt{Дополнительная заработная плата}. Учитываются все выплаты непосредственным исполнителям за время,
не проработанное на производстве, в том числе: оплата очередных отпусков, компенсации за неиспользованный отпуск и др.
Эти выплаты составляют 20\% от основной заработной платы:
$C_{\textup{з.доп}} = 0.2 \cdot C_{\textup{з.осн}} = 0.2 \cdot 360503.85 = 72100.77 (\textup{руб})$.

Отметим, что командировки в нашей работе не предполагаются. Согласно нормативным документам суммарные отчисления в
пенсионный фонд, фонд социального страхования и фонды обязательного медицинского страхования составляют 30\% от
размеров заработной платы:
$C_{\textup{з.отч}} = (C_{\textup{з.осн}} + C_{\textup{з.доп}}) \cdot 0.3 = (360503.85 + 72100.77) \cdot 0.3 = 129781.39 \textup{руб}$.

Таким образом затраты на выплату исполнителям заработной платы составляют:
$C_{\textup{зарп}} = C_{\textup{з.осн}} + C_{\textup{з.доп}} + C_{\textup{з.отч}}
 = 360503.85 + 72100.77 + 129781.39 = 562386.01 \textup{руб}$.

\subsection{Затраты на обеспечением необходимым оборудованием}
Расчет затрат $C_{\textup{об}}$ на расходные материалы и комплектующие изделия представлен в таблице~\ref{tab:equipment}.

\begin{table}[ht!]
  \centering
  \caption{Затраты на расходные материалы и оборудование}
  \label{tab:equipment}
  \begin{tabular}{|c|c|c|c|c|}
    \hline
    № & Наименование & Количество & Цена за ед., руб. & Сумма, руб. \\
    \hline
    1 & \makecell{Ноутбук Lenovo \\ G50-70 (Celeron 2957U)} & 2 & 18 000 & 36 000 \\
    \hline
    2 & \makecell{Набор Erich Krause \\ шариковые ручки R-301, 10шт} & 1 & 134 & 134 \\
    \hline
    3 & \makecell{Бумага для принтера \\ <<Ballet Classic>>, \\ формат А4, 500 листов} & 1 & 237 & 237 \\
    \hline
    4 & \makecell{Принтер Canon PIXMA iP2840} & 1 & 2 099 & 2 099 \\
    \hline
    \multicolumn{4}{|c|}{Итого $C_{\textup{об}}$} & 38470 \\
    \hline
  \end{tabular}
\end{table}

\subsection{Затраны на организацию рабочих мест}
Расчет затрат, связанных с организацией рабочих мест для исполнителей проекта, следует проводить
ориентируясь на требования санитарных норм и правил и на стоимость аренды помещения требуемого
уровня сервиса. В соответствии с санитарными нормами расстояние между рабочими столами с видеомониторами
должно быть не менее 2м, а между боковыми поверхностями видеомониторов - не менее 1.2м. Площать на одно
рабочее место с терминалом или ПК должна составлять не менее $6\textup{м}^{2}$, а объем - не менее $20\textup{м}^{3}$.
Площадь, предусмотренная для размещения одного принтера, соответствует 0.5 площади рабочего места исполнителя.
Расположение рабочих мест в подвальных помещениях не допускается. Помещения должны быть оборудованы системами
отопления, кондиционирования воздуха или эффективной приточно-вытяжной вентиляцией.

Таким образом, необходимо подобрать помещение для рабочих мест 2х исполнителей и прощади для размещения принтера,
что суммарно составляет не менее $6 \cdot 2 + 6 \cdot 0.5 = 15\textup{м}^2$.

В ходе исследования имеющихся предложений аренды офисных помещений были найдены варианты
размещения рабочего помещения, приведенные в таблице~\ref{tab:office}.

\begin{table}[ht!]
  \centering
  \caption{Офисные помещения}
  \label{tab:office}
  \begin{tabular}{|c|c|c|c|c|}
    \hline
    № & Адрес & Площадь & \thead{Стоимость, \\ $\textup{руб}/\textup{месяц}$} & \thead{Ссылка на сайт \\ агенства недвижимости} \\
    \hline
    1 & \makecell{м. Бауманская, \\ Аптекарский переулок} & 16 & 6500 & \makecell{http://realty.dmir.ru/rent/ \\ ofis-moskva-aptekarskiy- \\ pereulok-116766398/} \\
    \hline
    2 & \makecell{м. Дмитровская, \\ Складочная улица, 1С23} & 25 & 7200 & \makecell{http://realty.dmir.ru/rent/ \\ ofis-moskva-skladochnaya- \\ ulica-141715032/} \\
    \hline
    3 & \makecell{м. Румянцево, \\ шоссе Киевское, с1кБ} & 15 & 8600 & \makecell{http://realty.dmir.ru/rent/ \\ ofis-rumyancevo-kievskoe- \\ shosse-140130033/} \\
    \hline
    4 & \makecell{м. Юго-Западная, \\ Большая Очаковская \\ улица, 47а} & 50 & 8000 & \makecell{http://realty.dmir.ru/rent/ \\ ofis-moskva-bolshaya- \\ ochakovskaya-ulica-138967926/} \\
    \hline
  \end{tabular}
\end{table}

Из таблицы~\ref{tab:office} следует, что наиболее подходящим вариантом является 1 варинат, офисное помещение
площадью $16 \textup{м}^{2}$, располагающееся рядом с метро Бауманская.

Затраты на аренду помещения можно вычислить по формуле~\ref{equ:arenda},
\begin{equation}
	C_{\textup{орг}} = \frac{C_{\textup{квм}}}{12} \cdot S \cdot T_{\textup{ар}}
\label{equ:arenda}
\end{equation}

где, $C_{\textup{квм}}$ - стоимость аренды $1\textup{м}^{2}$ в год, т.о. стоимость аренды офисного помещения составит:
$C_{\textup{орг}} = 6500 \cdot 3 = 19500 \textup{руб}$.


\subsection{Накладные расходы}
Накладные расходы состоят из расходов на производство, управление, техническое обслуживание и прочее.
С учетом минимизации затрат, накладные расходы составляют 60\% от основной заработной платы:
$C_{\textup{накл}} = 0.6 \cdot C_{\textup{з.осн}} = 0.6 \cdot 360503.85 = 216302.31 \textup{руб}$.

\subsection{Суммарные затраты на реализацию программного проекта}
Расчет суммарных затрат на реализацию программного проекта приведен в тиблице~\ref{tab:total_project_cost}.
На рис.~\ref{fig:cost_pie_chart} приведена круговая диаграмма, отражающая структуру затрат проекта.

\begin{table}[ht!]
  \centering
  \caption{Суммарные затраты на реализацию}
  \label{tab:total_project_cost}
  \begin{tabular}{|c|c|c|c|}
    \hline
    № & Статья расходов & Затраты, руб. & \% \\
    \hline
    1 & Заработная плата исполнителям & 562386.01 & 67.2 \\
    \hline
    2 & Закупка и аренда оборудования & 38470 & 5 \\
    \hline
    3 & Организация рабочих мест & 19500 & 2 \\
    \hline
    4 & Накладные расходы & 216302.31 & 25.8 \\
    \hline
    \multicolumn{2}{|c|}{Суммарные затраты} & 836658.32 & 100 \\
    \hline
  \end{tabular}
\end{table}

\begin{figure}[ht!]
  \centering
  \includegraphics[width=0.9\textwidth]{inc/gnuplot/cost_pie_chart}
  \caption{Структура затрат проекта}
  \label{fig:cost_pie_chart}
\end{figure}

Деньги на разработку выделяет ФГБОУ ВПО <<МГТУ им. Н.Э.Баумана>>, являющееся заказчиком данного проекта.


\section{Исследование рынка}
На данный момент существуют следующие операционные системы: JOS и Xv6.
Разработы в MIT для платформы Intel x86. Используются в курсе MIT <<Operating System Engineering>>.

В МГТУ им. Н.Э.Баумана в курсе <<Проектирование операционных систем>> используется
адаптированная версия JOS.

К преимуществам данных систем можно отнести то, что они охватывают многие аспекты разработки операционных
систем: загрузка, виртуальная память, процессы, системные вызовы, файловая система, драйверы устройств и виртуализация.

Основным недостатом является то, что они разработы под устаревшую платформу x86. Поэтому основным преимуществом
операционной системы, разрабатываемой в рамках данного проекта является поддержка современной архитектуры AMD64.

Предварительная оценка стоимости разрабатываемого ПО показала, что цена одного комплекта не будет превышать 25000 рублей.

В разрабатываемой операционной системе в первую очередь заинтересованы высшие учебные заведения, в т.ч. МГТУ им. Н.Э. Баумана.
Поэтому годовое количество инсталяций данной операционной системы составит: $N_{\textup{о}}^{\textup{р}} = 60$ (количество учащихся на потоке).

\section{Планирование цены и прогнозирование прибыли}
Частичная стоимость разработки, приходящаяся на каждый комплект ПО, определяется исходя из
данных о планируемом объеме поставок по формуле~\ref{equ:partial_cost},
\begin{equation}
	\Delta{K} = \frac{K \cdot (1 + H_{\textup{ст}})}{N^{o}_{p}}
\label{equ:partial_cost}
\end{equation}

где $K$ - стоимость проекта, $N^{o}_{p}$ - планируемое число копий ПО, $H_{\textup{ст}}$ - ставка
банковского процента по долгосрочным кредитам (более одного года).

Так как оплата затрат на разработку осуществляется из бюджета МГТУ им. Н.Э.Баумана -- $H_{\textup{ст}} = 0$, т.о.
$\Delta{K} = \frac{836658.32}{60} = 13944.31$~(руб).

По результатам мониторинга рынка установим значение $K_{\textup{пр}} = 25000$~рублей. Тем самым, сумма от продаж
за год составит $60 \cdot 25000 = 1500000$~рублей, что обеспечивает срок окупаемости проекта менее 1 года.
Определим процент прибыли от одной реализации ПО по формуле~\ref{equ:realization_cost},
\begin{equation}
	D_{\textup{приб}} = (\frac{K_{\textup{пр}}}{\Delta{K} + K_{\textup{вн}}} - 1) \cdot 100\%
\label{equ:realization_cost}
\end{equation}

где $K_{\textup{вн}} = 0$ - затраты на внедрение.

Для данного проекта: $D_{\textup{приб}} = (\frac{25000}{13944.31} - 1) \cdot 100\% = 79.28\%$.

Сумма расчетной прибыли от продажи каждой установки ПО с учетом налога на добавочную стоимость $H_{\textup{ндс}} = 18\%$:
$C_{\textup{приб}} = (\Delta{K} + K_{\textup{вн}}) \cdot D_{\textup{приб}} \cdot (1 - H_{\textup{ндс}})
= 13944.31 \cdot 79.28 \cdot (1-0.18) = 9065.14$~(руб).

За первые три месяца разработки продажи равны нулю, т.к. продукт ещё не разработан. При этом осуществляются
выплаты заработной платы и производятся другие, ранее расчитанные расходы на разрботку в размере 836658.32~рублей.

Будем считать, что за каждые последующие месяцы продается 60 экземпляров программы (5 в месяц).

Фрагмент таблицы общего баланса приведен в таблице~\ref{tab:common_balance}, из него видно, что
затраты на разработку окупятся в ноябре 2016 года.

\begin{table}[ht!]
  \centering
  \caption{Результаты экспертных оценок}
  \label{tab:common_balance}
  \begin{tabular}{|c|c|c|c|c|}
    \hline
    \thead{Период расчета} & \thead{Баланс \\ начальный, руб} & \thead{Сумма \\ продаж, руб} & \thead{Чистая \\ прибыль, руб} & \thead{Баланс \\ конечный, руб} \\
    \hline
     02-04.2016 & 836658.32 & 0.00 & -836658.32 & 0.00 \\
    \hline
     05.2016 & 0.00 & 125000.00 & 125000.00 & 125000.00 \\
    \hline
     06.2016 & 125000.00 & 125000.00 & 125000.00 & 250000.00 \\
    \hline
     07.2016 & 250000.00 & 125000.00 & 125000.00 & 375000.00 \\
    \hline
     08.2016 & 375000.00 & 125000.00 & 125000.00 & 500000.00 \\
    \hline
     09.2016 & 500000.00 & 125000.00 & 125000.00 & 625000.00 \\
    \hline
     10.2016 & 625000.00 & 125000.00 & 125000.00 & 750000.00 \\
    \hline
     11.2016 & 750000.00 & 125000.00 & 125000.00 & 875000.00 \\
    \hline
  \end{tabular}
\end{table}


\clearpage
\section{Выводы}
Результаты проведенных организационно-экономических расчетов позволили оценить структуру работ и необходимое
количество исполнителей. Общие затраты труда для выполнения программного проекта составили 74~чел/дня или 592~чел/часа.
Затраты на разработку ПО составляют 836658.32 рублей.

Деньги на оплату расходов на разработку будут выделены из бюджета МГТУ им. Н.Э.Баумна.

Исходя из временных требований к реализации проекта, а также требований к квалификации персонала, была определена
численность исполнителей: 2 человека -- инженер и программист. По результатам построения сетевого графика и
диаграммы Гантта можно сделать вывод о том, что введение дополнительных разработчиков не принесет положительного эффекта.

Из структуры затрат проекта видно, что основной статьей расходов является заработная плата исполнителей - 67.2\%.

Стоимость одной копии продукта составила 25000 рублей, при условии продажи 60 экземпляров в год.
Планирование цены позволило спрогнозировать срок окупаемости проекта в пределах одного года.

Исходя из сказанного, разработка данной операционной системы является экономически целесообразной.
