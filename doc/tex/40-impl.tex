\chapter{Технологический раздел}
\label{cha:impl}

\section{Средства разработки}
Разработка и отладка ОС производилась с использованием системы Gentoo 2.2 (ядро
GNU Linux 4.1.15). Для сборки ядра использовались: GNU Make 4.1, gcc 4.9.3, GNU
Binutils 2.25, GNU Perl 5.20.2, GNU gdb 7.10.1.

Разработка ОС выполнялась на языке Си с ассемблерными вставками. В ассемблерных
вставках используется синтаксис AT\&T, т.к. он является стандартным для средств
разработки GNU и большинства Unix-подобных систем.

\section{Синтаксис AT\&T}
Существует 2 альтернативных синтаксиса ассемблера: AT\&T и Intel. Их основные
различия перечислены ниже.

\paragraph{Порядок операндов.} В синтаксисе AT\&T присваивание выполняется слева
напрво (\textbf{movl \$1, \%eax}). В Intel -- наоборот, справа налево (\textbf{mov eax, 1}).

\paragraph{Суффиксы размерностей операндов.} В AT\&T команды заканчиваются однобуквенным
суффиксом, показывающим разрядность операндов:
\begin{enumerate}[1.]
	\item \textbf{q} -- quad, восемь байт;
	\item \textbf{l} -- long, четыре байта;
	\item \textbf{w} -- word, два байта;
	\item \textbf{b} -- byte, один байт.
\end{enumerate}

В случае Intel -- для обозначения размер операндов используются префиксы: \textbf{byte ptr},
\textbf{word ptr} и т.д.

\paragraph{Префиксы операндов.} В AT\&T операнд дополняется особым символом-префиксом,
указывающим его вид:
\begin{itemize}
\item регистр -- процентом: \%eax;
\item непосредственный операнд -- символом USD: \$1;
\item косвенный адрес перехода -- звездочкой: *addr.
\end{itemize}

Прочие операнды не имеют префикса: \textbf{0x10000} (содержимое памяти, по адресу 0x10000),
\textbf{varname}. В AT\&T \textbf{varname} всегда обозначает ячейку памяти, а \textbf{\$varname} --
ее адрес.

\paragraph{Адресация база-масштаб-смещение.} В синтаксисе Intel сложение базы со смещением
имеет вид: \textbf{[var+eax+2*ebx]}. В синтаксисе AT\&T используется форма \textbf{var(\%eax,2,\%ebx)}. В случае, если какая-то
из частей адреса отсутствует, она пропускается: \textbf{-4(,\%eax,2)} есть \textbf{[2*eax-4]} в нотации Intel.

\section{Тестовое окружение}
При разработке операционной системы, как и при разработке любого другого ПО,
необходима возможность запуска и проверки работоспобности внесенных изменений.
Существует два основных способа запуска ОС: на реальной машине
и на виртуальной машине.

К достоинствам использования виртуальной машины можно отнести: возможность
запуска ОС сразу после компиляции, без необходимости синхронизации изменений
между несколькими компьютерами. Кроме того, некоторые виртуальные машины выводят
отладочную информацию, которая может быть полезна при отладке. Основной недостаток
использования виртуальных машин заключается в том, что они поддерживают ограниченный
набор устройств и могут содержать ошибки.

Основным достоинством тестирования на реальной машине является то, что если
ОС работает на одной машине -- с большой вероятностью она будет работать и
на других машинах такой же архитектуры. В то же время, то, что ОС работает
на виртуальной машине не гарантирует, что она будет работать и на реальной.
Однако, для тестирования на реальной машине необходимо либо тестировать на
той же машине, на которой ведется разработка (отнимает много времени, т.к.
требуются постоянные перезагрузки), либо приобрести вторую машину, что требует
дополнительных расходов.

Поэтому использование виртуальной машины является предпочтительным. В работе используется
эмулятор QEMU 2.5.0, т.к. он имеет поддержку GDB, что позволяет
выполнять отладку на уровне исходных кодов. Кроме того, данный эмулятор имеет встроенный
монитор, который позволяет переключаться между различными CPU, выводить содержимое
регистров, кеша TLB, таблиц страничного преобразования и других системных структур данных.


\section{Организация исходных кодов}
Исходные файлы ОС разделены по следующим каталогам:
\begin{itemize}
\item \textbf{user} -- файлы режима пользователя.
\item \textbf{stdlib} -- файлы стандартной библиотеки, общей для ядра и прикладных программ.
\item \textbf{stdlib/i386} -- место размещения файлов, полученых в ходе компиляции 32-битной
	версии библиотеки.
\item \textbf{stdlib/x86\_64} -- место размещения файлов, полученных в ходе компиляции 64-битной
	версии библиотеки.
\item \textbf{kernel} -- основные файлы ядра ОС.
\item \textbf{kernel/boot} -- файлы первого загрузчика ОС.
\item \textbf{kernel/loader} -- файлы второго загрузчика ОС.
\item \textbf{kernel/interrupt} -- файлы, связанные с обработкой прерываний.
\item \textbf{kernel/lib/console} -- файлы, связанные с выводом на экран.
\item \textbf{kernel/lib/disk} -- файлы, связанные с доступом к диску.
\item \textbf{kernel/lib/memory} -- файлы, связанные с работой с памятью.
\item \textbf{kernel/lib/i386} -- место размещения файлов, полученных в ходе компиляции 32-битной
	версии библиотеки ядра.
\item \textbf{kernel/lib/x86\_64} -- место размещения файлов, полученных в ходе компиляции 64-битной
	версии библиотеки ядра.
\item \textbf{kernel/misc} -- заголовочные файлы, не подходящие под указанные выше категории.
\end{itemize}

\section{Исполняемые форматы}
В качестве формата исполняемых файлов был выбран ELF -- стандартный формат исполняемых файлов
для большинства Unix-подобных систем. Преимуществами данного формата является то, что он хорошо
документирован и абсолютное большинство Unix-подобных систем имеют компилятор и линковщик для
создания ELF-файлов.

Для упрощения работы с ELF-файлами были разработаны макросы: \texttt{ELF32\_PHEADER\_FIRST},
\texttt{ELF32\_PHEADER\_LAST}, \texttt{ELF64\_PHEADER\_FIRST} и \texttt{ELF64\_PHEADER\_LAST}.
Пример использования макросов для загрузки второго загрузчика ОС, показан в листинге~\ref{lst:elf_macro_example}.

\begin{lstlisting}[language=C,
caption={Пример использования макросов для работы с ELF-файлами},
label={lst:elf_macro_example}]
	for (struct elf32_program_header *ph = ELF32_PHEADER_FIRST(elf_header);
	     ph < ELF32_PHEADER_LAST(elf_header); ph++) {
		....
	}
\end{lstlisting}

\section{Сборка ядра}
Для запуска ядра в эмуляторе необходимо подготовить образ диска, который включает в себя
бинарный файл первого загрузчика, бинарный файл второго загрузчика и бинарный файлы ядра.

После того как эти файлы собраны, образ диска генерируется с помощью утилиты \textbf{dd},
в листинге~\ref{lst:make-iso} показано правило, используемое для его генериации.

\begin{lstlisting}[language=Make, caption={Правило для генерации образа диска}, label={lst:make-iso}]
${IMAGE}: $(KERNEL) $(BOOTLOADER) $(LOADER)
	dd if=/dev/zero of=${IMAGE} bs=1M count=10
	dd if=$(BOOTLOADER) of=${IMAGE} conv=notrunc
	dd if=$(LOADER) of=${IMAGE} seek=1 conv=notrunc
	dd if=$(KERNEL) of=${IMAGE} bs=1M seek=1 conv=notrunc
\end{lstlisting}

\subsection{Сборка первого загрузчика}
Основные фрагменты мейкфайла для сборки загрузчика представлены в листинге~\ref{lst:make-bootloader}.

\begin{lstlisting}[language=Make, caption={Фрагмент мейкфайла для сборки первого загрузчика}, label={lst:make-bootloader}]
BOOTLOADER_LDFLAGS = ${LDFLAGS} ${X86_LDFLAGS} \
		     -Wl,--entry=boot_entry -Wl,-Ttext -Wl,0x7c00

$(BOOTLOADER): ${bootloader_objects}
	$(CC) -o $@ ${BOOTLOADER_LDFLAGS} $^
	$(OBJCOPY) --only-keep-debug $@ $@.debug
	$(OBJCOPY) -S -O binary -j .text $@ $@.strip
	$(PERL) ${BOOTLOADER_DIR}/sign.pl $@.strip
	mv $@.strip $@
\end{lstlisting}

Следует обратить внимание на опции линковщика: \texttt{-Wl,--entry=boot\_entry -Wl,-Ttext -Wl,0x7c00},
которые сообщают линковщику какой символ является точкой входа в загрузчик (\texttt{boot\_entry}) и по
какому адресу он должен быть расположен (\texttt{0x7c00}).

После сборки загрузчик имеет размер больше 510 байт, т.к. содержит отладочные символы, которые
необходимы для выполнения отладки на уровне исходных кодов, поэтому их необходимо сохранить
в отдельный файл с помощью утилиты \texttt{objcopy}, после чего их можно удалить из основного файла
с помощью той же утилиты.

Заключительным этапом подготовки загрузчика является создание сигнатуры, чтобы BIOS определил, что
первый сектор является загрузочным, для этого необходимо чтобы байты 510 и 511 загрузчика (если считать с 0)
были равны 0x55 и 0xAA соответственно). Для этого используется скрипт \texttt{sign.pl}, фрагмент
которого приведен в листинге~\ref{lst:make-signature-pl}.

\begin{lstlisting}[language=Perl, caption={Фрагмент скрипта для создания сигнатуры}, label={lst:make-signature-pl}]
my $filename = shift or die "Usage: $0 <FILE>\n";
open my $fh, '>>', $filename
	or die "can't open `$filename': $!\n";

print {$fh} "\0" x (510 - $size);
print {$fh} "\x55\xAA";

close $fh
\end{lstlisting}

\subsection{Сборка второго загрузчика}
Ядро ОС и второй загрузчик используют общий код: управление памятью, вывод на экран,
работа с диском. Однако, ядро является 64-битным, а заргузчик -- 32-битным, поэтому
необходимо иметь возможность собирать библиотеки под разные архитектуры: x86 и x86-64.

Для реализации возможности сборки под разные архитектуры, используется иерархия мейкфайлов,
это позволяет включать мейкфайл для сборки библиотеки в мейкфайл верхнего уровня несколько
раз, используя различные опции. В листинге~\ref{lst:make-multilib} приведен фрагмент мейкфайла, используемый
для сборки двух версий библиотеки. Результаты сборки сохраняются в директории \texttt{i386} и \texttt{x86\_64}.

\begin{lstlisting}[language=Make, caption={Фрагмент мейкфайла для сборки библиотек}, label={lst:make-multilib}]
KERNEL_LIB_ARCH := ${ARCH32}
KERNEL_LIB_CFLAGS := ${CFLAGS} ${X86_CFLAGS} -O0 \
		    -DKERNEL_BASE=${KERNEL_BASE} -DVADDR_BASE=0
include ${KERNEL_LIB_DIR}/Makefile.part

KERNEL_LIB_ARCH := ${ARCH64}
KERNEL_LIB_CFLAGS := ${KERNEL_CFLAGS}
include ${KERNEL_LIB_DIR}/Makefile.part
\end{lstlisting}

При линковке загрузчика используется скрипт, которые задает имя и адрес точки входа, а так
же экспортирует символ \texttt{end}, которые используется загрузчик для определения свободной
области памяти. В листинге~\ref{lst:make-loader-linker} приведены основные фрагменты скрипта линковки.

\begin{lstlisting}[language=Make, caption={Фрагменты скрипта линковки}, label={lst:make-loader-linker}]
ENTRY(_start)
SECTIONS
{
	/* Start from 1MB */
	. = 1M;

	.text BLOCK(4K) : ALIGN(4K)
	{
		*(.text)
		*(.long_mode_asm)
	}
	....
	PROVIDE(end = .);
}
\end{lstlisting}


\subsection{Сборка ядра}
Так как ядро не поддерживает файловую систему, прикладные программы были добавлены в образ ядра
с помощью линковщика, используя опцию \texttt{-Wl,--format=binary -Wl,\${USER\_PROGS}}. Данная
опция говорит линковщику о необходимости создать специальные символы: \texttt{\_binary\_<путь>\_start},
\texttt{\_binary\_<путь>\_end} и \texttt{\_binary\_<путь>\_size} для каждого файла из \texttt{\${USER\_PROGS}}.
Ядро использует данные символы для загрузки и запуска прикладных программ.

В листинге~\ref{lst:make-kernel-user} приведены основные фрагменты мейкфайла для добавления
прикладных программ в исполняемый файл ядра ОС.

\begin{lstlisting}[language=Make, caption={Фрагмент мейкфайла для добавления бинарных файлов}, label={lst:make-kernel-user}]
USER_DIR = user
USER_PROGS = ${USER_DIR}/hello.bin \
	     ${USER_DIR}/fork.bin \
	     ${USER_DIR}/spin.bin \
	     ${USER_DIR}/exit.bin \
	     ${USER_DIR}/read_kernel.bin \
	     ${USER_DIR}/read_unmap.bin \
	     ${USER_DIR}/write_kernel.bin \
	     ${USER_DIR}/write_unmap.bin \
	     ${USER_DIR}/yield.bin

KERNEL_LDFALGS = ${LDFLAGS} -O0 -lgcc \
		 -Wl,--format=binary -Wl,${USER_PROGS} -Wl,--format=default
\end{lstlisting}

\section{Реализация загрузчика ОС}
\subsection{Открытие вентиля линиции A20}
Для открытия вентиля линии A20 используется контроллер клавиатуры 8042, взаимодействие с которым
осуществляется черты порты 0x60 и 0x64. Перед отправкой команды или данных контроллеру необходимо дождаться, пока
он не будет готов их принять. Код выполняющий ожидание готовности контроллера приведен в листинге~\ref{lst:wait_8042}.

\begin{lstlisting}[language={[x86masm]Assembler},
caption={Ожидание готовности контроллера 8042},label={lst:wait_8042}]
wait_8042:
	inb $0x64, %al
	testb $0x2, %al
	jnz wait_8042
	ret
\end{lstlisting}

Код, выполняющий открытия вентиля A20 показан в листинге~\ref{lst:enable_a20}.

\begin{lstlisting}[language={[x86masm]Assembler},
caption={Открытие вентиля линии A20},label={lst:enable_a20}]
	// tell 8042 that we want to write
	call wait_8042
	movb $0xd1, %al
	outb %al, $0x64

	// enable a20
	call wait_8042
	movb $0xdf, %al
	outb %al, $0x60
\end{lstlisting}

Первая группа команд сообщает контроллеру, что за ней последует операция записи.
Вторая группа команд отправляет в порт данных значение 0xdf, что приводит к открытию
вентиля линии A20.

\subsection{Определение доступных областей физической памяти}
Для определения доступных областей физической памяти используется функция 0xe820 прерывания 0x15 BIOS.
Данную функцию необходимо вызывать в цикле, до тех пор пока не будут возвращены все известные области
памяти. Код, использующий данную функцию приведен в листинге~\ref{lst:detect_memory}.

\begin{lstlisting}[language={[x86masm]Assembler},
caption={Определение доступных областей физической памяти},label={lst:detect_memory}]
detect_high_memory:
	movl $0xe820, %eax
	movl $24, %ecx
	addw $24, %di
	int $0x15

	// carry flag is clear in case all is ok
	jc memory_detected

	// magic value must be inside `%eax' in case success call
	cmpl $0x534d4150, %eax // `SMAP'
	jne memory_detected

	// increase records count
	incl %esi

	// `%ebx' will be set to `0' at the end of list
	test %ebx, %ebx
	jnz detect_high_memory
\end{lstlisting}


\subsection{Переход в защищенный режим}
Как было сказано в предыдущих разделах, для перехода в защищенный режим системному ПО
необходимо объявить GDT, содержащую минимум 3 дескриптора сегмента: нулевой, дескриптор сегмента
кода и дескриптор сегмента данных. Для объявления дескрипторов сегментов в заголовочкон файле
\texttt{kernel/misc/gdt.h} объявлен макрос \texttt{SEG} и константы, используемые для улучшения
читаемости кода. В листинге~\ref{lst:declare_gdt} приведен пример объявления GDT и ее дескриптора.

\begin{lstlisting}[language={[x86masm]Assembler},
caption={Объявление GDT и ее дескриптора},label={lst:declare_gdt}]
// force 4 byte alignment
.p2align 2

gdt:
	SEG(0x0, 0x0, 0x0) // null seg
	SEG(UST_X|USF_D|USF_P|USF_S|USF_G|UST_R, 0x0, 0xfffff) // code seg
	SEG(USF_D|USF_P|USF_S|USF_G|UST_W, 0x0, 0xfffff) // data seg

gdtdesc:
	.word (. - gdt - 1)
	.long gdt
\end{lstlisting}

В листинге~\ref{lst:enter_protected_mode} приведен код, выполняющий переход в защищенный режим,
с последующей инициализацией сегментых дескрипторов.
\begin{lstlisting}[language={[x86masm]Assembler},
caption={Переход в защищенный режим},label={lst:enter_protected_mode}]
	// enable protected mode (set first bit in cr0)
	movl %cr0, %eax
	orl  $0x1, %eax
	movl %eax, %cr0

	lgdt gdtdesc
	ljmp $0x8, $complete_flush

.code32
complete_flush:
	// set up protected mode segment registers
	movw $0x10, %ax
	movw %ax, %ds
	movw %ax, %es
	movw %ax, %ss
\end{lstlisting}

\subsection{Переход в длинный режим}
Как было сказано в предыдущих разделах, для перехода в длинный режим, необходимо
активировать PAE, загрузить в регистр CR3 физический адрес PML4, установить бит
EFER.LME в 1 и включить страничное преобразование (установить бит CR.PG в 1). В
листинге~\ref{lst:enter_long_mode} приведен фрагмент кода, выполняющего все эти действия.

\begin{lstlisting}[language={[x86masm]Assembler},
caption={Переход в длинный режим},label={lst:enter_long_mode}]
	// Enable PAE
	movl %cr4, %eax
	btsl $5, %eax
	movl %eax, %cr4

	// Setup CR3
	movl pml4, %cr3

	// Enable long mode (set EFER.LME=1)
	movl $0xc0000080, %ecx // EFER MSR number
	rdmsr                  // Read EFER
	btsl $8, %eax          // Set LME=1
	wrmsr                  // Write EFER

	// Enable paging to activate long mode
	movl %cr0, %eax
	btsl $31, %eax
	movl %eax, %cr0
\end{lstlisting}

\subsection{Передача информации ядру}
Для передачи информации, подготовленной загрузчиком ядру испольются структуры:
\texttt{kernel\_ptr} и \texttt{kernel\_config}. Определение структур приведено
в листинге~\ref{lst:kernel_config_structs}.

\begin{lstlisting}[language=C,
caption={Структуры для передачи информации ядру},
label={lst:kernel_config_structs}]
union kernel_ptr {
	uint64_t uintptr;
	void *ptr;
};

struct kernel_config {
	union kernel_ptr pml4;

	union kernel_ptr pages;
	uint64_t pages_cnt;

	union kernel_ptr gdt;
};
\end{lstlisting}

Следует заметить, что в защищенном режиме размер указателя составляет 32 бита, а
в 64-битном длинном режиме -- 64-бит. Обединение \texttt{kernel\_ptr} используется
чтобы указатели в защищенном 32-битном режиме занимали 64 бита. Это позволяет
упростить ядру интерпретацию переданных загрузчиком данных.

\section{Реализация ядра ОС}
\subsection{Прерывания и исключения}
Для возможности обработки исключений и прерываний, необходимо загрузить в IDT, содержащую
дескрипторы шлюзов обработчиков прерываний и исключений. В листинге~\ref{lst:load_idt}
приведен фрагмент кода, выполняющий загрузку IDT.

\begin{lstlisting}[language=C,
caption={Загрузка IDT},
label={lst:load_idt}]
struct idtr {
	uint16_t limit;
	void *base;
} __attribute__((packed)) idtr = {
	sizeof(idt)-1, idt
};

// Load idt
asm volatile("lidt %0" :: "m" (idtr));
\end{lstlisting}

Для объявления точек входа в обработчики прерываний используются 2 макроса, их
код приведен в листинге~\ref{lst:declare_interrupt_handler_entry}.

\begin{lstlisting}[language=C,
caption={Макросы для объявления точек входа в обработчики прерываний},
label={lst:declare_interrupt_handler_entry}]
#define interrupt_handler_no_error_code(name, num) \
	.globl name;                               \
	.type name, @function;                     \
	.align 4;                                  \
	name:                                      \
	pushq $0; /*instead of error code*/        \
	pushq $(num);                              \
	jmp interrupt_handler_common

#define interrupt_handler_with_error_code(name, num) \
	.globl name;                                 \
	.type name, @function;                       \
	.align 4;                                    \
	name:                                        \
	pushq $(num);                                \
	jmp interrupt_handler_common
\end{lstlisting}

Первый макрос используется для объявления точек входа в обработчики прерываний для
которых процессор не добавляет в стек код ошибки. Второй -- для прерываний с кодом
ошибки. Это позволяет использовать одну структу для представления контекста прерывания.

После сохранения номера прерывания все обработчики прерываний используют общий код,
который сохраняет в стеке контекст текущего процесса, загружает в сегментные регистры
дескрипторы сегментов данных ядра и вызывает обработчик прерываний, написанный на
языке Си. Код, выполняющий эти действия представлен в листинге~\ref{lst:int_handler_common}.

\begin{lstlisting}[language=C,
caption={Общий код обработчиков прерываний},
label={lst:int_handler_common}]
interrupt_handler_common:
	pushq $0x0 // reserve space for segment registers
	movw %ds, 0(%rsp)
	movw %es, 2(%rsp)
	movw %fs, 4(%rsp)
	movw %gs, 6(%rsp)

	// Save RONs: r15-r8, rbp-rax
	....

	pushq %rax
	movw $GD_KD, %ax
	movw %ax, %ds
	movw %ax, %es
	popq %rax

	// Doesn't return
	call interrupt_handler
\end{lstlisting}

Для возможности обработки прерываний необходимо активировать APIC,
в листиге~\ref{lst:apic_enable} приведен код, который для этого предназначен.
\begin{lstlisting}[language=C,
caption={Активация локального APIC},
label={lst:apic_enable}]
// 0x1B - msr of local apic
// bit 11 - global enable/disable APIC flag
asm volatile(
	"movl $0x1b, %ecx\n\t"
	"rdmsr\n\t"
	"btsl $11, %eax\n\t"
	"wrmsr"
);
\end{lstlisting}

После инициализации локального APIC необходимо настроить перенаправление прерываний (через IO APIC).
Для этого были разработан макрос \texttt{IOAPIC\_WRITE}, который записывает в регистр, переданный
в качестве превого параметра, значение, переданное в качестве второго параметра.

В листинге~\ref{lst:ioapic_redirect} приведен фрагмент кода перенаправления прерываний от клавиатуры
с использованием этого макроса.

\begin{lstlisting}[language=C,
caption={Настройка IOAPIC},
label={lst:ioapic_redirect}]
// keyboard
IOAPIC_WRITE(IOREDTBL_BASE+2, INTERRUPT_VECTOR_KEYBOARD);
IOAPIC_WRITE(IOREDTBL_BASE+3, local_apic_id);
\end{lstlisting}

Для возврата из обработчика прерываний используется функция \texttt{task\_run},
которая переводит процесс в состояние \texttt{TASK\_STATE\_RUN} и восстанавливает
контекст процесса, сохраненный обработчиком прерывания. Код данной функции
приведен в листинге~\ref{lst:task_run}.

Следует обратить внимание, что перед запуском процесса в регистре флагов процесса
устанавливается флаг RFLAGS.IF, это выполняется для того, чтобы активировать прерывания
при возврате в пространство пользователя, т.к. при передаче управления обработчику
прерывания процессор сбрасывает данный флаг.

\begin{lstlisting}[language=C,
caption={Запуск процесса},
label={lst:task_run}]
void task_run(struct task *task) {
	// Always enable interrupts
	task->context.rflags |= RFLAGS_IF;
	task->state = TASK_STATE_RUN;

	asm volatile(
		"movq %0, %%rsp\n\t"

		// restore gprs: rax-rbp, r8-r15
		...

		// restore segment registers
		"movw 0(%%rsp), %%ds\n\t"
		"movw 2(%%rsp), %%es\n\t"
		"addq $0x8, %%rsp\n\t"

		// skip interrupt_number and error_code
		"addq $0x10, %%rsp\n\t"

		"iretq" : : "g"(task) : "memory"
	);
}
\end{lstlisting}

\subsection{Системные вызовы}
Прикладные программы получают доступ к сервисам ядра используя механизм системных
вызовов. Для выполенения системного вызова используется код приведенный в
листинге~\ref{lst:syscall}.

Как видно из листнига, для передачи номера
системного вызова используется регистр RAX, этот же регистр используется
для последующего возврата результата системного вызова.

Параметры в системный вызов передаются через регистры RBX, RCX, RDX, RDI, RSI.
Системные вызовы в разрабатываемой ОС либо не принимают аргументов, либо принимают один
аргумент, поэтому 5 регистров достаточно. При реализации системных вызовов,
требующих больше 5 аргументов, прикладному ПО необходимо создать на стеке
структуру, содержащую требуемые аргументы и передавать ее адрес в одном из регистров.
Ядро сможет обращаться к этой структуре, поскольку обработка системного вызова
выполняется в контексте процесса, совершившего системный вызов.

\begin{lstlisting}[language=C,
caption={Выполнение системных вызовов},
label={lst:syscall}]
asm volatile("int %1\n"
	: "=a" (ret)
	: "i" (INTERRUPT_VECTOR_SYSCALL),
	  "a" (syscall),
	  "b" (arg1),
	  "c" (arg2),
	  "d" (arg3),
	  "D" (arg4),
	  "S" (arg5)
	: "cc", "memory");
\end{lstlisting}

Для выполнение системных вызовов прикладные процессы используют следующие функции:
\texttt{sys\_puts}, \texttt{sys\_yield}, \texttt{sys\_exit}, \texttt{sys\_fork}. В
листинге~\ref{lst:sys_puts} приведен код функции \texttt{sys\_puts}, остальные функции
реализованы аналогично.

\begin{lstlisting}[language=C, caption={Реализация функции \texttt{sys\_puts}}, label={lst:sys_puts}]
void sys_puts(const char *string)
{
	return (void)syscall(SYSCALL_PUTS, (uintptr_t)string, 0, 0, 0, 0);
}
\end{lstlisting}


\section{Клонирование процессов}
Одной из особенностей разработанной ОС является возможность эффективного клонирования
процессов, используя механиз копирования при записи. Клонирование процессов
осуществляется с использование системного вызова \texttt{fork()}. Данный системный
вызов создает новый процесс, выделяет ему PML4 и отображает в нее все страницы
родительского процесса, причем для страниц, которые были доступны для записи сбрасывается
бит <<W>> и устанавливается бит <<COW>> (11), означающий что страница должна быть
скопирована при попытке записи в нее. В листинге~\ref{lst:remap_cow} приведен фрагмент кода,
используемых для отображения страниц.

\begin{lstlisting}[language=C,
caption={Отображение страниц для реализации копирования при записи},
label={lst:remap_cow}]
if ((perm & PTE_W) != 0 || (perm & PTE_COW) != 0) {
	perm = (perm | PTE_COW) & ~PTE_W;
	if (page_insert(src->pml4, p, va_addr, perm) != 0)
		return -1;
	if (page_insert(dest->pml4, p, va_addr, perm) != 0)
		return -1;
} else {
	if (page_insert(dest->pml4, p, va_addr, perm) != 0)
		return -1;
}
\end{lstlisting}

Когда один из процессов попытается записать данные в одну из страниц c битом <<COW>> произойдет страничное
исключение. Обработчик страничного исключения скопирует страницу и установит для ее копии бит <<W>>. Для
того чтобы скопировать страницу, обработчик прерывания выполняет следующие действия:
\begin{enumerate}[1.]
\item выделяет новую физическую страницу из списка свободных страниц, используя функцию \texttt{page\_alloc};
\item отображает новую физическую страницу по виртуальному адресу \texttt{KERNEL\_TEMP}), используя функцию \texttt{page\_insert};
\item копирует в \texttt{KERNEL\_TEMP} содержимое оригинальной страницы, используя функцию \texttt{memcpy};
\item отображает новую физическую страницу по адресу, по которому произошло исключение, используя функцию \texttt{page\_insert},
	при этом удаляется отображение оригинальной физической страницы по этому адресу;
\item удаляет отображение новой страницы в \texttt{KERNEL\_TEMP}, т.к. оно больше не нужно.
\end{enumerate}

Таким образом реализуется отложенное копирование страниц. В листинге~\ref{lst:page_fault_handler}
приведен фрагмент обработчика страничного прерывания, реализующий копирование при записи.

\begin{lstlisting}[language=C,
caption={Фрагмент обработчика страничного исключения},
label={lst:page_fault_handler}]
if ((*pte & PTE_COW) != 0) {
	if ((new = page_alloc()) == NULL) {
		terminal_printf("page_fault_handler: can't allocate page\n");
		goto fail;
	}
	if (page_insert(task->pml4, new, KERNEL_TEMP, (*pte & PTE_FLAGS_MASK)| PTE_W) != 0)
		goto fail;
	memcpy((void *)KERNEL_TEMP, (void *)ROUND_DOWN(va, PAGE_SIZE), PAGE_SIZE);
	if (page_insert(task->pml4, new, ROUND_DOWN(va, PAGE_SIZE), (*pte & PTE_FLAGS_MASK) | PTE_W) != 0)
		goto fail;
	page_remove(task->pml4, KERNEL_TEMP);

	task_run(task);
}
\end{lstlisting}


\subsection{Потоки ядра}
Еще одной особенностью данной ОС являются потоки ядра. Они реализованы на основе процессов, но
выполняются в адресном пространстве ядра. Для создания потока ядра используется функция \texttt{thread\_create},
которая позволяет задавать имя потока, главную функцию потока, аргумент, который необходимо передать в фунцию
и его размер.

Начальной функцией для всех потоков является функция \texttt{thread\_foo}, которая принимает в качестве
параметров указатель на дескриптор потока, указатель на функцию потока и указатель на аргумент, который
нужно передать этой функции. Код функции \texttt{thread\_foo} приведен в листинге~\ref{lst:thread_foo}.

\begin{lstlisting}[language=C,
caption={Точка входа потоков ядра},
label={lst:thread_foo}]
static void thread_foo(struct task *thread, thread_func_t foo, void *arg) {
	assert(thread != NULL && foo != NULL);
	foo(arg);
	task_destroy(thread);

	// call schedule
	asm volatile ("int3");
}
\end{lstlisting}

Согласно~\cite{x86_64_abi}, для передачи аргументов в функции используются регистры RDI, RSI, RDX, R10, R8, R9.
В листинге~\ref{lst:create_args} приведен фрагмент кода, используемый для передачи параметров в фунцию \texttt{thread\_foo}.

\begin{lstlisting}[language=C,
caption={Передача параметров в функцию \texttt{thread\_foo}},
label={lst:create_args}]
task->context.gprs.rdi = (uintptr_t)task;
task->context.gprs.rsi = (uintptr_t)foo;
task->context.gprs.rdx = (uintptr_t)data;
\end{lstlisting}

Стек потоков ядра можно было реализовать несколькими способами:
\begin{enumerate}[1.]
\item Заранее выделить пул страниц, которые могут быть использованы в качестве стека потоков ядра;
\item Выделять страницы из общего пула и отображать их в определенный виртуальный адрес.
\end{enumerate}

Преимущество первого подхода заключается в том, что он минимизирует накладные расходы, связанные
с переключеним между потоками, т.к. в этом случае все потоки могут использовать одну общую таблицу
PML4 и нет необходимости изменять значение регистра CR3 при переключении контекста. Основным
недостатком является ограниченное число потоков, т.к. область памяти, в которой будут расположены,
которые можно использовать для стека должна быть выделена заранее.

Преимуществом второго подхода является отсутсвтие необходимости резервировать определенные
области ядра только для создания стеков. Кроме того, данный способ позволяет работать с потоками
так же, как и с процессами, т.к. использовать одни и те же функции для передачи управления и
уничтожения, что упрощает код. Основным недостатком данного способа являются накладные расходы,
связанные с обновлением значения регистра CR3. Однако, в учебной ОС данный недостаток не является
существенным, поэтому был выбран второй способ.

Следует заметить, что стек потока ядра находится вне виртуального адресного пространства ядра,
поэтому из потока ядра нельзя напрямую вызвать планировщик для переключения задач. Для решения данной
проблемы ядро использует прерывание int3, обработчик которого вызывает переключение задач.

\section{Тестирование}
Для тестирования ОС написаны прикладные программы, которые проверяют различные функции ядра ОС:
вывод на экран, чтение и запись данных в различные области памяти, создание и уничножение
процессов, вытесняющая многозадачность.

\paragraph{Вывод на экран.} Для тестирования вывода на экран используется функция \texttt{sys\_puts},
принимающая указатель на строку, завершающуюся нулевым символом. В листинге~\ref{lst:user_hello} приведен
код программы, печатающей на экран сообщение \texttt{Hello world}.

\begin{lstlisting}[language=C, caption={Использование функции \texttt{sys\_puts}}, label={lst:user_hello}]
#include "user/syscall.h"

int main(void)
{
	sys_puts("Hello world\n");
	return 0;
}
\end{lstlisting}

\paragraph{Чтение и запись в различные области.} Для проверки ограничений чтения и записи в
области ядра и неотображенные области, написы прикладные программы: \texttt{read\_kernel},
\texttt{read\_unmap}, \texttt{write\_kernel} и \texttt{write\_unmap}. Которые, соответственно,
выполняют чтение из области ядра, чтение из неотображенной области, запись в область ядра,
запись в неотображенную область. В результате должно происходить страничное исключение и
уничтожение прикладного процесса. В листинге~\ref{lst:read_kernel} приведен код программы,
выполняющей чтение из области ядра.

\begin{lstlisting}[language=C, caption={Чтение из области ядра}, label={lst:read_kernel}]
#include "user/syscall.h"
#include "stdlib/string.h"

#define KERNEL_BASE 0xFFFFFFF800000000

int main(void)
{
	char *kernel = (char *)KERNEL_BASE;
	char buffer[10];

	sys_puts("before read\n");
	memcpy(buffer, kernel, sizeof(buffer));
	sys_puts("after read");

	return 0;
}
\end{lstlisting}

\paragraph{Добровольное освобождение процессора.} Для проверки функции \texttt{sys\_yield}
написана программа \texttt{yield}, которая создает дочерний процесс и выводит сообщения
на экран в дочернем и в родительском процессах, выполняя системный вызов \texttt{YIELD}
после вывода сообщения. В листинге~\ref{lst:user_yield} приведен фрагменты программы
\texttt{yield}. В результате выполнения программы сообщение дочернего и родительского процессов
должны чередоваться.

\begin{lstlisting}[language=C, caption={Использование функции \texttt{sys\_yield}}, label={lst:user_yield}]
for (int i = 0; i < 10; i++) {
	sys_puts("before yield\n");
	sys_yield();
	sys_puts("after yield\n");
}
\end{lstlisting}


\paragraph{Вытеснение процессов.} Для проверки работы вытесняющей многозадачности написана
программа \texttt{spin}, которая создает дочерний процесс и выводит на экран сообщения в
дочернем и в родительском процессах. Фрагмент кода программы приведен в листинге~\ref{lst:user_spin}.
В результате выполнения программы сообщения дочернего и родительского процессов должны чередоваться.

\begin{lstlisting}[language=C, caption={Фрагмент кода программы \texttt{spin}}, label={lst:user_spin}]
for (int i = 0; i < 10; i++) {
	if (pid == 0) {
		sys_puts("I'am still alive (child)\n");
	} else {
		sys_puts("I'am still alive (parent)\n");
	}

	pause();
}
\end{lstlisting}

\section{Отладка}
Для отладки ядра, использовался эмулятор QEMU совместно с отладчиком GDB~\cite{gdb}.
GDB позволяет выполнять отладку как на уровне исходных кодов, так и на уровне машинных команд,
выводить и изменять значения переменных, регистров и произвольных участков памяти.

Для начала отладки необходимо запустить QEMU в отладочном режиме, для этого необходимо
к опциям командной строки добавить: \textbf{-s -S}, либо выполнить команду \textbf{make qemu-gdb}.

Далее необходимо подключиться GDB к QEMU. Для этого нужно запустить gdb и ввести команду:
\texttt{target remote localhost:1234}. После этого необходимо задать архитектуру процессора,
используя команду \texttt{set architecture <требуемая архитектура>} и загрузить отладочные
символы: \texttt{add-symbol-file <путь к файлу с отладочными символами> <адрес по которому их нужно загрузить>}.

Для упрощения работы с GDB был создания специальный файл: <<.gdbint>>, в который были добавлены
сокращения, для описанных выше комманд. Таким образом, чтобы подключиться к QEMU, после запуска
GDB достаточно ввести команду \texttt{qemu}. В листинге~\ref{lst:gdbinit-qemu} приведено описание
команды \texttt{qemu}.

\begin{lstlisting}[language=C, caption={Реализация команды <<qemu>>}, label={lst:gdbinit-qemu}]
define qemu
	if $argc != 1
		target remote localhost:1234
	else
		target remote $arg0
	end
end
\end{lstlisting}

\subsection{Отладка загрузчиков}
После подключения GDB к QEMU для отладки загрузчиков необходимо выполнить следующие действия:
\begin{enumerate}[1.]
	\item \texttt{set architecture i8086} -- задать архитектуру процессора x86;
	\item \texttt{add-symbol-file kernel/boot/bootloader.debug 0x7c00} -- загрузить отладочные
		символы первого отладчика;
	\item \texttt{add-symbol-file kernel/loader/loader 0x100000} -- загрузить отладочные символы
		второго отладчика;
	\item \texttt{layout src} -- установить режим отладки на уровне исходных кодов;
	\item устновить точку останова, используя команду \texttt{break};
	\item \texttt{continue} -- продолжить выполнение до точки останова.
\end{enumerate}

В результате GDB остановит выполнение на
указанной точке останове, после чего можно использовать все его возможности для отладки и изучения работы ядра.

Для упрощения этих действий, в файл <<.gdbinit>> была добавлена функция \texttt{debug-loader}, код
которой приведен в листинге~\ref{lst:gdbinit-debug-loader}.

\begin{lstlisting}[language=C, caption={Реализация команды <<debug-loader>>}, label={lst:gdbinit-debug-loader}]
define debug-loader
	set architecture i8086

	add-symbol-file kernel/boot/bootloader.debug 0x7c00
	add-symbol-file kernel/loader/loader 0x100000

	layout src
end
\end{lstlisting}

\subsection{Отладка ядра}
GDB не позволяет обращаться к памяти за пределами первых 4 гигабайт, пока процессор не перешел в длинный
режим. По этой причине нельзя сразу после подключения к QEMU загрузить отладочние символы ядра и утсановить
на одном из них точку останова. Для решения этой проблемы можно поступить следующим образом:
\begin{enumerate}[1.]
	\item В код ядра (например в функцию \texttt{kernel\_main}) добавить конструкцию вида:
		\texttt{int a = 0; while (a == 0);}. Данная конструкция представляет собой
		бесконечный цикл, из которого можно выйти, используя отладчик.
	\item Запустить QEMU в режиме отладки.
	\item Подключиться GDB к QEMU.
	\item Ввести команды \texttt{qemu} и \texttt{continue}.
	\item Нажать \texttt{Ctrl + C} для возврата командной строки GDB и ввести команду \texttt{debug-kernel}.
		После этого на экране появится код ядра.
	\item Для выхода из цикла можно воспользоваться следующей командой: \texttt{set variable \$a = 1}.
		После этого можно продолжить отладку ядра.
\end{enumerate}

Код команды \texttt{debug-kernel} приведен в листинге~\ref{lst:gdbinit-debug-kernel}.
\begin{lstlisting}[language=C, caption={Реализация команды <<debug-kernel>>}, label={lst:gdbinit-debug-kernel}]
define debug-kernel
	set architecture i386:x86-64

	add-symbol-file kernel/kernel 0xFFFFFFF800200000

	layout src
end
\end{lstlisting}

