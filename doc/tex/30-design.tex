\chapter{Конструкторский раздел}
\label{cha:design}

\section{Режимы работы процессора}
Унаследованная (legacy) архитектура x86 предусматривает четыре режима работы процессора:
\begin{enumerate}[1.]
	\item Реальный режим (англ. Real Mode)
	\item Защищенный режим (англ. Protected Mode)
	\item Режим виртуального 8086 (англ. Virtual-8086 Mode)
	\item Режим системного управления (англ. System Management Mode)
\end{enumerate}

Архитектура AMD64 поддерживает все эти режимы и добавляет новый режим,
названный <<длинный>> режим (англ. Long Mode).

\subsection{Длинный режим}
Длинный режим включает в себя 2 подрежима: 64-битный режим и режим совместимости.
64-битный режим поддерживает несколько новых возможностей, включая возможность
использовать 64-битное адресное пространство. Режим совместимости обеспечивает
бинарную совместимость с существующим 16 и 32-битным прикладным ПО при работе
в 64-битном окружении.

Перед активацией и переходом в длинный режим, операционная система должна перейти в
защищенный режим. Процесс перехода в длинный режим описан в главе~\ref{sec:long_mode_activation}.

\subsection{64-битный режим}
64-битный режим -- подрежим длинного режима, предусматривает поддержку 64-разрядного
ПО, добавляя следующие возможности:
\begin{enumerate}[1.]
\item 64-битные виртуальные адреса.
\item Доступ к битам 63:32 регистров общего назначения.
\item Дополнительные 8 регистров общего назначения (R8-R15).
\item 64-битный счетчик команд (RIP).
\item Плоская модель памяти с одним сегментом кода, данных и стека.
\end{enumerate}

Данный режим может быть активирован системным ПО для различных сегментов кода. В данном режиме
механизм сегментного преобразования адреса отключен. Для управления памятью используется
механизм страничного преобразования.

Размер адреса по умолчанию составляет 64 бита, размер операндов -- 32 бита.

\subsection{Режим совместимости}
Режим совместимости -- подрежим длинного режима, позволяет системному ПО обеспечивать
бинарную совместимость с существущим 16 и 32-битным прикладным ПО,
т.е. запускать данное ПО без перекомпиляции в 64-битной ОС в длинном режим.

В режиме совместимости, приложениям доступны только первые 4 гигабайта виртуального адресного пространства.

Данный режим, как и 64-битный может быть активирован системым ПО для различных сегментов кода.
В данном режиме сегментное преобразование адреса работает так же, как и в унаследованной
архитектуре x86. С точки зрения прикладного ПО, режим совместимости не отличается от унаследованного
защищенного режима. С точки зрения системного ПО -- необходимо использовать механизмы длинного режима для
преобразования адресов и обработки исключений и прерываний.

\subsection{Унаследованные режимы}
Унаследованный режим состоит из трех подрежимов: реальный режим, защищенный режим и режим виртуального 8086.
Страничное преобразование в защищенном режиме не является обязательным. Унаследованный режим сохраняет
бинарную совместимость не только с существующим 16 и 32-битным прикладным ПО, но и с существующим 16 и 32-битным
системым ПО.

\subsubsection*{Реальный режим}
В данном режиме, также называемом режимом реальных адресов, процессору доступен 1~мегабайт физической памяти.
Обработка прерываний и формирование адреса выполняется так же, как и реальном режиме процессора 80286.
Страничное преобразование адреса не поддерживается. Все ПО выполняется на нулевом уровне привилегий.

Процессор начинает работу в реальном режиме.

\subsubsection*{Защищенный режим}
В данном режиме процессору доступно 4~гигабайта физической и виртуальной памяти. Доступны все возможности
сегментного преобразования и аппаратного переключения задач. Если страничное преобразование не используется --
виртуальные адреса совпадают с физическими.

В защищенном режиме ПО выполняется на уровнях привилегий 0-3. Как правило, прикладное ПО выполняется на третьем
уровне привилегий, а системное - на 0, 1 и 2.

\subsubsection*{Режим виртуального 8086}
Данный режим позволяет системному ПО запускать 16-битное ПО реального режима на виртуальном процессоре 8086.
В данном режиме ПО, написанное для процессоров 8086, 8088, 80186 и 80188 можно запустить из защищенного режима,
как задачу третьего уровня привилегий. Процессору будет доступен 1~мегабайт виртуального адресного пространства, а
преобразование адресов будет выполняться как в реальном режиме.

Режим не поддерживается, когда процессор находится в длинном режиме (в этом случае попытки перехода в режим
виртуального 8086 игнорируются).

\subsection{Режим системного управления (SMM)}
Режим системного управления -- режим разработанный для действий, связанных с настройкой аппратного обеспечения,
которые <<незаметны>> для системного ПО. Например, управление питанием. Данный режим, как правило, используется
встроенным программным обеспечением и низкоуровневыми драйверами устройств. Код и данные для SMM сохранены в
области SMM, изолированной от основной памяти.

Вход в SMM осуществляется посредством специального прерывания (SMI). При возникновении SMI процессор переходит
в SMM, переключается на отдельное адресное пространство, в котором находится обработчик SMM и передает ему управление.
В данном режиме процессору доступно 4~гигабайта оперативной памяти; адресация выполняется как и в реальном режиме.

\section{Системные ресурсы}
Операционная система выполняет системные операции (управление памятью, изменение режима работы процессора и др.)
используя системные ресурсы. Эти ресурсы состоят из системных регистров (управляющих и моделезависимых)
и системных структур данных (различные таблицы).

\subsection*{Управляющие регистры}
Регистры, управляющие работой процессора в архитектуре AMD64, включают:
\begin{description}
	\item[CR0] Позволяет изменять ражим работы процессора и управляет некоторыми возможностями процессора.
	\item[CR2] Используется механизмом страничного преобразования. При возникновении страничного исключения, содержит
		виртуальный адрес по которому произошло исключение.
	\item[CR3] Используется механизмом страничного преобразования. Содержит базовый адрес таблицы страниц верхнего
		уровня, управляет кешированием данной таблицы.
	\item[CR4] Содержит дополнительные флаги для различных возможностей процессора.
	\item[CR8] Используется для управления приоритетами внешних прерываний.
	\item[RFLAGS] Хранит состояние процессора и некоторые управляющие флаги. В основном, используется для
		управления аппаратным переключением задач, прерываниями и режимом виртуального 8086.
	\item[EFER] моделезависимый регистр, содержащий состояние процессора и управляющие флаги, для
		возможностей, которые не управляются регистрами CR0 и CR4.
\end{description}

Управляющие регистры CR1, CR5 -- CR7, CR9 -- CR15 зарезервированы.

В унаследованном режиме все управляющие регистры, в т.ч. RFLAGS -- 32-битные. EFER -- 64-битный во всех режимах.
Архитектура AMD64 расширяет все 32-битные управляющие регистры до 64 бит.

В 64-битном режиме старшие 32 бита регистров CR0 и CR4 зарезервированы и должны быть заполнены нулями.
Запись 1 в один из старших бит приведет исключению общей защиты, \#GP.

Старшие 32 бита регистра RFLAGS всегда равны 0. Попытки записать туда не 0 игнорируются процессором.
Подробное описание полей регистров приведено в \cite[стр. 42]{amd_pm_v2}.

\section{Сегментное преобразование адреса}
Унаследованная архитектура x86 поддерживает механизм сегментного преобразования адреса, который
позволяет системному ПО создавать отдельное виртуальное адресное пространство для каждого процесса.
Размер и расположение сегмента в виртуальном адресном пространстве произвольны. Инструкции и данные
могут располагаться как в одном, так и в нескольких сегментах, каждому из которых будут назначены
отдельные атрибуты доступа.

Механизм сегментного преобразования предусматривает 10 сегментных регистров, каждый из которых
определяет один сегмент. 6 из этих регистров (CS, DS, ES, FS, GS и SS) определяют пользовательские сегменты.
Пользовательские сегменты содержат команды, данные и стек. Они доступны как для системного, так и для прикладного ПО.
Оставшиеся 4 регистра (GDTR, LDTR, IDTR и TR) определяют системные сегменты. Системные сегменты содержат структуры
данных инициализируемые и используемые только системным ПО. Сегментные регистры содержат (в теневой части)
базовый адрес, указывающий на начало сегмента, размер сегмента и атрибуты доступа.

Несмотря на то, что сегментное преобразование обеспечивает высокую гибкость при перемещении и защите данных,
обычно изоляция и защита адресных пространств выполнятся эффективнее с использованием программной и аппаратной
поддержки страничного преобразования. По этой причине большинство современных систем не используют сегментное преобразование.
Однако механизм сегментного преобразования нельзя полностью отключить, поэтому понимание его работы
является необходимым при разработке ПО для длинного режима.

В длинном режиме, работа сегментного преобразования зависит от того, в каком из подрежимов находится процессор:
\begin{itemize}
\item В режиме совместимости сегментное преобразование работает так же, как и в унаследованном режиме.
\item В 64-битном режиме, сегментное преобразование отключено, задавая плоское 64-битное адресное пространство.
	Однако некоторые функции сегментых регистров (в частности системных сегментных регистров) продолжают использоваться.
\end{itemize}

\subsection{Сегментное преобразование в реальном режиме}
В данном режиме процессору доступен 1~мегабайт физической памяти. 20-битный физичеческий адрес вычисляется
путем сдвига влево на 4~бита 16-битного селектора сегмента и сложения результата с 16-битным эффективным адресом.

Каждый 64-килобайтный сегмент (CS, DS, ES, FS, GS, SS) выровнен по 16-байтной границе. Базовый адрес сегмента это
минимальный адрес в данном сегменте, он равен (селектор сегмента * 16). Для загрузки селекторов в сегментные регистры
можно использовать инструкции POP и MOV.

GDT, LDT и TSS не используются в данном режиме.

\subsection{Сегментное преобразование в защищенном режиме}
Системное ПО может использовать механизм сегментного преобразования для реализации одной из двух основных
моделей: плоская модель памяти и мульти-сегментная модель памяти. Данные модели памяти поддерживаются в
унаследованном режиме и в режиме совместимости.

\subsubsection*{Мульти-сегментная модель памяти}
В данной модели памяти, каждый сегментный регистр может ссылаться на произвольно расположенные в памяти сегменты разных расмеров.
Сегменты могут иметь размер от одного байта до 4х гигабайт. При использовании страничного преобразования, разные сегменты
могут быть отображены на одну страницу и разные страницы могут быть отображены на один сегмент.

Режим совместимости позволяет использовать мульти-сегментную модель памяти для поддержки унаследованного ПО.
Однако в режиме совместимости, мульти-сегментная модель ограничена первыми 4 гигабайтами виртуального
адресного пространства. Для доступа к памяти выше 4х гигабайт, необходимо перейти в 64-битный режим, который
не поддерживает сегментацию.

\subsubsection*{Плоская модель памяти}
Плоская модель памяти -- это простейшая форма сегментного преобразования. Несмотря на то, что сегментное преобразование
не может быть выключено, плоская модель памяти позволяет системному ПО обойти часть механизмов сегментного преобразования.
В плоской модели памяти все базовые адреса сегментов равны 0, а размеры сегментов равны 4 гигабайтам. Установка
базового адреса сегмента в 0 фактически отключает сегментное преобразование (сегмент:смещение = смещение).

\subsubsection*{Сегментное преобразование в 64-битном режиме}
В 64-битном режиме сегментное преобразование отключено. Аппаратное обеспечение игнорирует значение
базового адреса сегмента и обрабатывает его как 0. Игнорируются размер и большинство атрибутов.
Атрибуты <<DPL>>, <<D>> и <<L>> дескриптора сегмента кода используются соответственно для установки
уровня привилегий, размера операндов по умолчанию и режима работы процессора (64-битный режим или режим совместимости).
Системные сегментные регистры всегда используются в 64-битном режиме.

\subsection{Структуры данных сегментного преобразования}
На рис.~\ref{fig:segmentation-data-structures} показаны структуры данных, используемые
маханизмом сегментного преобразования, среди них:
\begin{itemize}
\item Дескрипторы сегментов. Описывают сегменты (базовый адрес сегмента в виртуальном
	адресном пространстве, его размер, атрибуты доступа и некоторые другие характеристики).
\item Таблицы дескрипторов. Сегментные дескрипторы хранятся в памяти в одной из трех таблиц: GDT, LDT, IDT.
\item Сегмент состояния задачи (TSS). Cпециальный тип системного сегмента, который содержит
	информацию о состоянии процесса (задачи), в т.ч. ссылки на необходимые процессу структуры данных.
\item Сегментные селекторы. Используются для выбора дескрипторов из таблиц дескрипторов.
\end{itemize}

\begin{figure}[ht!]
  \centering
  \includegraphics[width=0.7\textwidth]{inc/dia/segmentation-data-structures}
  \caption{Струкруты данных сегментного преобразования}
  \label{fig:segmentation-data-structures}
\end{figure}

Механизм сегментного преобразования задействует следующие регистры: CS, DS, ES, SS, GS, FS, GDTR, IDTR, LDTR, TR.
Структуры данных (рис.~\ref{fig:segmentation-data-structures}) связаны с регистрами следующим образом:
\begin{itemize}
	\item Сегментные регистры (CS, DS, ES, FS, GS, SS). Используются чтобы ссылаться
		на пользовательские сегменты. При загрузке селектора сегмента в сегментный регистр процессор
		автоматически загружает выбранный дескриптор в теневую часть сегментного регистра.
	\item Регистры таблиц дескрипторов (GDTR, LDTR, IDTR). Задают виртуальный базовый адрес и размер таблиц дескрипторов.
	\item Регистр задачи (TR). Задает положение и размер текущего сегмента состояния задачи (TSS).
\end{itemize}

\subsubsection*{Сегментные селекторы}
Селекторы сегментов указывают на дескрипторы в GDT и LDT.
Формат селектора сегмента показан на рис.~\ref{fig:segment-selector}.

\begin{figure}[ht!]
  \centering
  \includegraphics[width=0.4\textwidth]{inc/dia/segment-selector}
  \caption{Формат селектора сегмента}
  \label{fig:segment-selector}
\end{figure}

Селектор состоит из следующих полей:
\begin{enumerate}[1.]
\item Индекс. Биты 15:3. Указывает на элемент в таблице дескрипторов.
	Дескрипторы имеют размер 8~байт, поэтому индекс умножается на 8
	чтобы получить смещение в таблице дескрипторов. Смещение прибавляется
	к базовому адресу GDT или LDT (в зависимости от значения TI), чтобы получить
	виртуальный адрес дескриптора.

	Некоторые элементы таблицы дескрипторов имееют размер 16~байт, вместо 8.
	Они занимают 2 элемента в таблице. Однако в длинном режиме смещение по
	прежнему вычисляется путем умножения индекса на 8. Системное ПО должно
	назначать селекторы так, чтобы они указывали на начало расширенных элементов.
\item Индикатор таблицы (TI). Бит 2. Указывает в какой таблице хранится дескриптор.
	Если бит сброшен в 0, это означает что селектор ссылается на запись в GDT.
	Иначе - селектор ссылается на запись в LDT.
\item Уровень привилегий (RPL). Биты 1:0. Равен уровню привилегий на котором
	находился процессор (CPL) при создании селектора. Используется для проверки прав доступа.
\end{enumerate}

Нулевые селекторы (индекс 0 и TI=0) используются чтобы сделать сегментные
регистры недействительными (аналог нулевых указателей).
При использовании сегментного регистра (не в 64-битном режиме),
содержащего нулевой селектор произойдет исключение сбоя защиты (\#GP)

Нулевые селекторы можно загружать только в регистры DS, ES, FS, GS и LDTR.

\subsubsection*{Сегментные регистры}
Для улучшения производительности, при загрузке нового значения в сегментный регистр,
процессор загружает дескриптор, соответсвующий селектору, в теневую часть сегментного регистра.
Благодаря этому снижается число обращений к памяти.

На рис.~\ref{fig:segment-register-format-x86} показаны видимая и теневая части сегментного регистра.
ПО не имеет прямого доступа к теневой части сегментных регистров (кроме GS и FS).

\begin{figure}[ht!]
  \centering
  \includegraphics[width=0.4\textwidth]{inc/dia/segment-register-format-x86}
  \caption{Формат сегментного регистра}
  \label{fig:segment-register-format-x86}
\end{figure}

В 64-битном режиме содержимое теневой части регистра CS игнориурется, кроме атрибутов <<D>>, <<L>> и <<DPL>>.
При вычислении адреса, базовый адрес считается равным 0, выход за пределы сегмента не проверяется.
Вместо этого выполняется проверка, что адрес находится в канонической форме. Содержимое теневой части
регистров DS, ES и SS игнорируется полностью.

\subsection{Таблицы дескрипторов}
Механизм сегментного преобразования использует таблицы дескрипторов.
Эти таблицы содержат дескрипторы, которые описывают расположение сегмента в виртуальной памяти,
его размер и атрибуты доступа. Обращение к полям дескриптора происходит при каждом обращении к
памяти (выборка инструкций, чтение/запись данных).

Как было сказано ранее, архитектура x86 поддерживает 3 типа таблиц дескрипторов: GDT, LDT, IDT.

\subsubsection*{GDT}
Для перехода в защищенный режим необходимо создать GDT. GDT содержит дескрипторы
сегментов кода и данных для сегментов, которые являются общими для всех задач.
Кроме пользовательских сегментов, GDT может содержать дескрипторы шлюзов и
другие системные дескрипторы. Системное ПО может расположить GDT в произвольном
области памяти, недоступной непривилегированному ПО.

На рис.~\ref{fig:gdt-ldt-access} показано как происходит доступ к GDT/LDT.

\begin{figure}[ht!]
  \centering
  \includegraphics[width=1.0\textwidth]{inc/dia/gdt-ldt-access}
  \caption{Доступ к GDT/LDT}
  \label{fig:gdt-ldt-access}
\end{figure}

\subsubsection*{Регистр глобальной таблицы дескрипторов}
Регистр GDT (GDTR) содержит базовый адрес и размер GDT. Регистр загружается командой LGDT.
На рис.~\ref{fig:gdtr-idtr-legacy-format} показан формат GDTR в унаследованном режиме и
режиме совместимости.

\begin{figure}[ht!]
  \centering
  \includegraphics[width=.4\textwidth]{inc/dia/gdtr-idtr-legacy-format}
  \caption{Формат GDTR и IDTR в унаследованном режиме}
  \label{fig:gdtr-idtr-legacy-format}
\end{figure}

На рис.~\ref{fig:gdtr-idtr-x64-format} показан формат GDTR в длинном режиме.

\begin{figure}[ht!]
  \centering
  \includegraphics[width=.9\textwidth]{inc/dia/gdtr-idtr-x64-format}
  \caption{Формат GDTR и IDTR в длинном режиме}
  \label{fig:gdtr-idtr-x64-format}
\end{figure}

GDTR состоит из двух полей:
\begin{description}
\item[Размер] 2 байта. Задает размер GDT в байтах. При обращении ПО за пределы GDT произойдет
	исключение общей защиты (\#GP).
\item[Базовый адрес] 8 байт. Содержит виртуальный адрес начала таблицы. GDT может быть расположена
	по любому адресу, однако системному ПО следует использовать адрес, выровненный по 4-байтной границе,
	чтобы избежать снижения производительности из-за доступа к невыровненным данным.

	В архитектуре AMD64 размер базового адреса в GDTR увеличен до 64 бит, что позволяет системному ПО,
	работающему в длинном режиме, расположить GDT в произвольном месте 64-битного адресного пространства.
	В унаследованном режиме процессор игнорирует старшие 4 байта.
\end{description}

\subsubsection*{LDT}
Системное ПО, после перехода в защищенный режим, может использовать локальные таблицы дескрипторов (LDT)
для хранения сегментов используемых одним или группой процессов. Как и GDT, LDT может быть расположена
в произвольной области памяти, недоступной непривилегированному ПО.

Дескрипторы LDT хранятся в GDT. Дескриптор LDT хранит базовый адрес таблицы, размер и атрибуты доступа.
Подробное описание формата LDT и LDTR приведено в \cite{amd_pm_v2}.

\subsubsection*{IDT}
Системное ПО может определить несколько IDT и переключаться между ними, используя регистр IDTR. Так же, как
GDT и LDT, IDT может быть расположена в произвольной области памяти, недоступной непривилегированному ПО.

IDT может содержать дескрипторы следующих типов:
\begin{itemize}
	\item Шлюз прерывания (англ. interrupt gate)
	\item Шлюз ловушки (англ. trap gate)
	\item Шлюз задачи (англ. task gate)
\end{itemize}

В главе ~\ref{sec:exceptions_and_interrupts} описано,
как механизм обработки прерываний использует дескрипторы шлюзов.

Обращение к элементам IDT происходит по номеру вектора прерывания. Смещение в таблице вычисляется путем
умножения номера вектора прерывания на размер элемента таблицы. Размер элемента таблицы зависит от
режима работы процессора следующим образом:
\begin{itemize}
\item В длинном режиме размер элемента IDT составляет 16 байт.
\item В унаследованном режиме размер элемента IDT составляет 8 байт.
\end{itemize}

На рис.~\ref{fig:indexing-an-idt} показано, как происходит индексация в IDT по номерy вектора прерывания.

\begin{figure}[ht!]
  \centering
  \includegraphics[width=1.0\textwidth]{inc/dia/indexing-an-idt}
  \caption{Индексация в IDT}
  \label{fig:indexing-an-idt}
\end{figure}


\subsubsection*{Регистр IDT}
Регистр IDT (IDTR) содержит базовый адрес и размер IDT.
Содержимое регистра загружается командой LIDT. Формат IDTR совпадает с форматом GDTR во
всех режимах работы процессора. На рис.~\ref{fig:gdtr-idtr-legacy-format} показан формат
IDT в унаследованном режиме, а на рис.~\ref{fig:gdtr-idtr-x64-format} -- в длинном режиме.

\subsection{Унаследованные дескрипторы сегментов}
\subsubsection*{Формат дескриптора}
Дескрипторы сегментов определяют и изолируют сегменты друг от друга. Существует 2 основных типа
дескрипторов, каждый из которых используется для описания сегментов (шлюзов) разных типов:
\begin{itemize}
	\item Дескрипторы пользовательских сегментов -- дескрипторы сегментов кода и данных (в т.ч. стека).
	\item Дескрипторы системных сегментов -- дескрипторы LDT, TSS и шлюзов (описывают программные точки входа).
\end{itemize}

На рис.~\ref{fig:legacy-segment-descriptor-format} показан общий формат дескриптора сегмента в унаследованном режиме.
В унаследованном режиме размер сегмента составляет 8 байт (2 двойных слова). На рисунке старшее двойное слово (смещение +4)
изображено вверху, младшее -- внизу.

\begin{figure}[ht!]
  \centering
  \includegraphics[width=1.0\textwidth]{inc/dia/legacy-segment-descriptor}
  \caption{Общий формат дескриптора в унаследованном режиме}
  \label{fig:legacy-segment-descriptor-format}
\end{figure}

Дескриптор имеет следующие поля:
\begin{itemize}
\item Размер сегмента. 20-битный размер сегмента формируется путем объединения
	бит 19:16 старшего двойного слова и 0:15 младшего двойного слова. Задает размер сегмента
	в байтах. Для сегментов данных бит E определяет какую границу сегмента задает данное поле,
	верхнюю или нижнюю.
\item Базовый адрес. 32-битный базовый адрес формируется путем объединения бит 31:24 и 7:0 старшего двойного слова
	с битами 15:0 младшего двойного слова. Содержит адрес начала сегмента в виртуальной памяти.
\item Бит <<S>>. Если равен 0 -- системный сегмент (LDT, TSS, шлюз), иначе -- пользовательский (код, данные).
\item <<Тип>>. Определяет тип сегмента.
\item <<DPL>>. Определят уровень привилегий дескриптора. Может иметь значения от 0 до 3, где 0 -- наибольший
	уровень привилегий, 3 -- наименьший.
\item Бит <<P>>. Определяет присутствует (загружен) ли сегмент в памяти. Если произойдет обращение к сегменту,
	у которого этот бит равен 0 -- произойдет исключение \#NP. Процессор никогда не изменяет значение
	этого бита, это делает только системное ПО.
\item <<AVL>>. Доступно для использования системным ПО. Процессор никогда не модифицирует это поле.
\item Бит <<D/B>>. Размер операнда по умолчанию. Используется для сегментов данных и сегментов кода.
	Если равен 1 -- размер операндов по умолчанию составляет 32 бита, иначе -- 16 бит.
\item Бит <<G>>. Бит гранулярности -- опредяет как обрабатывать размер сегмента. Если равен 0 -- размер сегмента
	задается в байтах. Если равен 1 -- размер сегмента задан в 4-килобайтных блоках.
\end{itemize}

\subsubsection*{Дескрипторы сегментов кода}
На рис.~\ref{fig:legacy-code-segment-descriptor-format} показан формат дескриптора сегмента кода.
Сегменты кода определяют режим работы процессора и уровень привилегий. Сегменты кода доступны только для исполнения,
либо только для чтения и исполнения. Запись в сегмент кода, на который ссылается регистр CS, запрещена.
\begin{figure}[ht!]
  \centering
  \includegraphics[width=1.0\textwidth]{inc/dia/legacy-code-segment-descriptor}
  \caption{Формат дескриптора сегмента кода в унаследованном режиме}
  \label{fig:legacy-code-segment-descriptor-format}
\end{figure}

Для дескрипторов сегментов кода бит <<S>> установлен в 1, означая что это пользовательский сегмент. Бит 11
используется чтобы отличать сегменты кода и сегменты данных (если бит установлен в 1 -- это сегмент кода,
в противном случае -- это сегмент данных). Биты 10:8 определяют характеристики доступа к сегменту кода:
\begin{itemize}
	\item Бит <<C>>. Процессор не меняет текущий уровень привилегий (CPL), если происходит передача управления
		на сегмент с более высоким уровнем привилегий, у которого бит C=1.
	\item Бит <<R>>. Если установлен в 1 -- сегмент доступен для чтения и исполнения. В противном случае --
		только для исполнения (при попытке чтения произойдет исключение \#GP).
	\item Бит <<A>>. Устанавливается в 1 при копировании дескриптора из GDT или LDT в регистр CS. Сбрасывается
		только системным ПО.
\end{itemize}

Бит D. В дескрипторах сегментов кода этот бит определяет размер операндов и адреса по умолчанию.
Если он сброшен в 0 -- размер по умолчанию состаляет 16 бит, в противном случае -- 32 бита.

\subsubsection*{Дескрипторы сегментов данных}
На рис.~\ref{fig:legacy-data-segment-descriptor-format} показан формат дескриптора сегмента данных.
Сегменты данных могут быть доступны либо только для чтения, либо для чтения/записи. Доступ к сегментам
данных осуществляется с использованием регистров DS, ES, FS, GS, SS. Регистр DS содержит селектор
сегмента данных, используемый по умолчанию.

Сегмент стека это одна из форм сегмента данных. Доступ к нему осуществляется через регистр SS. Он должен
быть доступен для чтения и записи.

\begin{figure}[ht!]
  \centering
  \includegraphics[width=1.0\textwidth]{inc/dia/legacy-data-segment-descriptor}
  \caption{Формат дескриптора сегмента данных в унаследованном режиме}
  \label{fig:legacy-data-segment-descriptor-format}
\end{figure}

Для дескрипторов сегментов данных бит <<S>> установлен в 1, означая что это пользовательский сегмент. Бит 11
сброшен в 0, обозначая сегмент данных. Биты 10:8 определяют характеристики доступа к сегменту:
\begin{itemize}
	\item Бит <<E>>. Если этот бит установлен в 1 -- значит это расширяющийся вниз сегмент. В этом случае
		поле <<размер>> определяет нижнюю границу сегмента, а поле <<базовый адрес>> -- верхнюю.
		Это может быть полезно для сегмента стека (т.к. указатель стека растет в сторону уменьшения адресов).
	\item Бит <<W>>. Если бит установлен в 1 -- сегмент доступен для записи. В противном случае попытка записи
		приведет к исключению \#GP.
	\item Бит <<A>>. Устанавливается в 1 при копировании дескриптора из GDT или LDT в сегментный регистр данных или стека.
		Сбрасывается только системным ПО.
\end{itemize}

\subsection{Сегментные дескрипторы длинного режима}
\subsubsection*{Дескрипторы сегмента кода}
В длинном режиме сегменты кода продолжают использоваться. Сегменты кода, их дескрипторы и селекторы
необходимы для установки режима работы процессора и уровня привилегий. Новый атрибут <<L>> определяет в
каком режиме работает процессор -- в 64-битном или режиме совместимости.
На рис.~\ref{fig:long-mode-code-segment-descriptor-format} показан формат дескриптора сегмента кода
в длинном режиме. В режиме совместимости все поля дескриптора интерпретируются также, как и в унаследованном режиме.

\begin{figure}[ht!]
  \centering
  \includegraphics[width=1.0\textwidth]{inc/dia/long-mode-code-segment-descriptor}
  \caption{Формат дескриптора сегмента кода в длинном режиме}
  \label{fig:long-mode-code-segment-descriptor-format}
\end{figure}

\paragraph{Игнорируемые в 64-битном режиме поля.}
В 64-битном режиме сегментное преобразование отключено, сегменты кода занимают все адресное пространство.
В этом режим значение поля <<базовый адрес>> игнорируется.
При вычислении виртуального адреса оно считается равным 0.

Проверка выхода за границу сегмента не выполняется, игнорируются размер и бит <<G>>. Вместо этого выполняется
проверка, что адрес находится в канонической форме.

Также игнорируются биты <<R>> и <<A>> в поле <<тип>>.

\paragraph{Бит <<L>>.} В длинном режиме у дескриптора кода появился дополнительный атрибут -- бит <<L>>. Если он равен 1 --
процессор работает в 64-битном режиме, иначе -- в режиме совместимости. В унаследованном режиме этот бит игнорируется.

Режим совместимости имеет бинарную совместимость с существующим 16 и 32-битным прикладным ПО.
Переход в режим совместимости осуществялется на основании атрибутов сегмента кода, это позволяет
64-битному системному ПО исполнять унаследованное ПО вместе с 64-битным ПО. Для запуска унаследованных
16 и 32-битных приложений системному ПО достаточно сбросить бит <<L>> в дескрипторе сегмента кода в 0.

Если процессор работает в 64-битном режиме (L=1) -- бит <<D>> должен быть равен 0. Это приводит к тому,
что размер операнда по умолчанию составляет 32 бита, а размер адреса по умолчанию составляет 64 бита.
Комбинация L=1 и D=1 зарезервирована для использования в будущем.

\subsubsection*{Дескрипторы сегмента данных}
В длинном режиме сегменты данных продолжают использоваться. На рис.~\ref{fig:long-mode-code-segment-descriptor-format}
показан формат дескриптора сегмента данных в длинном режиме. В режиме совместимости все поля дескриптора
интерпретируются также, как и в унаследованном режиме.

\begin{figure}[ht!]
  \centering
  \includegraphics[width=1.0\textwidth]{inc/dia/long-mode-data-segment-descriptor}
  \caption{Формат дескриптора сегмента данных в длинном режиме}
  \label{fig:long-mode-data-segment-descriptor-format}
\end{figure}

\paragraph{Игнорируемые в 64-битном режиме поля.}
В 64-битном режиме сегментное преобразование отключено. Интерпретация базового адреса зависит от
используемого сегментного регистра:
\begin{itemize}
\item При использовании регистров DS, ES, SS поле <<базовый адрес>> игнорируется и считается равным 0.
\item Регистры GS и FS обрабатываются специальным образом. При использовании этих регистров можно
	использовать ненулевое значение базового адреса для вычисления виртуального адреса.
\end{itemize}

Проверка выхода за границу сегмента не выполняется, игнорируются размер и бит <<G>>. Бит <<D/B>> не
используется в 64-битном режиме.

Биты <<E>>, <<W>> и <<A>> игнорируются.

Поле <<DPL>> также игнорируется, проверки доступа к сегментам данных не выполняются.
Системное ПО может использовать механизм страничной защиты для ограничения доступа в данным.

\subsubsection*{Системные дескрипторы}
\label{subsec:system_desriptor_format}

Как показано на рис.~\ref{fig:long-mode-system-segment-descriptor-format} в 64-битном режиме
системные дескрипторы LDT и TSS увеличены на 64 бита. Увеличение дескрипторов, позволяет
хранить в них 64-битный базовые адреса, поэтому сегменты, которые они описывают могут быть
расположены в произвольном месте в памяти. Расширенные дескрипторы могут быть загружены в
соответствующие регистры (LDTR или TR) только из 64-битного режима.

\begin{figure}[ht!]
  \centering
  \includegraphics[width=1.0\textwidth]{inc/dia/long-mode-system-segment-descriptor}
  \caption{Формат системного дескриптора в длинном режиме}
  \label{fig:long-mode-system-segment-descriptor-format}
\end{figure}

Базовый адрес 64-битного системного сегмента должен быть в канонической форме, иначе при
загрузке сегмента произойдет исключение \#GP. Выход за границу системного сегмента проверяется и
в 64-битном режиме и в режиме совместимости с учетом бита гранулярности <<G>>.

На рис.~\ref{fig:long-mode-system-segment-descriptor-format} показано, что биты 12:8 двойного слова +12
должны быть сброшены в 0. Эти биты соотвествуют биту <<S>> и полю <<тип>> в унаследованном дескрипторе.
Сброс этих бит в 0 соответствует неправильному типу дескриптора в унаследованном режиме и приведет
к возникновению исключения \#GP при попытке обратиться отдельно к старшей половине 64-битного системного сегмента.

\subsubsection*{Дескрипторы шлюзов}
В длинном режиме дескрипторы шлюзов увеличены на 64 бита, что позволяет хранить в них 64-битные смещения.
Формат 64-битного дескриптора шлюза прерывания и шлюза ловушки показан на рис.~\ref{fig:long-mode-interrupt-gate-descriptor}.

\begin{figure}[ht!]
  \centering
  \includegraphics[width=1.0\textwidth]{inc/dia/long-mode-interrupt-gate-descriptor}
  \caption{Формат дескриптора шлюза прерывания и шлюза ловушки в длинном режиме}
  \label{fig:long-mode-interrupt-gate-descriptor}
\end{figure}

Селектор сегмента (в дескрипторе шлюза) должен указывать на 64-битный сегмент кода (CS.L=1, CS.D=0).
В противном случае, при обращении к дескриптору произойдет исключение \#GP.

Исключение также возникает, если адрес, содержащийся в поле <<смещение>>, находится не в канонической форме.

В 64-битном режиме элементы таблицы векторов прерываний имеют размер 128 бит. Процессор автоматически
умножает номер вектора прерывания на 16 для определения смещения.

\paragraph{Поле <<IST>>.} Биты 2:0 байта +4. В длинном режиме в дескрипторах шлюзов прерываний и ловушек
появилось новое 3-битное поле -- IST. Данное поле используется в качестве индекса в IST фрагменте TSS длинного режима.
Если IST не равно 0 -- индекс ссылается на элемент IST в TSS, значение которого процессор загружает в регистр RSP при
возникновении прерывания. Если IST равно 0 -- процессор использует унаследованный механизм переключения стека.

\subsection{Защита сегментов}
Архитектура AMD64 разработана, чтобы полностью поддерживать унаследованный механизм защиты сегментов.
Данный механизм позволяет системному ПО ограничивать программам доступ к данным и коду других программ.

Защита на уровне сегментов включена в режиме совместимости. 64-битный режим устраняет часть проверок,
оставляя только проверки доступа к таблицам системных дескрипторов.

Предпочтительный метод организации защиты памяти в операционной системе длинного режима -- использование
механизма страничного преобразования.

\subsubsection*{Концепция уровней доступа}
Механизм защиты сегментов используется чтобы изолировать и защищать код и данные различных процессов. В защищенном
режиме данный механизм поддерживает 4 уровня привилегий. Уровни привилегий обозначаются номерами от 0 до 3,
где 0 обозначает наибольший уровень привилегий, 3 -- наименьший. Системное ПО обычно назначает уровни привилегий
следующим образом~\cite{amd_pm_v2}:
\begin{itemize}
\item Уровень 0. Этот уровень используется критичными компонентами системного ПО, которым
	необходим прямой доступ и контроль над всеми процессорными и системными ресурсами.
	Сюда входит встроенное ПО, функции управления памятью и обработка прерываний.
\item Уровень 1 и 2. Эти уровни используются менее критичными компонентами системного ПО,
	которым необходим доступ к ограниченному набору процессорных и системных ресурсов.
	Сюда входят драйверы устройств и библиотечные функции. Функции данного уровня могут
	использовать функции более высокого уровня привилегий для работы с памятью и файлами.
\item Уровень 3. Этот уровень используется прикладным ПО. Доступ к системным ресурсам осуществляется
	через системные вызовы.
\end{itemize}

На рис.~\ref{fig:privilege-levels} показаны отношения между уровнями привилегий.

\begin{figure}[ht!]
  \centering
  \includegraphics[width=0.7\textwidth]{inc/dia/privilege-levels}
  \caption{Концепция уровней доступа}
  \label{fig:privilege-levels}
\end{figure}

\subsubsection*{Типы уровней привилегий}
Существует три типа уровней привилегий, используемых процессором для контроля доступа к сегментам: CPL, DPL и RPL.

\paragraph{CPL.} Текущий уровень привилегий -- уровень привилегий, на котором процессор находится в данный
момент. Он хранится во внутреннем регистре процессора, недоступном для ПО. Изменить текущий уровень привилегий
можно путем передачи управления на сегмент кода с другим уровнем привилегий.

\paragraph{DPL.} Уровень привилегий дескриптора -- уровень привилегий, который системное ПО назначает сегменту (шлюзу).
Используется при проверке прав доступа, чтобы определить можно ли приложению обращаться к сегменту (шлюзу), на который
ссылается дескриптор. Данное поле хранится в дескрипторе сегмента (шлюза).

\paragraph{RPL.} Отражает уровень привилегий процесса, который создал селектор сегмента (шлюза). RPL может быть использован
в вызываемой программе, чтобы определить уровень привилегий вызывающей программы. Данное поле хранится
в селекторе сегмента (шлюза).

Как происходит проверка прав доступа с использованием CPL, DPL и RPL рассказано в \cite{amd_pm_v2}.

\section{Страничное преобразование}
Механизм страничного преобразования x86 позволяет системному ПО создавать отдельные адресные
пространства для различных процессов. Эти адресные пространства известны как
виртуальные адресные пространства. Системное ПО использует механизм страничного преобразования
для отображения вируальных страниц на физические, используя иерархию таблиц страничного преобразования,
известных как таблицы страниц.

Механизм страничного преобразования и таблицы страниц используются чтобы обеспечить
каждый процесс областью физической памяти для хранения кода и данных, недоступной другим процессам.
Процесс не может получить доступ к физической памяти, которая не отображена на его адресное
пространство системным ПО.

Системное ПО может использовать механизм страничного преобразования чтобы отобразить одну физическую
страницу в различные адресные пространства. Такие страницы можно делать доступными только для чтения,
чтобы предотвратить их модификацию процессами.

Общие страницы, как правило, используются для организации доступа к разделяемым библиотекам из
разных процессов. Доступная только для чтения копия библиотеки отображается в виртуальное адресное
пространство каждого процесса, однако в физической памяти находится только одна копия библиотеки.
Данная возможность также позволяет хранить копию операционной системы и различных драйверов устройств
в адресном пространстве процессов. Это позволяет избежать накладных расходов связанных с переключением
на другое адресное пространство при обращении к функциям ОС.

Область адресного пространства, отведенная для системного ПО, содержит системные данные, которые
должны быть недоступны для прикладного ПО. Системное ПО использует таблицы страниц для защиты этих
данных, отмечаяя такие страницы как доступные только супервизору, тем самым запрещая доступ к ним со
стороны прикладного (непривилегированного) ПО.

Системное ПО может использовать механизм страничного преобразования, чтобы отображать большие
виртуальные адресные пространства в меньший объем физической памяти. Каждому процессу доступно
32-битное или 64-битное виртуальное адресное пространство. Системное ПО использует свободные
физические страницы для отображения наиоблее часто используемых виртуальных страниц. Наименее часто
используемые виртуальные страницы выгружаются на диск.

\subsection{Механизм страничного преобразования}
Унаследованная архитектура x86 поддерживает преобразование 32-битных виртуальных адресов в 32-битные физические (
36-битные и 40-битные физические адреса доступны в специальных режимах). Архитектура AMD64 расширяет эту возможность,
позволяя преобразовывать 64-битные виртуальные адреса в 52-битные физические (различные реализации процессоров
могут поддерживать более короткие виртуальные и физические адреса).

Виртуальные адреса преобразуются в физические, используя иерархию таблиц страниц, созданную и управляемую системным ПО.
Элементы таблиц указывают на таблицы страниц следующего уровня. Одна таблица может содержать до 1024 элементов,
каждый из которых указывает на таблицу страниц следующего уровня. Элементы таблицы страниц заключительного уровня
указывают на физические страницы.

На рис.~\ref{fig:long-mode-page-translation} показана иерархия таблиц страниц, использующаяся в длинном режиме.
В унаследованном режиме механизм преобразования страниц использует подмножество данной иерархии (в унаследованном
режиме не сущесвует PML4, а наличие PDP зависит от используемого режима преобразования).
Как показано на рисунке, виртуальный адрес делится на несколько частей, каждая из которых используется как смещение
в соответствующей таблице страниц. Младшая часть виртуального адреса испльзуется как смещение в физической странице.

\begin{figure}[ht!]
  \centering
  \includegraphics[width=1.0\textwidth]{inc/dia/long-mode-page-translation}
  \caption{Преобразование виртуального адреса в физический в длинном режиме}
  \label{fig:long-mode-page-translation}
\end{figure}

\subsubsection*{Опции страничного преобразования (бит <<PG>>)}
Режим страничного преобразования зависит от того, какие опции были активированы. Существует
4 опции, влияющие на страничное преобразование:
\begin{itemize}
	\item Активация страничного преобразования (бит CR0.PG)
	\item Расширение физических адресов (бит CR4.PAE)
	\item Расширение размеров страниц (бит CR4.PSE)
	\item Активация длинного режима (бит EFER.LMA)
\end{itemize}

\subsubsection*{Активация страничного преобразования}
Страничное преобразование управляется битом <<PG>> регистра CR0 (бит 31). Если бит CR0.PG установлен в 1 --
страчное преобразование включено, иначе -- выключено. Архитектура AMD64 использует бит CR0.PG для активации
и деактивации длинного режима.

\subsubsection*{Расширение физических адресов (бит <<PAE>>)}
Возможность расширения физических адресов управляется битом <<PAE>> в регистре CR4 (бит 5). Если бит CR4.PAE
установлен в 1 -- использется расширение физических адресов, в противном случае -- не используется.

Установка CR4.PAE в 1 включает поддержку преобразования виртуальных адресов в 52-битные физические адреса.
Это приводит к увеличению длинны полей структур данных страничного преобразования с 32 бит до 64 (для
возможности хранения расширенных физических адресов).

\subsubsection*{Расширение размера страниц (бит <<PSE>>)}
Функция расширения размера страниц контролируется битом <<PSE>> в регистре CR4 (бит 4). Установка CR4.PSE в 1
позволяет системному ПО использовать 4-мегабайтные физические страницы. 4-мегабайтные страницы могут
использовать совместно с 4-килобайтными или вместо них.

Выбор между 2-мегабайтными и 4-мегабайтными физическими страницами определяется битами CR4.PAE и CR4.PSE
следующим образом:
\begin{itemize}
\item Если CR4.PAE=1 -- максимальный размер физических страниц составляет 2~мегабайта, независимо от значения CR4.PSE.
\item Если CR4.PAE=0 и CR4.PSE=1 -- максимальный размер физизических страниц составляет 4~мегабайта.
\item Если CR4.PAE=0 и CR4.PSE=0 -- доступны только 4-килобайтные физические страницы.
\end{itemize}

В длинном режиме значения бита CR4.PSE игнорируется, т.к. бит CR4.PAE должен быть установлен в 1.

\subsubsection*{Размер страницы в каталоге страниц (бит <<PS>>)}
В PDE и PDPE имеется бит <<PS>> (бит 7), который влияет на размер используемых физических страниц.

Если PDE.PS установлен в 1 -- каталог страниц становится таблицей страниц последнего уровня.
Размер страницы определяется в соответствии со значениями CR4.PAE и CR4.PSE. Если PDE.PS сброшен в 0 --
используются стандартные 4-килобайтные физические страницы.

Если PDPE.PS установлен в 1 -- PDP становится таблицей последнего уровня.
При этом используются физические страницы размеров 1~гигабайт.

\subsection{Страничное преобразование в унаследованном режиме}
В унаследованном режиме поддерживаются 2 формы страничного преобразования:
\begin{itemize}
\item Нормальное. Используется если CR4.PAE=0. При этом 32-битные элементы таблиц страниц
	используются для преобразования 32-битных виртуальных адресов в физические адреса длинной до 40 бит.
\item PAE. Используется если CR4.PAE=1. При этом 64-битные элементы таблиц страниц используются
	для преобразования 64-битных виртуальных адресов в физические адреса длинной до 52 бит.
\end{itemize}

В унаследованном режиме механиз страничного преобразования использует до 3х уровней таблиц страниц.
Сюда входят следующие таблицы:
\begin{enumerate}[1.]
	\item Таблица страниц. Каждый элемент таблицы страниц (PTE) указывает на физическую страницу. Данная таблица
		существует только если используются 4-килобайтные страницы.
	\item Каталог страниц. При использовании 4-килобайтных страниц -- элементы каталога страниц (PDE)
		указывают на таблицы страниц. При использовании 2 или 4-мегабайтных страниц -- PDE указывает
		сразу на физическую страницу. Если CR4.PAE=0 -- каталог страниц является таблицей страниц верхнего уровня.
	\item PDP. Каждый элемент этой таблицы (PDPE) указывает на каталог страниц.
		Используется только если CR4.PAE=1. В этом случае является таблице страниц верхнего уровня.
\end{enumerate}

\subsubsection*{Регистр CR3}
Регистр CR3 используется для хранения адреса таблицы страниц верхнего уровня (либо каталог страниц,
либо PDP). Формат регистра CR3 зависит от используемого режима преобразования.
На рис.~\ref{fig:legacy-cr3-non-pae} показан формат регистра CR3 в нормальном режиме (CR4.PAE=0).
Описание формата регистра CR3 при использовании PAE можно найти в ~\cite{amd_pm_v2}.

\begin{figure}[ht!]
  \centering
  \includegraphics[width=1.0\textwidth]{inc/dia/legacy-cr3-non-pae}
  \caption{Формат регистра CR3 в унаследованном режиме (без PAE)}
  \label{fig:legacy-cr3-non-pae}
\end{figure}

Регистр CR3 в унаследованном режиме имеет следующие поля:
\begin{itemize}
\item Базовый адрес. Содержит физический адрес таблицы страниц верхнего уровня. В случае нормального
	преобразования (CR4.PAE=0) -- это 20-битное поле, занимает биты 31:12 и указывает на начало
	каталога страниц. Адрес каталога страниц выровнен по 4-килобайтной границе (младшие 12 бит равны 0).
\item Бит <<PWT>> (3). Бит сквозной записи. Определяет политику кеширования. Если установлен в 1 --
	используется сквозная запись, в противном случае -- отложенная.
\item Бит <<PCD>> (4). Бит отключения кеширования. Если установлен в 1 -- таблица страниц кешируется.
\item Зарезервированные поля. Должны быть сброшены в 0 при записи нового значения в регистр CR3.
\end{itemize}

\subsubsection*{Нормальное преобразование страниц (без PAE)}
При использовании физических страниц размером 4~килобайта виртуальный адрес делится на 3 части.
Как показано на рис.~\ref{fig:legacy-4kb-non-pae-page-translation} поля виртуального адреса
используются следующим образом:
\begin{itemize}
\item Биты 31:22 -- индекс в каталоге страниц.
\item Биты 21:12 -- индекс в таблице страниц.
\item Биты 11:0 -- смещение в физической странице.
\end{itemize}

\begin{figure}[ht!]
  \centering
  \includegraphics[width=1.0\textwidth]{inc/dia/legacy-4kb-non-pae-page-translation}
  \caption{Использование 4-килобайтных страниц в унаследованном режиме (без PAE)}
  \label{fig:legacy-4kb-non-pae-page-translation}
\end{figure}

На рис.~\ref{fig:legacy-4kb-pde-non-pae} показан формат элемента каталога страниц (PDE),
а на рис.~\ref{fig:legacy-4kb-pte-non-pae} -- формат элемента таблицы страниц (PTE).
Каждая таблица имеет размер 4~килобайта и содержит 1024 32-битных элемента. Поля этих элементов
будут описаны в ~\ref{subsec:page_table_fields}.

\begin{figure}[ht!]
  \centering
  \includegraphics[width=1.0\textwidth]{inc/dia/legacy-4kb-pde-non-pae}
  \caption{Элемент каталога страниц при использовании 4-килобайтных страниц (без PAE)}
  \label{fig:legacy-4kb-pde-non-pae}
\end{figure}

\begin{figure}[ht!]
  \centering
  \includegraphics[width=1.0\textwidth]{inc/dia/legacy-4kb-pte-non-pae}
  \caption{Элемент таблицы страниц при использовании 4-килобайтных страниц (без PAE)}
  \label{fig:legacy-4kb-pte-non-pae}
\end{figure}

\subsection{Страничное преобразование в длинном режиме}
Для работы страничного преобразования в длинном режиме необходимо использовать PAE.
CR4.PAE должен быть установлен в 1 до активации длинного режима, в противном случае
произойдет исключение \#GP.

Структуры данных страничного преобразования при включенном PAE позволяют отображать 64-битные
виртуальные адреса в 52-битные физические. PAE расширяет элементы каталога страниц (PDE) и
элементы таблицы страниц (PTE) до 64 бит, позволяя использовать физические адреса длинной более 32 бит.

Архитектура AMD64 расширяет формат PDPE, определяя ранее зарезервированные биты.
Также была добавлена таблица страниц 4го уровня (PML4).

Так как в длинном режиме PAE всегда включен -- бит <<PS>> в элементах каталога страниц (PDE.PS)
позволяет выбирать между 4-килобайтными и 2-мегабайтными страницами. Значение бита CR4.PSE игнорируется.

\subsubsection*{Регистр CR3}
В длинном режиме регистр CR3 используется для указания базового адреса PML4. Размер регистра CR3 был
увеличен до 64 бит, чтобы иметь возможность хранить PML4 в произвольном месте 52-битного физического
адресного пространства. На рис.~\ref{fig:long-mode-cr3} показан формат регистра CR3 в длинном режиме.

\begin{figure}[ht!]
  \centering
  \includegraphics[width=1.0\textwidth]{inc/dia/long-mode-cr3}
  \caption{Формат регистра CR3 в длинном режиме}
  \label{fig:long-mode-cr3}
\end{figure}

Все поля регистра CR3 интерпретируются так же, как и в унаследованном режиме.

\subsubsection*{Использование 4-килобайтных страниц}
При использовании 4-килобайтных физических страниц в длинном режиме виртуальный адрес делится на 6 частей.
4 из них используются в качестве индексов в таблицах страниц. Как показано на рис.~\ref{fig:long-mode-4kb-page-translation},
поля виртуального адреса используются следующим образом:
\begin{itemize}
\item Биты 63:48 -- знаковое расширение бита 47, как этого требует каноническая форма адреса.
\item Биты 47:39 -- индекс в PML4.
\item Биты 38:30 -- индекс в PDP.
\item Биты 29:21 -- индекс в каталоге страниц.
\item Биты 20:12 -- индекс в таблице страниц.
\item Биты 11:0 -- смещение в физической странице.
\end{itemize}

\begin{figure}[ht!]
  \centering
  \includegraphics[width=1.0\textwidth]{inc/dia/long-mode-4kb-page-translation}
  \caption{Использование 4-килобайтных страниц в длинном режиме}
  \label{fig:long-mode-4kb-page-translation}
\end{figure}

На рис.~\ref{fig:long-mode-4kb-pml4e} показан формат элемента PML4.
На рис.~\ref{fig:long-mode-4kb-pdpe} показан формат элемента PDP.
На рис.~\ref{fig:long-mode-4kb-pde} показан формат элемента каталога страниц.
На рис.~\ref{fig:long-mode-4kb-pte} показан формат элемента таблицы страниц.

Каждая таблица имеет размер 4~килобайта и содержит 512 64-битных элементов. Поля этих элементов
описаны в ~\ref{subsec:page_table_fields}.

\begin{figure}[ht!]
  \centering
  \includegraphics[width=1.0\textwidth]{inc/dia/long-mode-4kb-pml4e}
  \caption{Элемент PML4 при использовании 4-килобайтных страниц в длинном режиме}
  \label{fig:long-mode-4kb-pml4e}
\end{figure}

\begin{figure}[ht!]
  \centering
  \includegraphics[width=1.0\textwidth]{inc/dia/long-mode-4kb-pdpe}
  \caption{Элемент PDP при использовании 4-килобайтных страниц в длинном режиме}
  \label{fig:long-mode-4kb-pdpe}
\end{figure}

\begin{figure}[ht!]
  \centering
  \includegraphics[width=1.0\textwidth]{inc/dia/long-mode-4kb-pde}
  \caption{Элемент каталога страниц при использовании 4-килобайтных страниц в длинном режиме}
  \label{fig:long-mode-4kb-pde}
\end{figure}

\begin{figure}[ht!]
  \centering
  \includegraphics[width=1.0\textwidth]{inc/dia/long-mode-4kb-pte}
  \caption{Элемент таблицы страниц при использовании 4-килобайтных страниц в длинном режиме}
  \label{fig:long-mode-4kb-pte}
\end{figure}

\subsection{Поля элементов таблиц страниц}
\label{subsec:page_table_fields}
Элементы таблиц страниц содержат контрольные и информационные поля, используемые для управления
страницами в виртуальной памяти. Большинство полей являются общими для элементов таблиц страниц
разных уровней и занимают одни и теже позиции. Однако положение некоторых полей меняется в
зависимости от положения таблицы страниц в общей иерархии. Некоторые поля имеют разные размеры
в зависимости от размера страницы, длинны физического адреса и режима работы процессора. Несмотря на
то, что позиция и размер этих полей может отличаться, их значение остается неизменным для всех
уровней иерархии, во всех режимах работы.

\paragraph{Базовый адрес таблицы страниц.} Содержит физический адрес таблицы страниц следующего уровня.
Адреса таблиц страниц всегда выровнены по границе 4 килобайта, поэтому необходимо хранить только биты, старше 11 (
т.к. биты 11:0 всегда равны 0).

\paragraph{Базовый адрес страницы.} Содержит базовый адрес физической страницы. Страницы могут иметь размер 4~Кб, 2~Мб,
4~Мб или 1~Гб. Их адреса всегда выровнены по границе равной размеру страницы.

\paragraph{Бит <<P>>.} Бит 0. Показывает находится ли таблица/физическая страница в памяти. Если сброшен в 0 --
таблицы/физической страницы в памяти нет.

ОС сбрасывает это бит, чтобы показать, что таблица/страница выгружена. При доступе к такой таблице/странице произойдет страничное
исключение (\#PF). ОС должна загрузить отсутсвующую страницу/таблицу в память и установить данный бит в 1.

Когда этот бит сброшен в 0 -- все остальные биты доступты для ОС. Такие страницы никогда не модифицируются процессором и
не попадают в TLB.

\paragraph{Бит <<R/W>>.} Бит 1. Этот бит контролирует возможность чтения/записи для всех физических страниц, отображенных
данным элементом таблицы. Если бит установлен в 1 -- страницы доступны для чтения и записи, в противном случае --
только для чтения.

\paragraph{Бит <<U/S>>.} Бит 2. Этот бит контролирует возможность доступа непривилегированного ПО (CPL=3) ко всем страницам,
отображенным данным элементом таблицы. Если бит установлен в 1 -- доступ разрешен для всех уровней привелений,
в противном случает -- только для уровней привилегий 0, 1 и 2.

\paragraph{Бит <<PWT>>.} Бит 3. Определяет политику кеширования таблицы/физической страницы, на которую указывает
элемент текущей таблицы. Если бит установлен в 1 -- используется политика сквозной записи, в противном случае -- отложенной.

\paragraph{Бит <<PCD>>.} Бит 4. Определяет является ли таблица/физическая страница, на которую указывает элемент
текущей таблицы, кешируемой. Если бит установлен в 1 -- таблица/физическая страница не кешируются.

\paragraph{Бит <<A>>.} Бит 5. Показывает был ли доступ к таблице/странице, на которую указывает элемент текущей таблицы.
Процессор устанавливает данный бит в 1 при обращении (чтение или запись) к таблице/странице. Процессор никогда не
сбрасывает этот бит. Системное ПО должно само сбрасывать этот бит при необходимости отслеживать частоту обращения к
таблице/странице.

\paragraph{Бит <<D>>.} Бит 6. Данный бит присутствует только в элементах таблицы страниц нижнего уровня. Он показывает была ли
страница, на которую укаывает элемент таблицы, модифицирована. Процессор устанавливает данный бит при записи в
физическую страницу. Процессор никогда не сбрасывает данный бит. Системное ПО должно само сбрасывать этот бит
при необходимости отслеживания частоты модификации физической страницы.

\paragraph{Бит <<PS>>.} Бит 7. Данный бит присутствует в PDE и PDPE длинного режима.
Если PDPE.PS или PDE.PS установлен в 1 -- данный элемент является заключительным в
иерархии таблиц страничного преобразования и содержит адрес физической страницы.

Размер физической страницы определяется следующим образом:
\begin{itemize}
\item Если EFER.LMA=1 и PDPE.PS=1 -- размер физической страницы составляет 1~гигабайт.
\item Если CR4.PAE=0 и PDE.PS=1 -- размер физической страницы составляет 4~мегабайта.
\item Если CR4.PAE=1 и PDE.PS=1 -- размер физической страницы составляет 2~мегабайта.
\end{itemize}

\paragraph{Бит <<G>>.} Бит 8. Данный бит присутствует только в элементах таблицы страниц нижнего уровня. Он показывает является
ли страница, на которую указывает элемент таблицы, глобальной. Если бит установлен в 1 -- страница является
глобальной, такая страница не удаляется из кеша TLB при загрузке нового значения в регистр CR3.

\paragraph{Биты <<AVL>>.} Данные биты не интерпретируются процессором и доступны для использования системным ПО.

\paragraph{Бит <<PAT>>.} Используется механизмом страничных атрибутов.

\paragraph{Бит <<NX>>.} Бит 63. Дает возможность запрешать исполнение кода для страниц, отображенных текущим элементом таблицы.

\paragraph{Зарезервированные биты.} Системное ПО должно сбрасывать зарезервированные биты в 0. Иначе возникнет
страничное исключение (\#PF).

\subsection{Кеш TLB}
При включенном страничном преобразовании виртуальные адреса автоматически преобразуются в физические,
используя иерархию таблиц страниц. Кеши TLB, служат для уменьшения накладных расходов, связанных с преобразованием.
TLB -- аппаратный кеш, который содержит результаты последних отображений виртуальных адресов в физические.
При каждом обращении к памяти, адрес проверяется в TLB. Если отображение найдено -- результат немедленно возвращается
процессору, это позволяет избежать лишних обращений к памяти (для доступа к таблицам страниц).

Системное ПО должно инвалидировать записи в TLB при внесении изменений
в структуры данных страничного преобразования.

\section{Сегмент состояния задачи (TSS)}
\subsection{Аппаратная многозадачность}
Задача (также называемая процессом) это программа, которую процессор может
исполнять, приостанавливать и позже продолжать исполнение с места последней
остановки. Пока задача приостановлена могут исполняться другие задачи. Каждая
задача имеет свой контекст, в который входят:
\begin{enumerate}[1.]
\item Сегмент кода и rIP (указатель на следующую инструкцию)
\item Сегменты данных
\item Сегменты стека для всех уровней привилегий
\item Регистры общего назначения
\item Регистр флагов (rFLAGS)
\item Локальная таблица дескрипторов
\item Регистр задачи (TR) и указатель на предыдущую задачу
\item Битовые карты разрешений ввода/вывода и прерываний
\item Указатель на таблицу страничного преобразования верхнего уровня (CR3)
\end{enumerate}

В защищенном режиме эта информация сохраняется в сегменте состояния задачи (TSS)
для возможности аппаратного переключения задач~\cite[стр. 327]{amd_pm_v2}.

\subsection{Ресурсы управления задачами}
В длинном режиме аппаратное переключение задач не поддерживается, но
операционная система по прежнему должна инициализировать некоторые ресурсы
управления задачами:
\begin{enumerate}[1.]
\item Сегмент состояния задачи (TSS) -- сегмент, в котором хранится состояние
	процессора, связанное задачей.
\item Дескриптор TSS -- дескриптор сегмента, описывающий TSS.
\item Селектор TSS -- селектор сегмента, который ссылается на дескриптор TSS,
	расположенный в GDT.
\item Регистр задачи (TR) -- регистр, в котором хранится селектор и дескриптор
	TSS текущей задачи.
\end{enumerate}

На рис.~\ref{fig:task_management_resources} показана связь между этими ресурсами в длинном режиме.
\begin{figure}[ht!]
  \centering
  \includegraphics[width=\textwidth]{inc/dia/task-management-resources}
  \caption{Ресурсы управления задачами}
  \label{fig:task_management_resources}
\end{figure}

\subsubsection*{Селектор TSS}
Селекторы TSS это селекторы, которые указывают на дескрипторы TSS в GDT.
Формат селектора TSS совпадает с форматом других сегментных селекторов и
показан на рис.~\ref{fig:tss_selector}.

\begin{figure}[ht!]
  \centering
  \includegraphics[width=0.4\textwidth]{inc/dia/tss-selector}
  \caption{Формат селектора TSS}
  \label{fig:tss_selector}
\end{figure}

Селектор состоит из следующих полей:
\begin{enumerate}[1.]
\item Индекс селектора. Биты 15:3. Определяет смещение дескриптора TSS в GDT.
\item Индикатор таблицы (TI). Бит 2. Данный бит должен быть сброшен в 0, это
	означает что селектор ссылается на запись в GDT. Дескрипторы TSS
	запрещено размещать в LDT и при попытки обратиться к дескриптору TSS,
	расположенному в LDT произойдет исключение \#GP.
\item Уровень привилегий (RPL). Биты 1:0. Равен уровню привилегий на котором
	находился процессор при загрузке селектора TSS в регистр задачи (TR).
\end{enumerate}

\subsubsection*{Дескриптор TSS}
Дескриптор TSS это системный дескриптор, который может быть расположен только
в GDT. Его формат описан ранее в \ref{subsec:system_desriptor_format}.

TSS должен иметь размер минимум 104 байта (т.е. минимальное значение поля <<предел>> -- 103).
Если размер TSS меньше -- при переключении задач будет сгенерировано исключение \#TS.

\subsubsection*{Регистр задачи (TR)}
Регистр задачи хранит адрес TSS, определяет его размер и атрибуты. TR состоит
из двух частей -- видимой и теневой. В видимой (доступной для ПО) части хранится селектор TSS,
теневая часть содержит дескриптор TSS. При загрузке селектора TSS в TR
процессор автоматически загружает дескриптор TSS из GDT в теневую часть.
Загрузка нового значения в TR осуществляется инструкцией LTR. На
рис.~\ref{fig:task_register} показан формат TR в длинном режиме.

\begin{figure}[ht!]
  \centering
  \includegraphics[width=.8\textwidth]{inc/dia/task-register}
  \caption{Формат регистра задачи (TR)}
  \label{fig:task_register}
\end{figure}

\subsubsection*{Формат TSS}
Несмотря на то, что аппаратный механизм переключения задач не поддерживается в
длинном режиме, 64х-битный TSS должен быть объявлен. Системное ПО должно
создать как минимум один TSS для использования в длинном режиме и
выполнить инструкции LTR в 64х-битном режиме, чтобы загрузить в TR 64-х битный TSS.

На рис.~\ref{fig:tss} показан формат TSS в длинном режиме. TSS включает
следующую информацию:
\begin{enumerate}[1.]
\item RSPn -- Байты 0x1B-0x04. 64-битные адреса указателей стека (RSP) для
	уровней привилегий 0,1 и 2.
\item ISTn -- Байты 0x5B-0x24. 64-битные адреса указателей стека, используемые
	механизмом IST.
\item Базовый адрес битовой карты разрешений ввода/вывода. Байты 0x67-0x66.
\end{enumerate}

\begin{figure}[ht!]
  \centering
  \includegraphics[width=\textwidth]{inc/dia/tss}
  \caption{Формат TSS}
  \label{fig:tss}
\end{figure}



\section{Исключения и прерывания}
\label{sec:exceptions_and_interrupts}

Исключения и прерывания приводят к передаче управления функциям операционной
системы. Эти функции называются обработчиками исключений и прерываний.
Как правило, прерывания обрабатываются незаметно для прерванного
процесса. Перед передачей управления обработчику исключения/прерывания
процессор сохраняет в его стеке rIP прерванной инструкции (адрес возврата).
Обработчик прерывания/исключения должен сохранить контекст прерванного процесса,
для возможности его последующего продолжения после завршения обработчика.

При возникновении прерывания или исключения процессор использует номер
прерывания как индекс в IDT. IDT используется во всех режимах работы процессора.

Прерывания можно резделить на 3 категории~\cite{amd_pm_v2}: исключения (англ. Exceptions),
программные прерывания (англ. Software Interrupts) и внешние прерывания (англ. External Interrupts).

Исключения возникают в результате ошибок во время исполнения ПО или внутренних
ошибок процессора. Исключения также могут возникать без наличия ошибочных
ситуаций, например при пошаговом выполнении программы. Исключения считаются
синхронными событиями, т.к. они возникают в результате исполнения прерванной
инструкции.

Программные прерывания возникают в результате вызова прерывания (int). В
отличие от исключений и внешних прерываний, программные прерывания позволяют
намеренно вызывать обработчики прерываний. Как и исключения, программные
прерывания являются синхронными событиями.

Внешние прерывания генерируются системой в результате возникновения ошибки
или какого-либо события вне процессора. Они доставляются на шину процессора
используя внешние сигналы. Внешние прерывания являются асинхронными событиями,
т.к. они возникают независимо от прерванной инструкции.

\subsection{Основные характеристики}
Исключения и прерывания имеют несколько различных характеристик, которые
зависят от того как они были доставлены и определяют как следует продолжить выполнение
прерванного процесса.

\subsubsection*{Точность}
По отношению к прерванной инструкции прерывания делятся на:
\begin{itemize}
\item Точные. Возникают на предсказуемой границе. Эта граница
как правило -- первая инструкция, которая не была выполнена во время
возникновения прерывания. Все предыдущие инструкции были завершены перед
передачей управления обработчику прерывания. Указатель на граничную инструкцию
автоматически сохраняется процессором. После завершения обработчика, выполнение
процесса может быть продолжено с сохраненной инструкции.
\item Неточные. Не гарантируется возникновение на предсказуемой границе.
Границей может быть любая инструкция, которая не была завершена во время
возникновения прерывания. Неточные события можно считать асинхронными, т.к.
источник прерывания может быть не связан с исполняемой инструкцией.
Обработчики неточных прерываний и исключений обычно собирают информацию о
состоянии системы и передают специально программному обеспечению для
диагностики. Выполнение прерванной программы не продолжается.
\end{itemize}

\subsubsection*{Перезапуск инструкций}
Как было сказано выше, точные исключения возникают на границе инструкции. Эта
граница может находиться либо перед, либо после инструкции.

Для большинства исключений граничной является инструкция вызвавшая исключение,
т.е. граница устанавливается перед этой инструкцией. В этом случае все
инструкции предсшествующие данной (в програмном порядке) считаются
выполненными.

Есть исключения для которых граничной является инструкция следующая за той,
которая вызвала исключение. В этом случае все предыдущие инструкции, включая
ту, которая вызывала исключение, считаются выполненными.

\subsubsection*{Типы исключений}
Исключения можно разделить на 3 типа в зависимости от того, являются ли они
точными и как они влияют на продолжение работы программы:
\begin{enumerate}[1.]
\item Ошибки (англ. Faults) -- являются точными исключениями возникающими на границе перед
инструкцией, вызвавшей исключение. Все изменения, вызванные данной инструкцией
отменются, поэтому инструкция может быть выполнена повторно.
\item Ловушки (англ. Traps) -- являются точными исключениями, возникающими на границе после
инструкции, вызвавшей исключение. Процессор полностью выполнил инструкцию,
вызвавшую исключение, т.о. все изменения, вызванные инструкцией сохранены.
rIP указывает на инструкцию, следующую за той, которая вызвала исключение.
\item Аварии (англ. Aborts) -- неточные инструкции, поэтому обычно они не позволяют
продолжить правильное выполнение процесса.
\end{enumerate}

\subsubsection*{Маскирование внешних прерываний}
Программное обеспечение может маскировать некоторые прерывания и исключения.
Маскирование может задержать или не допустить запуск механизма обработки
прерываний при возникновении прерывания. Внешние прерывания можно разделить на
маскируемые и немаскируемые.

Маскируемые прерывания приводят к вызову обработчика прерывания только если
rFLAGS.IF=1. В противном случае они не доставляются до тех пор, пока rFLAGS.IF
не станет равным 0.

Немаскируемые прерывания (NMI) обрабатываются независимо от значения rFLAGS.IF.
Однако при возникновении NMI прерывания, последующие NMI прерывания будут
замаскированы до выполнения инструкции IRET.

\subsubsection*{Маскирование при переключение стека}
Процессор откладывает обработку маскируемых прерываний и отладочных исключений
при выполнении определенных последовательностей инструкций, которые обычно
используются программным обеспечением для переключения стека. Стандартная
последовательность, используемая при переключении стека выглядит следующим
образом:
\begin{enumerate}[1.]
\item Загруить селектор сегмента стека в регистр SS
\item Загрузить смещение начала стека в регистр rSP
\end{enumerate}

Если прерывание возникает после того, как загружен SS, но до загрузки rSP -
указатель стека прерванной программы будет неверным во время выполнения
обработчика прерывания.

Чтобы недопустить проблем с указателем стека, вызванных прерываниями,
процессор не обрабатывает внешние прерывания и отладночные исключения пока
инструкция следующая за MOV SS или POP SS не будет выполнена.

\subsection{Векторы прерываний}
Определенным исключениям и прерываниям назначены фиксированные номера (также
называемые векторами прерываний или просто векторами). Вектор прерывания
используется механизмом обработки прерываний для определения точки входа в
обработчик прерывания. Можно использовать до 256 векторов прерываний. Первые 32 вектора
зарезервированы для преопределенных прерываний и исключительных ситуаций.
Для программных прерываний можно использовать любые из свободных векторов.

В таблице~\ref{tab:interrupts} перечислены поддерживаемые номера прерываний, их названия,
мнемонические обозначения и краткое описание.

\begin{center}
    \begin{longtable}{|c|c|c|c|}
    \caption{Описание прерываний}
    \label{tab:interrupts}
    \\ \hline
    \thead{Вектор} & \thead{Название} & \thead{Мнемоническое \\ обозначение} & \thead{Причины \\ возникновения} \\
    \hline \endfirsthead
    \subcaption{Таблица~\ref{tab:interrupts} -- Описание прерываний (продолжение)}
    \\ \hline
    \thead{Вектор} & \thead{Название} & \thead{Мнемоническое \\ обозначение} & \thead{Причины \\ возникновения} \\
    \hline \endhead
    \hline \subcaption{Продолжение на след. стр.}
    \endfoot
    \hline \endlastfoot
    0   & Divide by Zero Error & \#DE & \makecell{Инструкции DIV, \\ IDIV, AAM} \\
    \hline
    1   & Debug & \#DB & \makecell{Доступ к инструкциям \\ и данным} \\
    \hline
    2   & Non Maskable Interrupt & \#NMI & Внешний сигнал NMI \\
    \hline
    3   & Breakpoint & \#BP & Инструкция INT3 \\
    \hline
    4   & Overflow & \#OF & Инструкция INTO \\
    \hline
    5   & Bound Range & \#BR & Инструкция BOUND \\
    \hline
    6   & Invalid Opcode & \#UD & Неверные инструкции \\
    \hline
    7   & Device Not Available & \#NM & Инструкции x87 \\
    \hline
    8   & Double-Fault & \#DF & \makecell{Исключение, возникшее \\ во время обработки \\
    другого исключения \\ или прерывания} \\
    \hline
    9   & \makecell{Coprocessor Segment \\ Overrun} & --- & \makecell{Не поддерживается \\
    (зарезервировано)} \\
    \hline
    10  & Invalid-TSS & \#TS & \makecell{Доступ к сементу \\ состояния задачи (TSS)} \\
    \hline
    11  & Segment Not Present & \#NP & \makecell{Загрузка сегментного \\ регистра} \\
    \hline
    12  & Stack & \#SS & \makecell{Загрузка SS и обращение \\ к стеку} \\
    \hline
    13  & General-Protection & \#GP & \makecell{Доступ к памяти, \\ различные проверки} \\
    \hline
    14  & Page-Fault & \#PF & \makecell{Доступ к памяти \\ при включенном страничном \\
    преобразовании} \\
    \hline
    15  & Зарезервировано & --- & --- \\
    \hline
    16  & \makecell{x87 Floating-Point \\ Exception-Pending} & \#MF & Инструкции x87 \\
    \hline
    17  & Alignment-Check & \#AC & \makecell{Доступ к памяти \\ по невыровненным адресам} \\
    \hline
    18  & Machine-Check & \#MC & Моделезависимые ошибки \\
    \hline
    19  & SIMD Floating-Point & \#XF & SSE инструкции \\
    \hline
    20-29  & Зарезервировано & --- & --- \\
    \hline
    30  & Security Exception & \#SX & \makecell{События, связанныие \\ с безопасностью} \\
    \hline
    31  & Зарезервировано & --- & --- \\
    \hline
    0-255  & \makecell{External Interrupts \\ (Maskable)} & \#INTR & Внешние прерывания \\
    \hline
    0-255  & Software Interrupts & --- & Инструкции INTn \\
    \hline
  \end{longtable}
\end{center}

В таблице~\ref{tab:interrupts_classification} приведена классификация векторов
прерываний.

\begin{center}
    \begin{longtable}{|c|c|c|c|}
    \caption{Классификация векторов прерываний}
    \label{tab:interrupts_classification}
    \\ \hline
    Вектор & Название & Тип & Точность \\
    \hline \endfirsthead
    \subcaption{Таблица~\ref{tab:interrupts_classification} -- Классификация векторов прерываний (продолжение)}
    \\ \hline
    Вектор & Название & Тип & Точность \\
    \hline \endhead
    \hline \subcaption{Продолжение на след. стр.}
    \endfoot
    \hline \endlastfoot
    0   & Divide-by-Zero-Error & Ошибка & Точное \\
    \hline
    1   & Debug & Ошибка или ловушка & Точное \\
    \hline
    2   & Non-Maskable-Interrupt & -- &  -- \\
    \hline
    3   & Breakpoint & Ловушка & Точное \\
    \hline
    4   & Overflow & Ловушка & Точное \\
    \hline
    5   & Bound-Range & Ошибка & Точное \\
    \hline
    6   & Invalid-Opcode & Ошибка & Точное \\
    \hline
    7   & Device-Not-Available & Ошибка & Точное \\
    \hline
    8   & Double-Fault & Авария & Не точное \\
    \hline
    9   & Coprocessor Segment Overrun & --- & --- \\
    \hline
    10  & Invalid-TSS & Ошибка & Точное \\
    \hline
    11  & Segment-Not-Present & Ошибка & Точное \\
    \hline
    12  & Stack & Ошибка & Точное \\
    \hline
    13  & General-Protection & Ошибка & Точное \\
    \hline
    14  & Page-Fault & Ошибка & Точное \\
    \hline
    15  & Зарезервировано & --- & --- \\
    \hline
    16  & x87 Floating-Point Exception-Pending & Ошибка & Не точное \\
    \hline
    17  & Alignment-Check & Ошибка & Точное \\
    \hline
    18  & Machine-Check & Авария & Не точное \\
    \hline
    19  & SIMD Floating-Point & Ошибка & Точное \\
    \hline
    20-29  & Зарезервировано & --- & --- \\
    \hline
    30  & Security Exception & --- & Точное \\
    \hline
    31  & Зарезервировано & --- & --- \\
    \hline
    0-255  & External Interrupts (Maskable) & --- & --- \\
    \hline
    0-255  & Software Interrupts & --- & --- \\
    \hline
  \end{longtable}
\end{center}

\paragraph{\#DE Деление на 0 (Вектор 0).}
Возникает когда делитель в инструкциях DIV и IDIV равен 0. Так же возникает,
если результат не помещается в регистры назначения.

Не возвращает код ошибки. Сохраненный rIP указывает на инструкцию, вызвавшую
исключение.

\paragraph{\#DB Отладка (Вектор 1).}
Подробно описано в \cite{amd_pm_v2}. Не возвращает код ошибки.

\paragraph{NMI Немаскируемое прерывание (Вектор 2).}
NMI возникает когда системное окружение сигнализирует процессору о наличии
немаскируемого прерывания. Не возвращает код ошибки.

\paragraph{\#BP Точка останова (Вектор 3).}
Возникает при выполнении инструкции INT3. Не возвращает код ошибки.

\paragraph{\#OF Переполнение (Вектор 4).}
Возникает в результате выполнения инструкции INTO если в RFLAGS установлен бит
переполнения (RFLAGS.OF=1). Не возвращает код ошибки.

\paragraph{\#BR Выход за границы (Вектор 5).}
Может возникнуть в результате выполнения инструкции BOUND. Не возвращает код ошибки.

\paragraph{\#UD Неверная операция (Вектор 6).}
Данное исключение возникает при попытке выполнить неправильную или
неопределенную операцию. Не возвращает код ошибки.

\paragraph{\#NM Устройство недоступно (Вектор 7).}
Не возвращает код ошибки.

\paragraph{\#DF Двойная ошибка (Вектор 8).}
Может возникнуть, когда второе исключение возникло, во время обработки
первого. Возвращает 0 в качесте кода ошибки.

\paragraph{\#TS Неверный TSS (Вектор 10).}
Возвращает код ошибки в формате селектора. Подробно описано в \cite[стр. 222]{amd_pm_v2}.

\paragraph{\#NP Отсутствует сегмент (Вектор 11).}
Возникает при попытке загрузить сегмент или шлюз у которого сброшен бит <<P>>.
Возвращает код ошибки в формате селектора.

\paragraph{\#SS Ошибка со стеком (Вектор 12).}
Возвращает код ошибки в формате селектора.

\paragraph{\#GP Нарушение защиты (Вектор 13).}
Возникает при нарушении защиты или неверном использовании функций AMD64.
Возвращает код ошибки в формате селектора.

\paragraph{\#PF Страничная ошибка (Вектор 14).}
Может возникнуть при доступе к памяти, в одной из следующих ситуаций:
\begin{enumerate}[1.]
\item Элемент таблицы страничного преобразования, задействованный в
преобразовании адреса, отсутствует в физической памяти.
\item Попытка процессора выполнить инструкцию из неисполняемой страницы.
\item При доступе к памяти была нарушена одна из проверок
(пользователь/супервизор, чтение/запись, или обе).
\item Зарезервированный бит в одном из элементов таблицы страничного
преобразования установлен в 1.
\end{enumerate}

Процессор сохраняет виртуальный адрес, вызвавший страничное
исключение в регистре CR2. Формат кода ошибки показан на
рис.~\ref{fig:page_fault_error_code}.

\paragraph{\#MF Исключение с плавающей точкой (Вектор 16).}
Не возвращает код ошибки.

\paragraph{\#AC Ошибка выравнивания (Вектор 17).}
Данное исключение возникает при обращении к невыровненным данным, если
включена проверка выравнивания. Возвращает 0 в качестве кода ошибки.

\paragraph{\#MC Machine-check (Вектор 18).}
Не возвращает код ошибки.

\paragraph{\#XF SIMD исключение с плавающей точнкой (Вектор 19).}
Не возвращает код ошибки.

\paragraph{\#SX Исключение защиты (Вектор 30).}
Возвращает код ошибки.

\paragraph{Прерывания определнные пользователем (Вектора 32-255).}
Возникают в следующих случаях:
\begin{enumerate}[1.]
\item Процессору поступает сигнал о наличии внешнего прерывания.
\item Программное обеспечение выполняет инструкцию INTn, в которой n
определяет номер прерывания.
\end{enumerate}

Не возвращают код ошибки. Значение rIP зависит от источника прерывания:
\begin{enumerate}[1.]
\item Обработчики внешних прерываний вызываются на границах инструкций. Сохраненный rIP
	указывает на прерванную инструкцию (не выполненную).
\item Если прерывание возникло в результате выполнения инструкции INTn,
сохраненный rIP указывает на инструкцию, следующую за INTn.
\end{enumerate}

Внешние прерывания можно замаскировать, установив rFLAGS.IF=0. Программные
прерывания нельзя отключить.

\subsection{Коды ошибок}
При обработке некоторых исключений процессор использует коды ошибок. Код
ошибки помещается в стек перед вызовом обработчика исключения. Код ошибки
может быть в двух форматах: в формате селектора и в формате страничного исключения.

\subsubsection*{Формат селектора}
На рис.~\ref{fig:selector_error_code} показан код ошибки в формате селектора.
\begin{figure}[ht!]
  \centering
  \includegraphics[width=\textwidth]{inc/dia/selector-error-code}
  \caption{Код ошибки в формате селектора}
  \label{fig:selector_error_code}
\end{figure}

Код ошибки в формате селектора содержит следующую информацию:
\begin{enumerate}[1.]
\item Бит <<EXT>> -- бит 0. Если этот бит установлен в 0, значит источник
	исключения находится в процессоре.
\item Бит <<IDT>> -- бит 1. Если этот бит установлен в 1, значит поле <<индекс
селектора>> ссылается на дескриптор шлюза, расположенный в IDT. В
противном случае - поле <<индекс селектора>> ссылается на дескриптор
расположенный либо в GDT, либо в LDT (определяется битом TI).
\item Бит <<TI>> -- бит 2. Если этот бит установлен в 1, значит поле <<индекс
селектора>> ссылается на дескриптор, расположенный в LDT, в противном случае -
в GDT.
\item Индекс селектора -- биты 15:3. Определяет индекс в GDT, LDT или IDT.
\end{enumerate}

\subsubsection*{Формат при страничном исключении}
На рис.~\ref{fig:page_fault_error_code} показан код ошибки в формате страничного исключения.

\begin{figure}[ht!]
  \centering
  \includegraphics[width=\textwidth]{inc/dia/page-fault-error-code}
  \caption{Код ошибки при страничном исключении}
  \label{fig:page_fault_error_code}
\end{figure}

Код ошибки в данном формате содержит следующую информацию:
\begin{enumerate}[1.]
\item Бит <<P>> -- бит 0. Если этот бит сбошен в 0, значит страничное исключение было
вызвано отсутсвием страницы. Иначе -- оно было вызвано нарушением одной из
страничных проверок.
\item Бит <<R/W>> -- бит 1. Если этот бит сброшен в 0 -- исключение было сгенерировано
при операции чтения. Иначе -- исключение было сгенерировано при операции записи.
\item Бит <<U/S>> -- бит 2. Если этот бит сброшен в 0 -- ошибка была вызвана при
доступе в режиме супервизора (CPL=0,1,2). В противном случае -- ошибка была
вызвана в режиме пользователя (CPL=3).
\item Бит <<RSV>> -- бит 3. Если этот бит установлен в 1 -- ошибка была вызвана в
результате прочтения процессором единицы (1) из зарезервированного поля в
элементе таблицы страничного преобразования.
\item Бит <<I/D>> -- бит 3. Если этот бит установлен в 1 -- ошибка была сгенерирована
при попытке извлечения инструкции. В противном случае этот бит
сброшен в 0. Этот бит определен только если включена функция запрета
исполнения (EFER.NXE=1 и CR4.PAE=1).
\end{enumerate}

\subsubsection*{Приоритеты}
Для возможности последовательной обработки одновременных прерываний,
прерываниям назначаются приоритеты. AMD64 разделяет прерывания на группы по
приоритетам, приоритеты внутри группы зависят от реализации.

При одновременном возникновении нескольких прерываний, процессор передает
управление обработчику прерывания с самым высоким приоритетом.
Низкоприоритетные внешние прерывания будут обработаны в порядке их приоритета,
после обработки прерываний с более высоким приоритетом. Низкоприоритетные
внутренние прерывания (исключения) отбрасываются. Они возникнут повторно,
когда обработчик прерывания завершится и вернет управление на прерванную
инструкцию. Программные прерывания так же отбрасываются, они будут повторно
возбуждены при перезапуске инструкции программного прерывания~\cite[стр. 232]{amd_pm_v2}.

\subsection{Обработа прерываний в длинном режиме}
\subsubsection*{Шлюзы прерываний и шлюзы ловушек}
В длинном режиме передача управления обработчику прерывания/исключения
осуществляется через дескрипторы шлюзов. В данном режиме IDT состоит из 256
16-байтных дескрипторов. Дескрипторы шлюзов делятся на 2 типа: дескрипторы шлюзов
прерываний (англ. Interrupt Gates) и дескрипторы шлюзы ловушек (англ. Trap Gates).

Отличие между ними в том, что при переходе через шлюз прерывания процессор
автоматически запрещает прерывания (сбрасывает RFLAGS.IF в 0).

\subsubsection*{Обработчики прерываний}
При возникновении прерывания процессор умножает номер прерывания на 16 и
использует результат в качестве смещения в IDT. Найденный дескриптор шлюза
содержит селектор сегмента и 64-битное смещение (виртуальный адрес обработчика
прерывания). Селектор сегмента указывает на дескриптор сегмента кода
расположенный в GDT или LDT. Дескриптор сегмента кода используется только для
проверки доступа и перевода процессора в длинный режим.

На рис.~\ref{fig:locating_interrupt_handler} показано как происходит поиск
обработчика прерывания в длинном режиме.

\begin{figure}[ht!]
  \centering
  \includegraphics[width=\textwidth]{inc/dia/locating-interrupt-handler}
  \caption{Определение положения обработчика прерывания в длинном режиме}
  \label{fig:locating_interrupt_handler}
\end{figure}


\subsubsection*{Стек обработчика прерывания}
В длинном режиме указатель стека прерванной программы (SS:RSP) всегда заносится в
стек обработчика прерывания, независимо от того был ли изменен уровень
привелегий. Регистр SS не используется в 64-битном режиме, SS заносится для
обеспечения возможности возврата из обработчика прерывания в режиме
совместимости.

В длинном режиме при вызове обработчика прерывания процессор выполняет
следующие действия~\cite{amd_pm_v2}:
\begin{enumerate}[1.]
\item Выравнивает стек обработчика прерывания (выполняет побитовое <<И>> RSP с маской 0xFFFF\_FFFF\_FFFF\_FFF0)
\item Если поле IST в дескрипторе шлюза прерывания не равно 0, заносит указатель стека из IST в RSP.
\item Если необходимо изменить уровень привилегий, уровень привелегий требуемого
	дескриптора (DPL) используется в качестве индекса в TSS для выбора
	нового указателя стека (если не используется IST), в регистр SS заносится 0.
\item Заносит в новый стек предыдущие значения SS:RSP. Значение SS дополняется
	6 байтами, чтобы быть выровненным по 8-байтной границе.
\item Заносит 64-битное значение RFLAGS в новый стек. Старшие 32 бита имеют
	значение 0.
\item Сбрасывает флаги TF, NT, RF в регистре флагов RFLAGS в 0.
\item Обрабатывает значение RFLAGS.IF в зависимости от типа шлюза прерывания:
	\begin{itemize}
	\item Если дескриптор шлюза, описывает шлюз прерывания -- RFLAGS.IF
		сбрасывается в 0.
	\item В противном случае - RFLAGS.IF остается без изменений
	\end{itemize}
\item Заносит в стек CS:RIP прерванной программы. Значение CS дополняется 6
	байтами, чтобы быть выровненным по 8-байтной границе.
\item Если данное прерывание возвращает код ошибки -- процессор заносит этот
	код в стек. Код ошибки дополняется 4 байтами, чтобы быть выровненным
	по 8 байтной границе.
\item Загружает селектор сегмента из дескриптора шлюза в регистр CS. Процессор
	проверяет что целевой сегмент кода является 64-битным сегментом кода.
\item Загружает смещение из дескриптора шлюза в RIP.
\end{enumerate}

На рис.~\ref{fig:interrupt_handler_stack} показано состояние стека после
передачи управления обработчику прерывания.

\begin{figure}[ht!]
  \centering
  \includegraphics[width=\textwidth]{inc/dia/interrupt-handler-stack}
  \caption{Стек обработчика прерывания без смены уровня привилегий}
  \label{fig:interrupt_handler_stack}
\end{figure}

В длинном режиме при переключении стека из-за смены уровня привилегий,
новое значение для SS выбирается не из TSS (как в защищенном режиме). В
длинном режиме из TSS берется только значение RSP, а в SS заносится ноль,
позволяя тем самым обрабатывать вложенные прерывания. В SS.RPL заносится
значение текущего уровня привилегий (CPL).

Стек обработчика прерывания при смене уровня привилегий, выглядит так же, как
стек обработчика прерывания без смены уровня привилегий, меняется только
регистр SS (в него заносится 0).

\subsubsection*{Таблица стеков обработчиков прерываний}
В длинном режиме введен новый механизм переключения стеков -- IST, который
можно использовать в качестве альтернативы описанному выше механизму.

При использование данного механизма указатель стека переключается всегда. Данный механизм
можно использовать отдельно для разных векторов прерываний, испльзуя поле IST
в дескрипторах шлюзов (в IDT).

На рис.~\ref{fig:interrupt_handler_stack_ist} показано как используется данный
механизм. Если при возникновении прерывания поле IST не равно 0, процессор
использует значение IST в качестве индекса в TSS, загружая таким образом указатель стека
(RSP) для обработчика прерывания. Если изменяется текущий уровень привилегий
-- в регистр SS загружается 0 и в SS.RPL заносится значение нового уровня
привилегий. После того как стек загружен, процессор сохраняет в него предыдущий
указатель стека, флаги процессора (RFLAGS), адрес возврата и код ошибки (если
необходим). После этого управление передается обработчику прерывания. Если два
обработчика прерывания используют один и тот же стек и второе прерывание
возникнет во время обработки первого -- оно затрет стек обработчика первого прерывания.

\begin{figure}[ht!]
  \centering
  \includegraphics[width=\textwidth]{inc/dia/interrupt-handler-stack-ist}
  \caption{Переключение стека с использованием IST}
  \label{fig:interrupt_handler_stack_ist}
\end{figure}

\subsubsection*{Возврат из обработчика прерывания}
Для возврата в прерванную программу используется команда IRET. Механизм
обработки прерываний всегда использует 64-битный стек при сохранении значений
для обработчика прерываний, обработчик прерываний всегда выполняется в
64-битном режиме.

Если при вызове обработчика прерывания в длинном режиме происходит изменение
уровня привилегий -- в SS будет загружен нулевой селектор. Если обработчик
прерывания будет прерван -- нулевой селектор будет сохранен в стеке, а в SS
будет повторно загружен другой нулевой селектор. Использование нулевого
селектора таким способом позволяет процессору правильно обрабатывать вложенные
вызовы обработчиков прерываний.

В длинном режиме, нулевой селектор в SS говорит о наличии вложенных обработчиков прерываний, поэтому
в длинном режиме инструкция IRET при определенных условиях (обработчик находится на 0, 1 или 2 уровне
привилегий в 64-битном режиме)  может выталкивать из стека нулевой селектор в SS и не вызывать исключение \#GP.

\section{Инициализация процессора и переход в длинный режим}
\label{sec:long_mode_activation}

\subsection{Инициализация реального режима}
Перед переходом в защищенный режим необходимо настроить базое окружение
реального режима. Это окружение включает:
\begin{itemize}
\item IDT реального режима, содержащую адреса точек входа в обработчики прерываний
	и исключений реального режима. После инициализации процессора базовый адрес IDT
	равен 0. При необходимости системное ПО может изменить его, загрузив новый адрес
	в регистр IDTR.
\item Обработчики прерываний и исключений реального режима. Должны быть загружены до
	включения внешних прерываний.
\item Указатель стека (SS:SP). Можно использовать значения SS:SP проинициализированные процессором.
\item Минимум один селектор сегмента данных для хранения структур данных защищенного режима,
	созданных в реальном режиме.
\end{itemize}

После того как выполнена инициализация реального режима, ПО может переходить к
инициализации защищенного режима.

\subsection{Инициализация защищенного режима}
Перед активацией длинного режима необходимо перейти в защищенный режим.
Рабочее окружение защищенного режима включает:
\begin{itemize}
\item IDT защищенного режима.
\item Обработчики прерываний и исключений, на которые ссылается IDT.
\item GDT, которая содержит:
	\begin{itemize}
	\item Дескриптор сегмента кода для защищенного режима.
	\item Доступный для чтения/записи сегмент данных, который можно использовать как стек защищенного режима.
	\end{itemize}
\end{itemize}

Системное ПО при необходимости может загрузить GDT, содержащую несколько дескрипторов сегментов данных,
дескрипторы TSS и LDT для ПО, которое выполняет инициализацию длинного режима.

После того, как структуры данных защищенного режима были проинициализированы, системному ПО необходимо
загрузить в регистры IDTR и GDTR (а также, при необходимости -- в LDTR и TR) указатели на эти структуры данных.
После того, как эти регистры загружены, переход в защищенный режим можно выполнить установив CR0.PE в 1.

Если при инициализации длинного режима используется унаследованный механизм страничного преобразования --
сначала необходимо проинициализировать таблицы страниц. Необходим как минимум один каталог страниц и одна
таблица страниц. Кроме того, необходимо загрузить в регистр CR3 физический адрес таблицы страниц верхнего
уровня. После завершения инициализации этих структур данных и перехода в защищенный режим, можно
включить страничное преобразование, установив CR0.PG в 1.

\subsection{Инициализация длинного режима}
В защищенном режиме системное ПО может подготовить структуры данных, необходимые для перехода в
длинный режим и сохранить их в произвольном месте в пределах первых 4х гигабайт физической памяти.
Эти структуры данных можно будет переместить за пределы 4х гигабайт после перехода в длинный режим.
Для перехода в длинный режим необходимы следующие структуры данных:
\begin{itemize}
\item IDT, содержащая 64-битные дескрипторы шлюзы прерываний.
\item 64-битные обработчики прерываний и исключений.
\item GDT, содержащая дескрипторы сегментов для ПО работающего в 64-битном режиме и в
	режиме совместимости, включающая:
	\begin{itemize}
	\item Десприпторы LDT, необходимые ОС и/или прикладному ПО.
	\item Дескрипторы TSS (минимум один).
	\item Дескрипторы сегментов кода для переключения между 64-битным режимом и режимом совместимости.
	\item Дескрипторы сегментов данных для ПО, работающего в режиме совместимости.
	\end{itemize}

	Существующая GDT защищенного режима может быть использована для хранения перечисленных
	выше дескрипторов.
\item 64-битный TSS для хранения указателей стека для 0, 1 и 2го уровней привилегий и/или указателей IST.
\item 4 уровня таблиц страниц. Для перехода в длинный режим, также необходимо активировать PAE.
\end{itemize}

Если страничное преобразование было активировано в процессе инициализации, его необходимо деактивировать перед
активацией длинного режима. После подготовки структур данных длинного режима и деактивации страничного преобразования,
можно выполнить активацию и переход в длинный режим.

\subsection{Активация и переход в длинный режим}
Для активации длинного режима необходимо установить бит EFER.LME в 1. Однако переход в длинный режим
не будет выполнен до активациии страничного преобразования. После того, как системное ПО
активирует страничное преобразование, при активированном длинном режиме, процессор перейдет в
длинный режим, установив при этом бит EFER.LMA в 1.

Для выбора подрежима работы процессора в длинном режиме используются 2 бита дескриптора сегмента кода (CS.L и CS.D).
Комбинация CS.L=1, CS.D=0 -- переводит процессор в 64-битный режим. Комбинация CS.L=0, CS.D=0 -- в
режим совместимости. Комбинация CS.L=1, CS.D=1 зарезервирована для будущего использования.

\subsubsection*{Переход в длинный режим}
Переход в длинный режим выполняется в несколько этапов: деактивация страничного преобразования (CR0.PG=0),
активация механизма расширения физических адресов (CR4.PAE=1), загрузка регистра CR3,
активация длинного режима (EFER.LME=1), активация страничного преобразования (CR0.PG=1).

Таким образом, для перехода в защищенный режим, системное ПО должно выполнить следующие действия:
\begin{enumerate}[1.]
\item Если используется страничное преобразование, его необходимо деактивировать, сбросив бит CR0.PG в 0.
	Для этого необходимо, чтобы инструкция MOV CR0, используемая для деактивации страничного преобразования
	располагалась на странице, виртуальный адрес которой совпадает с физическим.
\item В любой последовательности:
	\begin{itemize}
	\item Активировать механизм расширения физических адресов, установив бит CR4.PAE в 1. Это
		необходимо сделать до активации страничного преобразования.
	\item Загрузить в регистр CR3 физический адрес PML4.
	\item Активировать длинный режим, установив бит EFER.LME в 1.
	\end{itemize}
\item Активировать страничное преобразование, установив бит CR0.PG в 1. В результате процессор установит бит EFER.LMA в 1.
\end{enumerate}

\subsubsection*{Обновление ссылок на таблицы системных дескрипторов}
После перехода в длинный режим регистры таблиц дескрипторов (GDTR, LDTR, IDTR, TR) ссылаются на таблицы унаследованного
режима, которые расположены в пределах первых 4-х гагабайт виртуального адресного пространства. Системному ПО необходимо
обновить эти регистры, чтобы они указывали на 64-битные версии таблиц дескрипторов, используя команды LGDT, LIDT, LLDT, LTR.

В длинном режиме необходимо использовать 64-битные дескрипторы шлюзов прерываний. Прерывания должны быть
отключены пока IDTR не будет обновлена. Для отключения внешних прерываний можно воспользоваться инструкцией CLI.

\subsubsection*{Перемещение таблиц страничного преобразования}
После перехода в длинный режим таблицы страниц расположены в пределах первых 4-х гигабайт физического
адресного пространства, т.к. инструкция MOV CR3 выполняется в защищенном режиме (в регистр CR3 записываются
только младшие 32 бита физического адреса). После перехода в длинный режим
системное ПО может переместить таблицы страниц в произвольное место физического адресного пространства,
обновив значение регистра CR3.

\section{APIC и IO APIC}
APIC осуществляет поддержку прерываний в архитектуре AMD64. Локальный APIC принимает прерывания и
доставляет их процессору. На рис.~\ref{fig:apic-schema} показана схема реализации APIC.

\begin{figure}[ht!]
  \centering
  \includegraphics[width=0.9\textwidth]{inc/dia/apic-schema}
  \caption{Схема реализации APIC}
  \label{fig:apic-schema}
\end{figure}

\subsection{Источники прерываний локального APIC}
С каждым ядром ЦП связан локальный APIC, который принимает прерывания от следующих источников:
\begin{itemize}
	\item Внешние прерывания от IOAPIC (включая LINT0 и LINT1)
	\item Унаследованные прерывания (INTR и NMI) от PIC
	\item Межпроцессорные прерывания (IPI) от других локальных APIC. Используются
		для отправки прерываний различным ядрам ЦП.
	\item Локальные прерывания. Локальный APIC получает прерывания от таймера,
		счетчиков производительности, термальных сенсоров, а также при возникновении
		ошибок APIC.
\end{itemize}

\subsection{Локальный APIC}
Локальный APIC контролируется битом <<AE>> регистра базового адреса APIC (APIC Base Address Register).
Это моделезависимый регистр, который имеет номер 0000\_001Bh (для доступа к регистру необходимо
использовать команды RDMSR и WRMSR, с номером нужного регистра в регистре ECX).
На рис.~\ref{fig:apic-base-address-register} показан формат регистра базового адреса APIC.

\begin{figure}[ht!]
  \centering
  \includegraphics[width=1.0\textwidth]{inc/dia/apic-base-address-register}
  \caption{Формат регистра базового адреса APIC}
  \label{fig:apic-base-address-register}
\end{figure}

Регистр базового адреса APIC включает следующие поля:
\begin{itemize}
\item Бит <<BSC>>. Бит 8. Показывает является ли текущее ядро загрузочным ядром загрузочного процессора.
\item Бит <<AE>>. Бит 11. Данный бит активирует APIC.
\item Базовый адрес APIC. Биты 51:12. Задает базовый физический адрес регистров APIC (по умолчанию
	равен s 0\_0000\_FEE0\_0000h).
\end{itemize}

\subsection{Регистры APIC}
Настройка APIC производится с использованием отображенных на память регистров APIC.
Адреса регистров вычисляются путем сложения базового адреса APIC и смещения регистра.
Базовый адрес APIC можно задавать через моделезависимый регистр базового адреса APIC.

Регистры APIC выровнены по 16-байтной границе. Доступ к ним должен осуществляться
по адресам, выровненным по 4-байтной границе.

Подробное описание и назначение регистров APIC можно найти в ~\cite{amd_pm_v2}.

Каждое ядро процессора в системе имеет уникальный идентификатор локального APIC (APIC ID).
Его значение определяется аппаратным обеспечением в зависимости от идентификатора процессора
и количества ядер. Как показано на рис.~\ref{fig:local-apic-id}, идентификатор локального APIC
находится в регистре APIC ID, в битах 31:24.

\begin{figure}[ht!]
  \centering
  \includegraphics[width=1.0\textwidth]{inc/dia/local-apic-id}
  \caption{Формат регистра APIC ID (смещение 20h)}
  \label{fig:local-apic-id}
\end{figure}

\subsection{Таймер APIC}
Таймер APIC это программируемый 32-битный счетчик, используемый ПО, для генерации событий.
Таймер может работать в двух режимах -- разовый и периодичный, в зависимости от значения бита <<TM>> (17)
в регистре LVT (показан на рис.~\ref{fig:apic-timer-lvt-register}). Когда значение счетчика таймера
достигает нуля -- генерируется прерывание (если бит <<M>> в регистре LVT сброшен). В случае использования
таймера в периодичном режиме, после генерации прерывания, счетчик таймера повторно инициализируется
значением из регистра, содержащего начальное значение счетчика и отсчет начинается сначала.

\begin{figure}[ht!]
  \centering
  \includegraphics[width=1.0\textwidth]{inc/dia/apic-timer-lvt-register}
  \caption{Формат регистра APIC Timer LVT (смещение 320h)}
  \label{fig:apic-timer-lvt-register}
\end{figure}

Для управления таймером определены 3 регистра: CCR (англ. Current Count Register),
ICR (англ. Initial Count Register) и DCR (англ. Divide Configuration Register).

Регистр ICR задает начальное значение счетчика таймера (рис.~\ref{fig:apic-timer-icr-register}).
\begin{figure}[ht!]
  \centering
  \includegraphics[width=1.0\textwidth]{inc/dia/apic-timer-icr-register}
  \caption{Формат регистра APIC Timer ICR (смещение 380h)}
  \label{fig:apic-timer-icr-register}
\end{figure}

Регистр CCR (рис.~\ref{fig:apic-timer-ccr-register}) инициализируется значением регистра ICR и
уменьшается на каждый тик таймера на значение, зависящее от регистра DCR. Когда значение
данного регистра достигает нуля, генерируется прерывание.
\begin{figure}[ht!]
  \centering
  \includegraphics[width=1.0\textwidth]{inc/dia/apic-timer-ccr-register}
  \caption{Формат регистра APIC Timer CCR (смещение 390h)}
  \label{fig:apic-timer-ccr-register}
\end{figure}

Регистр DCR (рис.~\ref{fig:apic-timer-dcr-register}) задает значение, которое вычитается из ICR
на каждом тике таймера. Значение определяется в соответствии с таблицей~\ref{tab:apic_timer_dcr_values}.
\begin{figure}[ht!]
  \centering
  \includegraphics[width=1.0\textwidth]{inc/dia/apic-timer-dcr-register}
  \caption{Формат регистра APIC Timer DCR (смещение 3E0h)}
  \label{fig:apic-timer-dcr-register}
\end{figure}

\begin{table}[ht!]
  \caption{Значения делителя}
  \label{tab:apic_timer_dcr_values}
  \begin{tabular}{|c|c|}
    \hline
    Биты 3, 1:0 регистра DCR & Значение делителя \\
    \hline
    000 & 2 \\
    \hline
    001 & 4 \\
    \hline
    010 & 8 \\
    \hline
    011 & 16 \\
    \hline
    100 & 32 \\
    \hline
    101 & 64 \\
    \hline
    110 & 128 \\
    \hline
    111 & 1 \\
    \hline
  \end{tabular}
\end{table}

\subsection{IOAPIC}
IOAPIC (I/O Advanced Programmable Interrupt Controller) используется в качестве замены PIC для
управления внешними прерываниями. Каждый IOAPIC может обрабатывать до 24х типов прерываний.

Настройка IOAPIC производится с использованием трех отображенных на память регистров:
IOAPICBASE, IOREGSEL и IOWIN. Для доступа к внутренним регистрам IOAPIC используется косвенная
адресация через регистры IOREGSEL и IOWIN, расположение которых определяется значением
регистра IOAPICBASE. Доступ ко всем регистрам осуществляется с использованием 32-битных
операций чтения и записи, т.е. для модификация одного поля (бит, байт и т.д.) необходимо
прочитать 32 бита, изменить нужное поле и записать обратно.

Подробное описание всех регистров IOAPIC приведено в ~\cite{ioapic}.

\subsubsection*{Отображенные на память регистры IOAPIC}
\paragraph{IOREGSEL.} Регистр IOREGSEL используется для выбора внутреннего регистра IOAPIC для чтения/записи.
После выбора требуемого регистра, данные можно прочитать/записать используя регистр IOWIN. Формат регистра
IOREGSEL показан на рис.~\ref{fig:ioapic-ioregsel}. По умолчанию данный регистр расположен по адресу FEC0\_0000h.
\begin{figure}[ht!]
  \centering
  \includegraphics[width=1.0\textwidth]{inc/dia/ioapic-ioregsel}
  \caption{Формат регистра IOREGSEL}
  \label{fig:ioapic-ioregsel}
\end{figure}

\paragraph{IOWIN.} Данный регистр используется для чтения/записи во внутренний регистр IOAPIC, который был
выбран с использованием регистра IOREGSEL. Формат данного регистра показан на рис.~\ref{fig:ioapic-iowin}.
По умолчанию данный регистр расположен по адресу FEC0\_0010h.
\begin{figure}[ht!]
  \centering
  \includegraphics[width=1.0\textwidth]{inc/dia/ioapic-iowin}
  \caption{Формат регистра IOWIN}
  \label{fig:ioapic-iowin}
\end{figure}

\subsubsection*{Внутренние регистры IOAPIC}
\paragraph{IOAPICID.} Как показано на рис.~\ref{fig:ioapic-id}, данный регистр содержит идентификатор IOAPIC (биты 27:24),
который используется в качестве физического имени IOAPIC. В данный регистр необходимо записать корректный идентификатор
IOAPIC перед использование других регистров IOAPIC. Данный регистр имеет номер 0 (для доступа через IOREGSEL).
\begin{figure}[ht!]
  \centering
  \includegraphics[width=1.0\textwidth]{inc/dia/ioapic-id}
  \caption{Формат регистра IOAPICID}
  \label{fig:ioapic-id}
\end{figure}

\paragraph{IOREDTBL[23:0].} Включает 24 64-битных элемента, для 24х различных сигналов. В отличие от PIC -- приоритеты
прерываний не зависят от физического номера сигнала (они определяются номером обработчика прерывания). Для каждого
сигнала ОС может задать полятность (высокая или низкая), режим возникновения прерывания (по фронту или по спаду) и некоторые
другие свойства. Информация из регистра IOREDTBL используется для преобразования номера физического сигнала в сообщение для
APIC. На рис.~\ref{fig:ioapic-ioredtbl} показан формат элемента IOREDTBL. Так как элементы имеют размер 64-бита, каждый
из них описываеся двумя регистрами APIC, например для доступа к первому элементу используются регистры с номерами 10h и 11h,
ко второму -- 12h и 13h и т.д.

\begin{figure}[ht!]
  \centering
  \includegraphics[width=1.0\textwidth]{inc/dia/ioapic-ioredtbl}
  \caption{Формат регистров IOREDTBL[23:0]}
  \label{fig:ioapic-ioredtbl}
\end{figure}

Каждый элемент IOREDTBL включает следующие поля:
\begin{itemize}
\item ID получателя. Биты 63:56. Содержит либо APIC ID, либо маску процессоров (зависит от бита DM).
\item Бит <<IM>> (16). Если установлен в 1 -- прерывания замаскированы.
\item Бит <<TM>> (15). Определяет режим срабатывания прерываний (1 -- по уровню сигнала, 0 -- по спаду).
\item Бит <<IRR>> (14). Используется для прерываний, чувствительных к уровню сигнала. Устанавливается в 1,
	когда локальный APIC получил прерывание. Сбрасывается в 0, когда локальный APIC прислал EOI (сигнал
	о завершении прерывания).
\item Бит <<POL>> (13). Определяет полярность сигнала прерывания (0 -- высокая, 1 -- низкая).
\item Бит <<DS>> (12). Содержит текущий статус доставки прерывания. Если значение равно 0 -- никаких действий,
	связанных с данным прерыванием не выполняется. Иначе -- прерывание отправлено, но ещё не доставлено.
\item Бит <<DM>> (11). Данный бит определяет как следует интерпретировать поле <<ID получателя>>.
	Если бит сброшен в 0 (физический режим), биты 59:56 содержат APIC ID получателя. В
	противном случае (логический режим), биты 63:56 содержат маску получателей.
\item Режим доставки. Биты 10:8. Сущесвтует несколько режимо доставки, в данной работе используется
	фиксированный режим (значение 0) -- доставляет указанное прерывание всем ядрам процессоров,
	перечисленным в поле <<ID получателя>>. Подробное описание всех режим приведено в~\cite{ioapic}.
\item Номер вектора. Биты 7:0. Содержит номер вектора прерывания, который необходимо вызвать.
	Допустимые значения лежат в диапазоне от 10h до FEh.
\end{itemize}
