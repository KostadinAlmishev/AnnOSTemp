\Conclusion

В результате работы по созданию ядра операционной системы была реализована его
основная функция как средства разделения аппаратных ресурсов: физической памяти,
процессорного времени и устройства вывода информации. В итоге была создана
многозадачная однопользовательская однопроцессорная система с монолитным ядром.

Недостатками полученной ОС являются отсутствие файловой системы и поддержки
многопроцессорной архитектуры. Тем не менее, созданная система поддерживает
современную архитектуры процессоров AMD64, реализует
полноценную изоляцию процессов, виртуальную память на основе страниц, вытесняющую
многозадачность, потоки уровня ядра и эффективное клонирование процессов на
основе копирования памяти при ее изменении.

Разработанную операционную систему можно улучшить следующим образом: добавить возможность
работы с несколькими процессорами~\cite{mp} и средства синхронизации,
добавить файловую систему и межпроцессное взаимодействие (англ. IPC), добавить
поддержку дополнительного аппаратного обеспечения, например -- сетевой карты.
