\Defines
\begin{description}
\item[16-битный режим] унаследованный режим или режим совместимости, в котором размер адреса по умолчанию составляет 16~бит.
\item[32-битный режим] унаследованный режим или режим совместимости, в котором размер адреса по умолчанию составляет 32~бита.
\item[64-битный режим] подрежим длинного режима. В данном режиме размер адреса по умолчанию составляет 64~бита и доступны
новые возможности, такие как расширенный набор регистров.
\item[Байт] 8 бит.
\item[Режим совместимости] подрежим длинного режима. В данном режиме размер адреса по умолчанию составляет 32 бита, он
позволяет запускать существующее 16 и 32-битное прикладное ПО без перекомпиляции.
\item[Унаследованный режим] режим работы процессора в котором существующее 16 и 32-битное ПО можно запустить без модификаций.
Включает 3 подрежима: реальный режим, защищенный режим, режим виртуального 8086.
\item[Длинный режим] режим работы процессора уникальный для архитектуры AMD64. Имеет 2 подрежима: 64-битный режим и
режим совместимости.
\item[Системное ПО] привилегированное ПО, которое управляет аппаратными ресурсами системы и контролирует доступ к этим ресурсам.
\item[Логический адрес] состоит из селектора сегмента и смещения (эффективного адреса).
\item[Эффективный адрес] смещение в сегменте.
масштабирующего коэффициента
\item[Линейный (виртуальный) адрес] равен сумме эффективного адреса и базового адреса сегмента.
\item[Физический адрес] адрес в физическом адресном пространстве.
\item[Каноническая форма адреса] форма адреса, в которой все биты начиная с наиболее значимого (в данной реализации) и до 63го совпадают.
\end{description}
