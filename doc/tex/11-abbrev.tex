\Abbreviations
\begin{description}
\item[ПО] программное обеспечение.
\item[SMM] System Management Mode, режим работы процессора, используемый, как правило
для настройки аппаратного обеспечения.
\item[SMI] System Management Interrupt, прерывание, при возникновении которого, процессор
переходит в SMM.
\item[CPL] текущий уровень привилегий процессора.
\item[DPL] уровень привилегий дескриптора.
\item[GDT] глобальная таблица дескрипторов.
\item[IDT] таблица дескрипторов прерываний.
\item[IST] таблица указателей стека в длинном режиме.
\item[LDT] локальная таблица дескрипторов.
\item[TSS] сегмент состояния задачи.
\item[PAE] механизм расширения физических адресов, позволяет использовать физические адреса длинной
до 52 бит.
\item[EFER] моделезависимый регистр включения расширенных возможностей.
\item[FLAGS] 16-битный регистр флагов.
\item[eFLAGS] 16 или 32-битный регистр флагов.
\item[EFLAGS] 32-битный регистр флагов.
\item[rFLAGS] 16, 32 или 64-битный регистр флагов.
\item[RFLAGS] 64-битный регистр флагов.
\item[IP] 16-битный счетчик команд.
\item[eIP] 16 или 32-битный счетчик команд.
\item[EIP] 32-битный счетчик команд.
\item[rIP] 16, 32 или 64-битный счетчик команд.
\item[RIP] 64-битный счетчик команд.
\item[CS] регистр сегмента кода.
\item[CR0,CR2-CR4,CR8] управляющие регистры процессора.
\item[MSR] моделезависимый регистр.
\end{description}


