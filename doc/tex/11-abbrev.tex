\Abbreviations
\begin{description}
\item[ОС] операционная система.
\item[ЦП] центральный процессор.
\item[ПО] программное обеспечение.
\item[x86] совокупное название архитектур компьютеров с 32-битными
	процессорами Intel i386, i486 и более старшими и совместимыми с ними,
	а также с работающими в 32-битном режиме совместимыми 64-битными процессорами.
\item[BIOS] встроенное в ПЗУ программное обеспечение для инициализации и доступа к
	аппаратуре компьютера архитектуры x86 (англ. Basic Input-Output System).
\item[PML4] таблица страниц верхнего уровня в длинном режиме (англ. Page Map Level 4).
\item[PML4E] элемент таблицы страниц 4го уровня (англ. Page Map Level 4 Entry).
\item[PDP] таблица указателей на директории страниц (англ. Page Directory Pointer).
\item[PDPE] элемент таблицы указателей на директории страниц (англ. Page Directory Pointer Entry).
\item[PDE] элемент директории страниц (англ. Page Directory Entry).
\item[PTE] элемент таблицы страниц (англ. Page Table Entry).
\item[CPL] текущий уровень привилегий процессора (англ. Current Privilege Level).
\item[DPL] уровень привилегий дескриптора (англ. Descriptor Privilege Level).
\item[RPL] уровень привилегий процесса, создавшего селектор (англ. Requestor Privilege Level).
\item[GDT] глобальная таблица дескрипторов (англ. Global Descriptor Table).
\item[LDT] локальная таблица дескрипторов (англ. Local Descriptor Table).
\item[IDT] таблица дескрипторов обработчиков прерываний (англ. Interrupt Descriptor Table).
\item[IST] таблица указателей стека в длинном режиме (англ. Interrupt Stack Table).
\item[LDT] локальная таблица дескрипторов (англ. Local Descriptor Table).
\item[TSS] сегмент состояния задачи (англ. Task State Segment).
\item[PAE] механизм расширения физических адресов, позволяет использовать физические адреса длинной
до 52 бит (англ. Physical Address Extension).
\item[EFER] моделезависимый регистр включения расширенных возможностей (англ. Extended Feature Enable Register).
\item[rFLAGS] 16, 32 или 64-битный регистр флагов.
\item[RFLAGS] 64-битный регистр флагов.
\item[rIP] 16, 32 или 64-битный счетчик команд.
\item[RIP] 64-битный счетчик команд.
\item[CS] регистр сегмента кода.
\item[CR0,CR2-CR4,CR8] управляющие регистры процессора.
\item[MSR] моделезависимый регистр (англ. Model Specific Register).
\item[EOI] сигнал о завершении обработки прерывания, посылаемый
	контроллеру прерываний (англ. End Of Interupt).
\item[APIC] улучшенный программируемый контроллер прерываний (англ. Advanced Programmable Interrupt Controller).
\item[IOAPIC] контроллер, расположенный на системной плате,
	используемый для управления внешними прерываниями (англ. I/O Advanced Programmable Interrupt Controller).
\end{description}
