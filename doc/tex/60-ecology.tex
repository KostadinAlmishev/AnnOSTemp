\chapter{Безопасность жизнедеятельности и охрана труда}
\label{cha:ecolog}

\section{Анализ опасных и вредных факторов при разработке программного
	обеспечения и мероприятия по их устранению}
Процесс создания программного обеспечения невозможен без использования
персональных электронно-вычислительных машин (ПЭВМ). Эксплуатация ПЭВМ
сопряжена с рядом вредных факторов, воздействию котрых подвергаются
пользователи. К таким факторам относятся электромагнитное излучение,
отраженный свет и блики, вибрация, шум.

В данном разделе рассматриваются основные виды опасных и вредных факторов, а
также допустимые нормы излучений, требования к освещенности, помещению в
целом, уровням шума на рабочем месте, которые регламентируются СанПиН
2.2.2/2.4.1340-03 <<Гигиенические требования к персональным
электронно-вычислительным машинам и организация рабочего места>>

\subsection{Требования к помещению для работы с ПЭВМ}
Обучение и работа с ПЭВМ должны проходить в помещении, которое не наносит
вреда здоровью и создает комфортные условия труда. Следует придерживаться
общих рекомендаций при выборе помещения и материалов для его внутренней
отделки:
\begin{enumerate}
\item Окна в помещениях, где используется вычислительная техника,
	преимущественно должны быть ориентированы на север и северо-восток.
	Оконные проемы должны быть оборудованы регулируемыми устройствами
	типа: жалюзи, занавесей, внешних козырьков и т.д.

\item Для внутренней отделки интерьера помещений, где расположены ПЭВМ, должны
	использоваться диффузно отражающие материалы с коэффициентом отражения
	для потолка: $0.7 - 0.8$; для стен: $0.5 - 0.6$; для потолка: $0.3 -
	0.5$.
\item Помещения, где размещаются рабочие места с ПЭВМ, должны быть оборудованы
	защитным заземлением в соответствии с техническими требования по
	эксплуатации.
\item Не следует размещать рабочие места с ПЭВМ вблизи силовых кабелей и
	вводов, высоковольтных транформаторов, технологического оборудования,
	создающего помехи в работе ПЭВМ.
\end{enumerate}

\subsection{Общие требования к организации рабочих мест пользователей ПЭВМ}
Пользователи ПЭВМ в первую очередь подвержены воздействию излучения от экрана
мониторов, поэтому экран видеомонитора должен находиться на расстоянии $600 -
700$~мм от глаз пользователя, но не ближе 500~мм с учетом размеров
алфавитно-цифровых знаков и символов.

Для комфортной организации рабочих мест расстояние между рабочими столами с
видеомониторами должно быть не менее 2.0~м, а расстояние между боковыми
поверхностями видеомониторов -- не менее 1.2~м. СанПиН 2.2.2/2.4.1340-03
рекомендует изолировать рабочие места друг от друга перегородками высотой $1.5
- 2.0$~м. Для снижения утомляемости пользователей рабочее кресло должно быть
подъемно-поворотным, регулируемым по высоте и по углам наклона сиденья и
спинки, а также расстоянию спинки от переднего края сиденья.

\subsection{Требования к микроклимату}
Процесс написания/изучения работы программного продукта требует повышенной
концентрации внимания, умственных усилий, что сопровождается
неврно-эмоциональным напряжением. Данный вид работ относится к категории 1а.
Такие помещения согласно СанПиН должны обеспечиваться оптимальными параметрами
микроклимата. Оптимальные параметры микроклимата указаны в
таблице~\ref{tab:optimal_microclimat}.

\begin{center}
    \begin{longtable}{|c|p{0.20\textwidth}|p{0.35\textwidth}|p{0.25\textwidth}|}
    \caption{Оптимальные параметры микроклимата}
    \label{tab:optimal_microclimat}
    \\ \hline
    Период года & Температура воздуха, \celsius & Относительная влажность воздуха, \% &
    Скорость движения воздуха, м/c (не более) \\
    \hline \endfirsthead
    \subcaption{Продолжение таблицы~\ref{tab:optimal_microclimat}}
    \\ \hline \endhead
    \hline \subcaption{Продолжение на след. стр.}
    \endfoot
    \hline \endlastfoot
    Холодный   & 22-24 & 40-60 & 0.1 \\
    \hline
    Теплый   & 23-25 & 40-60 &  0.1 \\
    \hline
  \end{longtable}
\end{center}

К вредным факторам при работе с ЭВМ относят также и запыленность помещения.
Этот фактор усугубляется влиянием электростатических полей персональных
компьютеров. В качестве норм по уменьшению запыленности, поддержания
температуры и влажности следует использовать систему кондиционирования.
Необходимо производить влажную уборку и проветривать помещение.

\subsection{Требования к уровню шума и вибрации}
Уровень шума на рабочем месте не должен превышать 50~дБA, а уровни вибрации не
должны превышать предельно допустимых значений, указанных в
таблице~\ref{tab:vibration}.

\begin{table}[h]
  \centering
  \caption{Допустимые нормы вибрации на рабочих местах с ПЭВМ}
  \label{tab:vibration}
  \begin{tabular}{|p{0.35\textwidth}|p{0.20\textwidth}|p{0.20\textwidth}|}
    \hline
    \multirow{2}{\hsize}{Среднегеометрические частоты октавных полос, Гц} &
    \multicolumn{2}{|c|}{Допустимые значения по виброскорости} \\
    \cline{2-3}
    & м/с & дБ \\
    \hline
    2 & 45 & 79 \\
    \hline
    4 & 22 & 73 \\
    \hline
    8 & 11 & 67\\
    \hline
    16 & 11 & 67 \\
    \hline
    31.5 & 11 & 67 \\
    \hline
    63 & 11 & 67 \\
    \hline
    Корректированные значения и их уровни в дБ & 20 & 72 \\
    \hline
  \end{tabular}
\end{table}

К внутренним источникам шума относятся вентиляторы, принтеры и другие
периферийные устройства ЭВМ.

Мощные источники шума, такие как сервера должны быть расположены в отдельных
помещениях с использованием средств звукоизоляции (звукопоглощающих материалов
для облицовки стен и потолка помещения) и толстых перегородок (стен).

К внешним источникам шума можно отнести шум с улицы и соседних помещений. Для
снижения шума улицы следует использовать более толстые шумопоглощающие
стеклопакеты. Для уменьшения шума соседних комнат следует использовать облицовку
стен звукопоглощающими материалами.

\subsection{Требования к освещенности}
Наиболее важным условием эффективной работы пользователей является соблюдение
оптимальных параметров системы освещения в рабочих помещениях.

В соответствии с СанПиН 2.2.2/2.4.1340-03 освещенность на поверхности рабочего
стола должна находиться в пределах 300-500~лк. Разрешается использование
светильников местного освещения для работы с документами (при этом светильники
не должны создвать блики на поверхности экрана).

Правильное расположение рабочих мест относительно источников освещения,
отсутствие зеркальных поверхностей и использование матовых материалов
ограничивает прямую (от источников освещения) и отраженную (от рабочих
поверхностей) блескость. При этом яркость светящихся поверхностей и потолка не превышает
$200~\textup{кд}/\textup{м}^{2}$, яркость бликов на экране ПЭВМ не превышает
$40~\textup{кд}/\textup{м}^{2}$.

\subsection{Требования к уровням электромагнитных полей на рабочих местах,
оборудованных ПЭВМ}
При использовании ПЭВМ пользователи подвергаются воздействию электромагнитного
излучения. Уровни электромагнитных полей нормируются по СанПиН
2.2.2/2.4.1340-03 и представлены в таблице~\ref{tab:vdu_emp} и
таблице~\ref{tab:visual_vdt}.

\begin{table}
  \centering
  \caption{Временные допустимые уровни ЭМП, создаваемых ПЭВМ}
  \label{tab:vdu_emp}
  \begin{tabular}{|p{0.22\textwidth}|p{0.37\textwidth}|p{0.20\textwidth}|}
    \hline
    \multicolumn{2}{|c|}{Наименование параметров} & ВДУ ЭМП \\
    \hline
    \multirow{2}{\hsize}{Напряженность электрического поля}
        & в диапазоне частот 5~Гц - 2~кГц & 25~В/м\\
        \cline{2-3}
        & в диапазоне частот 2~кГц - 400~кГц & 2.5~В/м\\
    \hline
    \multirow{2}{\hsize}{Плотность магнитного потока}
        & в диапазоне частот 5~Гц - 2~кГц & 250~нТл\\
        \cline{2-3}
        & в диапазоне частот 2~кГц - 400~кГц & 25~нТл\\
    \hline
    \multicolumn{2}{|c|}{Электростатический потенциал экрана видеомонитора} & 500~В \\
    \hline
  \end{tabular}
\end{table}

\begin{table}
  \centering
  \caption{Визуальные параметры ВДТ, контролируемые на рабочих местах}
  \label{tab:visual_vdt}
  \begin{tabular}{|p{0.45\textwidth}|p{0.45\textwidth}|}
    \hline
    Параметры & Допустимы значения \\
    \hline
    Яркость белого поля & Не менее $ 35~\textup{кд}/\textup{м}^{2}$ \\
    \hline
    Неравномерность яркости рабочего поля & Не более $\pm 20\% $ \\
    \hline
    Контрастность (для монохромного режима) & Не менее 3:1\\
    \hline
    Временная нестабильность изображения (мелькания) & Не должна фиксироваться \\
    \hline
  \end{tabular}
\end{table}


\section{Рассчет вентиляции}
Согласно СП 60.13330.2012 минимальный расход воздуха в общественных зданиях составляет
$40~\textup{м}^3/\textup{час}$ (при наличии естественного проветривания) и
$60~\textup{м}^3/\textup{час}$ (при отсутствии естественного проветривания).

Рассмотрим компьютерный класс со следующими характеристиками:
\begin{enumerate}
\item Количество мест: 30
\item Длинна: 15~м
\item Ширина: 9~м
\item Высота: 4~м
\end{enumerate}

Исходя из того, что для общественно-бытовых зданий на одного человека
полагается $ 40~\textup{м}^3/\textup{час} $. Рассчитаем требуемое
количество притока воздуха: $L_{\textup{пр}} = 30 \cdot 40 = 1200~\textup{м}^3/\textup{час}$.

Согласно СНиП II-Л.6-67: для аудиторий до 150 мест, учебных кабинетов,
чертежных залов, залов курсового проектирования кратность воздухообмена в час
должна составлять не менее $20~\textup{м}^3$ наружного воздуха на 1 место, т.о.
норма вытяжки составляет: $L_{\textup{выт}} =  20 \cdot 30 = 600~\textup{м}^3/\textup{час}$.

Так как $L_{\textup{пр}} > L_{\textup{выт}}$, увеличим $L_{\textup{выт}}$ до
значения $L_{\textup{пр}}$, т.о. $L_{\textup{пр}} = L_{\textup{выт}} = 1200~\textup{м}^3/\textup{час}$.

Таким образом для обеспечения необходимого воздухообмена можно использовать
вентилятор <<Standler Form Charly Fan Table C-025>>.

\section{Утилизация пластмасс, являющихся частью жидко-кристаллического монитора}
Процесс переработки начинается с ручного демонтажа составных частей электронной техники.
Демонтированные компоненты, как правило, сортируются на пластик, металл, печатные платы,
провода, люминесцентные лампы, ЖК-дисплеи для дальнейшей переработки.
В данной главе будет рассмотрен процесс утилизации пластика.

Пластмассы — это материалы на основе природных или синтетических полимеров, способные
под воздействием нагревания или давления деформироваться в изделия сложной конфигурации
и затем устойчиво сохранять полученную ими форму. В зависимости от технологического
процесса производства, применяемого наполнителя и связующего (смолы) пластмассы могут
быть композиционными, слоистыми или литыми, а по природе применяемой
смолы — термореактивными или термопластичными.

При производстве пластмасс в процессе переработки полимерных материалов происходит
выделение газообразных продуктов (аммиак, метиловый спирт, окись углерода), органических
кислот, фенола, стирола. Для локализации выделяющихся веществ необходимо предусмотреть
местные отсосы от оборудования с подключением их к системам вытяжной вентиляции.
В процессе переработки термопластических материалов происходит накопление твердых
отходов (слитки и куски полимеров, литники, обрезки, изделия с дефектами), которые
могут быть полностью переработаны на дробильном оборудовании и вновь использованы
как вторичное сырье в виде добавок к основному производству.

Но при этом образуется почти такое же количество отходов, которые не могут быть
использованы: они вместе с бытовыми отходами отправляются на полигон твердых
бытвых отходов (ТБО). Пластмассы мало используются как вторичное сырье из-за многообразия их типов
и сложности их составов. Производство пластмасс не связано с загрязнением сточных
вод, так как по технологии должно быть обеспечено оборотное водоснабжение.

Основные направления утилизации и ликвидации отходов пластмасс таковы:
\begin{enumerate}
\item захоронение на полигонах и свалках;
\item переработка их по заводской технологии;
\item сжигание совместно с ТБО и промышленными отходами;
\item пиролиз или раздельное сжигание в специальных печах;
\item использование отходов пластмасс как готового материала в других технологических процессах.
\end{enumerate}

Наиболее оптимальным методом использования отходов пластмасс является их
переработка по заводским технологиям. Общая схема процесса переработки
включает следующие стадии:
\begin{enumerate}[1.]
\item отделение непластмассовых компонентов (ветошь, картон, остатки упаковки:
	бумажные, деревянные или металлические) и сортировка отходов по внешнему виду
\item измельчение отходов пластмассы (иногда в несколько стадий) до размеров,
	достаточных для осуществления их дальнейшей переработки
\item отмывка измельченных отходов от загрязнений органического и минерального
	характера
\item классификация и сушка отходов
\item смешивание полученных отходов (при необходимости) со стабилизаторами,
	красителями, наполнителями и гранулируют
\item переработка гранулята в изделия
\end{enumerate}

При качественной предварительной рассортировке пластмасс по видам, достижении высокой
степени очистки и выделения отдельных отходов из смесей их переработка практически не
отличается от переработки первичных пластмасс. При этом необходимо учитывать способность
полимеров сохранять или изменять свойства в процессе многократной переработки, что
вообще определяет целесообразность выполнения переработки отходов. Изменение физико-химических
свойств большинства полимеров при многократной переработке связано со снижением молекулярной
массы пластмасс, разветвленностью их структуры. Снижение молекулярной массы пластмасс
приводит к изменению их прочностных показателей.

Особенностью повторной переработки поливинилхлорида (ПВХ) является необходимость
его дополнительной стабилизации. Отходы мягкого ПВХ используются для получения
бытовых изделий, пленочных покрытий и пленок. При этом 20\% отходов измельчают
на смесительных вальцах, смешивают с товарным ПВХ, красителями, смазками и
стабилизатором, а затем пропускают через систему подогревательных и отделочных
вальцов. Из отходов полиэтилена высокого давления производят мешки для мусора,
трубы, хозяйственные ведра, уплотнительные профили и прокладки. Полипропиленовые
отходы перерабатывают в текстильные шпули, детали сантехники, дверные ручки, ящики для растений.

Выполнение утилизации смесей отходов без предварительного разделения их
составляющих делает процесс утилизации более дешевым, но физико-механические
свойства полученных при этом изделий гораздо хуже.

Таким образом процесс утилизации и переработки пластмасс является актуальной темой.
