\Introduction

Для изучения основ функционирования операционных систем недостаточно
изучения только теоретического материала. Для понимания работы ядра
ОС необходимо изучать и модифицировать его исходный код.

В настоящее время существует множество операционных систем с открытым
исходным кодом: GNU Linux, FreeBSD, ReactOS и др. Однако ядра этих
операционных систем плохо подходят для учебного процесса, т.к. они
имеют большой объем исходного кода и обладают высокой сложностью.

По этой причине были разработаны несколько учебных операционных
систем: JOS, xv6 и PhantomEx. Ядра этих ОС имеют небольшой объем
исходного кода и сравнительно небольшую сложность, по сравнению с
эксплуатируемыми ОС, что делает их пригодными для обучения.

Однако данные ОС имеют один существенный недостаток: они разработаны
под устаревшую архитектуру x86.

Таким образом разработка учебной операционной системы для более современной
архитекруты AMD64 является актуальной задачей.
