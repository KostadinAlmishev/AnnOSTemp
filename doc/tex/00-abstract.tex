% Также можно использовать \Referat, как в оригинале
\begin{abstract}

В данной дипломной работе рассмотрены вопросы разработки учебной
операционной системы.

В аналитическом разделе рассмотреты основы теории операционных систем. Рассмотрены
существующие операционные учебные системы. Рассмотрены сущесвтующие форматы исполняемых
файлов и средства виртуализации.

В конструкторском разделе описаны режимы работы процессора,
механизмы работы сегментного преобразования, страничного преобразования,
обработки прерываний, процесс перехода в длинный режим.

В технологическом разделе рассмотрены средства разработки и отладки, загрузка ОС,
организация памяти ОС. Описана реализация системных вызовов, многозадачности и
потоков ядра.

В организационно-экономическом разделе выполнено определение этапов работ.
Проведен рассчет трудоемкости выполняемых работ, определно количество исполнителей.
Проведены рассчеты по определению структуры затрат на разработку проекта, а также
произведено планирование цены ПО и прогнозирование прибыли.

В разделе охраны труда и экологии рассмотрены вредные факторы, влияющие на
пользователей при использовании ПО. Приведен рассчет воздухообмена в
компьютерном классе. Рассмотрен процесс утилизации пластмасс,
являющихся частью жидко-кристаллического монитора.

В заключении делаются выводы о результатах, достигнутых в ходе выполнения
данной работы.

\end{abstract}
